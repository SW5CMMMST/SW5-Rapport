%Author: Marc
\subsection{Adding devices to the network}\label{AddDev}
As it is wished for the network to be dynamic in size, the network must be able to add new devices to the network without manual interference.
For this to happen a network must first be aware that a device, that is currently not part of the network, has come in range of the network.
To discover this new device two possible approaches are available; either the network should ping out searching for devices, or a new device should ping the network in an attempt to join.
Each approach has its merits and its downsides which will be analysed in this section such that an informed choice can be made in the design phase.
In the case that the network handles the detection of new devices within its range, this would have to be part of the frame.

\bigskip
\noindent While there are many ways of defining how this frame may look, the basic concept is that a device should either ping for new devices on each time-slot, which is quite the overhead; ping on every Nth frame which increases worst case transmission time by an entire frame or maybe an entirely different alternative.
If the network pings for new devices, any new device receiving the ping should respond, which means that the time slots for pinging should be longer than regular transmission time slots.
What is worth noticing here is that no matter what way this is done, it increases the worst case transmission time.
On the other hand this also means there should be no complications appending the new device to the frame, assuming free time-slots are known by all units in the network.
Furthermore it removes the possibility of a device that is not part of the network creating noise for a device in the network.

Alternatively having the lone device ping the network achieves the opposite; by removing the management from the network the worst case transmission time is lower as the responsibility of joining the network lies with the outside device.
This also creates the issue of knowing when a device can ping the network without interfering with the communication in said network; as such a way of figuring out when a free time-slot lies, and how to inform the network that a new device has joined, is imperative for this to work without complications.
Depending on further analysis of the different aspects of the protocol a choice as well as further consideration and development of said choice is made in section <insert reference to design of this in design section here>.