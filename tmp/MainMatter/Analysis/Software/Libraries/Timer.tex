\subsubsection{Timer}
The library Timer is used to perform repeated actions in a given period.
There are functions to call a function repeatedly, to wait a given period and then call a function, to toggle the state of a pin repeatedly and finally to pulse a pin repeatedly. 
This can be useful as i.e. pulsing an LED can be performed without a call to the delay function and halting other functions of the program.
The library works by creating internal objects which will be updated every time a call to update is performed. 
Therefore it is necessary to call the update function often, for this to work. 
To stop an event one must save the id returned when the event was started and use this to call the stop method. 

The function prototypes used in this library is:
\begin{lstlisting}[style=customc]
  int8_t every(unsigned long period, void (*callback)(void));
  int8_t every(unsigned long period, void (*callback)(void), int repeatCount);
  int8_t after(unsigned long duration, void (*callback)(void));
  int8_t oscillate(uint8_t pin, unsigned long period, uint8_t startingValue);
  int8_t oscillate(uint8_t pin, unsigned long period, uint8_t startingValue, int repeatCount);
  int8_t pulse(uint8_t pin, unsigned long period, uint8_t startingValue);
  int8_t pulseImmediate(uint8_t pin, unsigned long period, uint8_t pulseValue);
  void stop(int8_t id);
  void update(void);
  void update(unsigned long now);
\end{lstlisting}
