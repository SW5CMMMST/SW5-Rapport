\chapter{Code for UPPAAL Model}\label{app:UPPAALCode}
In this appendix the code for the UPPAAL project can be seen.

\begin{lstlisting}[language={[GUI]Uppaal}, % use GUI flavor
columns={[l]flexible},
frameround=fftt, frame=shadowbox, rulesepcolor=\color{gray},
caption={Code for the global declarations.}]
// Place global declarations here.
clock now;
const int MAX_DEVICES = 4;
int N = 2;
const int S = 20;
const int maxS = 25;
const int minS = 20;
broadcast chan tx;
broadcast chan tick;
typedef int[0, MAX_DEVICES] id_t;
\end{lstlisting}


\begin{lstlisting}[language={[GUI]Uppaal}, % use GUI flavor
columns={[l]flexible},
frameround=fftt, frame=shadowbox, rulesepcolor=\color{gray},
caption={Code for the local declarations for Device.}]
// Input Parameters: const id_t id, bool connected

// Place local declarations here.
int i;
\end{lstlisting}

\begin{lstlisting}[language={[GUI]Uppaal}, % use GUI flavor
columns={[l]flexible},
frameround=fftt, frame=shadowbox, rulesepcolor=\color{gray},
caption={Code for the local declarations for Clock.}]
// Place local declarations here.
clock x
\end{lstlisting}

\begin{lstlisting}[language={[GUI]Uppaal}, % use GUI flavor
columns={[l]flexible},
frameround=fftt, frame=shadowbox, rulesepcolor=\color{gray},
caption={Code for system declarations.}]
// Place template instantiations here.
Dev0 = Device(0, true);
Dev1 = Device(1, true);
Dev2 = Device(2, false);
Clck = Clock();
// List one or more processes to be composed into a system.
system Clck, Dev0, Dev1, Dev2;
\end{lstlisting}