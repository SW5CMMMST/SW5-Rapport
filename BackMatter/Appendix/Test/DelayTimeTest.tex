\chapter{RadioHead Time Sent Test} % (fold)
\label{cha:radiohead_time_sent_test}
The purpose of this test is to determine how long it takes to send a message of a given length.
This information is useful at later points during a real-time analysis.
The result is therefor a formulae which given the message length outputs the time it takes to send said message. 

\section*{Theory}
As described in \myref{subsubsec:RadioHead} there is an overhead of 104 bits for each message, along with the 4-to-6 bit conversion.
The bit-rate is 2000 bits pr. second, which gives the theoretical minimum time it takes to send a message:
\begin{align*}
t(n,c)=c+\frac { 104 bit+n*\frac { 6 }{ 4 } *\frac { 8 bit }{ 1 byte }  }{ 2000\frac { bit }{ second }  } 
\end{align*}
Where \textit{n} is the length of the message in bytes, \textit{c} is some constant in seconds which is the processing of the message. This equation could result in the following examples:
\begin{align*}
t(10 byte, 0.01 second) = 0.122 second\\
t(20 byte, 0.01 second) = 0.182 second\\
t(30 byte, 0.01 second) = 0.242 second\\
t(40 byte, 0.01 second) = 0.302 second
\end{align*}

\section*{Test Setup}
The hardware setup used in this test is the same as the one used in the test presented in \myref{cha:radio_frequency_module_reception_test}.
However this test was only performed at the distance of 1 metre, but this should not affect the time it takes to send a message, as this is determined by the timer on the Arduino.

\subsection*{Software used in the test}
The software used on the Arduinos for this test was different than the other test in \myref{cha:radio_frequency_module_reception_test}. 
The full source code is in \myref{app:t2receiver} and \myref{app:t2transmitter}.
In the software the message lengths of 1 to 60 bytes was set as the range to be tested. This can be seen in the core of the transmitter code, which is as follows:
\begin{lstlisting}[style=customc]
void loop()
{
    driver.send((uint8_t *)msg, len);
    driver.waitPacketSent();
    len = (len % 60) + 1;
    delay(100); // In milliseconds 
}
\end{lstlisting}

The transmitter will send a message of length \texttt{len}, wait 100 milliseconds, then send a message of length \texttt{len + 1}, and redefine \texttt{len} to \texttt{len + 1}.
If \texttt{len} is 60 then the next value will be 1. 
There is placed a delay, of 100 milliseconds, in the end of the loop so the receiver has some time to process the received message; this will have to be subtracted when processing the test results.

The core of the receiver code is as follows: 
\begin{lstlisting}[style=customc]
void loop()
{
  uint8_t buf[RH_ASK_MAX_MESSAGE_LEN];
  uint8_t buflen = sizeof(buf);

  if (driver.recv(buf, &buflen)) // Non-blocking
  {
    Serial.print(buflen);
    Serial.print(", ");
    Serial.println(millis() - timer);
  
    timer = millis();
  }
}
\end{lstlisting}
The \texttt{millis} function returns the time in milliseconds since the Arduino was powered on. 
This code prints the length of a package received and the time, in milliseconds, since the last package was received.

\section*{Results}
\vspace {10 mm}
\figur{Graphs/time_length.pdf}{0.9}{Result of running the test.}{time_length}
These results are gathered over 800 data points, so each point is the worst-case of at least 10 samples. 
In this plot the 100 milliseconds delay as seen in the transmitter code is present.
The worst-case value for 1 byte is 171 milliseconds, and for 60 bytes it its 526 milliseconds. 
Linear regression was performed on the data collected which yielded the formulae $f(x)=6.0101 * x + 165.7826$ with $R^2 = 1$. 
The time it takes so send a message of length x bytes is therefore $f(x)=6.0101 * x + 65.7826$, when the 100 millisecond delay is removed.

\section*{Conclusion}
This test shows that minimising the message length has a linear impact on the time it takes to send a message. 
Furthermore the time it takes to send such a message can be estimated in milliseconds, using the formulae $f(x)=6.0101 * x + 65.7826$, where x is the length of the message in bytes. 
This result will impact the timing of the system as a whole.  
% chapter radiohead_time_sent_test (end)  