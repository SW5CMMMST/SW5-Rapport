\subsection{UPPAAL}\label{subsec:uppaal}
\begin{tcolorbox}[floatplacement=b,float,colback=white!5,colframe=aaublue!50,title=What is UPPAAL \cite{tutorial04}?]
\textit{``UPPAAL is a toolbox for verification of real-time systems jointly developed by Uppsala University and Aalborg University.} [...] \textit{The tool is designed to verify systems that can be modelled as networks of timed automata extended with integer variables, structured data types, user defined functions, and channel synchronisation.''}
\end{tcolorbox} 

UPPAAL have been used to find problems with and verify a wide range of protocols and applications. 
This includes and is not limited to: 
\begin{enumerate}[label=\itshape \alph*\upshape)]
    \item TDMA Protocol Start-Up Mechanism \cite{Lonn:1997:FVT:826040.827011}
    \item Bang \& Olufsen audio/video protocol \cite{Havelund97formalmodeling}
    \item Car supervision system \cite{gebremichael2004formal}
\end{enumerate}
The UPPAAL application is a GUI interface made in Java, and the verification is done with an engine written in C++. 
UPPAAL works by consisting of several models each of these is a timed automata, which is a finite-state machine with clocks. 
All clocks in the system progress synchronously. 
This means that a UPPAAL model consists of states and edges or transition, each of these edges can fire. 

In UPPAAL it is possible to make queries, to verify whether certain properties are true or not for a given model.
It has a special syntax, for example: \textbf{A[] not deadlock} which asks if for all states there is no deadlock.
\textbf{E<>} can be used instead of \textbf{A[]} and this asks if there exists a path of transitions so that a property is true.
One final example is \textbf{A<>} which express that for all paths the property will at some point in time be true.
