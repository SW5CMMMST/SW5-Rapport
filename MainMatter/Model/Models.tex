\section{UPPAAL Models}\label{UPPAAL_Models}
This section will specify what a UPPAAL model consists of, how it is constructed and how it works.

A UPPAAL model consists of one or more templates, which are timed automata, all templates can have local code, while it is also possible to create global code available to all templates, which are simply called declarations in the UPPAAL GUI.
Both the global and the local code, which is written in a language made for UPPAAL, can have variables and functions which can be called by instances of their respective templates.
The UPPAAL modelling language has certain special types like \texttt{clock} and \texttt{channels}.
A clock is what makes the automatas timed, and is a variable which can consist of natural numbers, it is continuously incremented to indicate that time is passing.
A channel is used to synchronise instances of templates so edges fire at the same time.
Instances of the templates are created in system declarations, which is used to setup the system for testing.
The templates all have an initial location like other automatas.
All locations and edges can have certain properties to control the transitions in the automata which is how the model for the protocol can be checked.

\bigskip\noindent
The properties of a locations are the following: 

\begin{description}[labelindent=\parindent]
    \item[Name]\hfill\\
    These are used to refer to certain locations from either the code for the model, or in the verification aspect of UPPAAL. It is also useful for documentation of the model. The color of names is dark red in all UPPAAL models.
    \item[Invariants]\hfill\\
    Are used to keep an instance of a template in a location until an expression no longer holds true. The invariants often make use of the clock type available in UPPAAL to make sure an instance of a template will leave a location after a certain time has passed. The color of invariants are a pink-ish color in all UPPAAL models.
\end{description}

\noindent
Other types of properties for locations are toggled variables which change how a location is computed:

\begin{description}[labelindent=\parindent]
	\item [Initial] states indicate where the template will start, and works just like in other state automata which has to have an Initial state. This is marked by a double circle on the state.
	\item [Urgent] states freeze time for the template, which means that all clocks for a template in an urgent state will not change. This is marked by a U on the state.
	\item [Committed] states also freeze time, but it does so for the entire system, as if any template is in a commited state, the next edge to be fired has to be an edge from a committed state. This is marked by a C on the state.
\end{description}

\noindent
Edges can have the following properties:

\begin{description}[labelindent=\parindent,labelsep=1ex]
    \item[Select] is used to non-deterministically give an identifier a value in a range, and is shown with a green-yellow color, such an edge is not present on the models in this chapter.
    \item[Guard] is used to make sure an edge can be used unless a certain expression is upheld. Guards are shown in a green color.
    \item[Sync] is used with the channel variables to make sure an edge is fired in all available templates. If an edge with the sync \texttt{chan!} is fired all templates which are in a location with a legal transition \texttt{chan?} has to be fired. This makes sure all templates make a transition when able to. Syncs are shown in a light-blue color.
    \item[Update] is used to change variables when an edge is fired, either to give them a certain value or maybe to increment or decrement variables. Updates are shown in a purple color.
\end{description}

\noindent
The rest of the chapter will verify that the protocol upholds the system requirements from \myref{requirements}, and the statements from \myref{sec:Pseudo}.
