\subsubsection{Limitations of Model-Checking}

It may not be possible to represent all variables a real system is affected by in a model-checker like UPPAAL.
It is not possible to guarantee that the protocol, the model and the application are equivalent, this means that even if the protocol and/or model is correct then the implemented application may not be. 
Deciding this is an undecidable problem within computer science, often referred to as $EQ_{TM}$, explained further in \textit{Introduction to the Theory of Computation} by Michael Sipser \citep[p. 220]{Sipser}.
Therefore some assumptions must be made. 
The first assumption for our project is that our protocol, model and application will be equivalent.
Other assumptions for the implementation of the protocol are the following: 
\begin{enumerate}[label=\itshape \alph*\upshape)]
    \item The network is completely connected
    \item Any transmission sent is also received by the other devices
    \item Only one new device is introduced to the network at a time.
\end{enumerate}
