\section{Model Driven Software Development}
Since reliability is the main focus points of the protocol it is essential that it is possible to test the model in-depth during development.

Model driven software development is one way to make this possible.
The general idea, as described by \citet{beydeda2005model}, is that the model is more abstract than its implementation.
Therefore a programmer could construct a model and have another program generate the code and any errors in the generated code is caused by the model itself or an error in the model-to-code compiler.
Changes in the generated code should therefore be avoided.

Making a model instead of an implementation makes it possible to test different ideas without worrying about the implementation details reducing the complexity of the task.

Since a model is more abstract than the implementation, it is feasible that another program could simulate devices using the protocol.
The result would be the possibility of testing a larger array of test-cases in a shorter period of time compared to tests using hardware.

In pure model driven software development, it would usually involve a generator which could generate code from the model. 
This is not the focus of this paper, therefore the implementation is done manually, from thoroughly designed models.
