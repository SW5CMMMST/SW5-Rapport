\chapter{Development Plan \& Problem Exploration}\label{chp:Problems}
This chapter will investigate how the protocol can be perceived mathematically as a graph, and how it can be divided into simpler subproblems which can be solved iteratively.

The requirements stated in the previous chapter, can be modelled as a graph problem where a weighted directed graph $G = (V, E)$ describes the network. 
The vertices, $V$, are the devices and the edges, $E$, are the communication paths between the devices. 
The weight, $W(v_1, v_2)$, is the probability of a message from $v_1$ successfully being received by $v_2$.
An example of this can be seen on \myref{fig:network}.
For this project only networks which can be depicted as a single complete or strongly graph will be considered, it order for all devies having the posiblility to excange data with every other device.
\myref{fig:network} is an example of a strongl graph network which can be described as: $E \subseteq V \times ]0,1] \times V$ where for any $v_i, v_j$ there exists a path from $v_i$ to $v_j$ with the reliability $r$ which is a number in the range $]0,1]$.

\tikzfigure{NetworkGraph.tex}{An example of a strongly connected unreliable network with probabilities modelled as a graph.}{network}

\noindent Creating a protocol for such a network is a complex problem to solve and many issues will become apparent throughout analysis of the problem.
In order to simplify the problem some assumptions are made about the network.
%We assume that the devices will always be in range of each other; this means that all vertices on the graph have a non-zero edge between all other vertices on the graph.
During development of the system, the system could be set up to give \textasciitilde100 \% probability of successfully receiving the message, by wiring the Arduinos together instead of using radio communication; this gives more room to work on the baseline communication issues without having to consider the unreliability of the radio frequency modules.
Asmwntioned before the networks can be set up as either complete graphs or strongly connected graphs.
This allows four sub problems which can be modelled as graphs:
\\
\begin{description}[labelindent=\parindent,itemsep=2em]
    \item[CCRC-problem:] Completely connected reliable communication problem   
    \begin{equation}
    \text{iff } \forall\, \{v_i, v_j\} \subset V: \, (v_i,r,v_j)\in E \mid r = 1    
    \end{equation}
    The CCRC-problem describes how all vertices are directly connected with each other, creating a completely connected graph, and all the weights of the edges are also 1, which means that all transmissions will be received. 
    
    \item[CCUC-problem:] Completely connected unreliable communication problem
    \begin{equation}
    \text{iff } \forall\, \{v_i, v_j\} \subset V: \, (v_i,r,v_j)\in E \mid r \in (0, 1]
    \end{equation}
    The CCUC-problem describes a completely connected graph however in this scenario there is not a \textasciitilde100\% chance the transmission will be received; all vertices are still in range of each-other. 
    
    \item[SCRC-problem:] Strongly connected reliable communication problem
    \begin{equation}
    \begin{gathered}
    \text{iff } \forall\, \{v_i, v_j\} \subset V: \text{ there is a directed path from } v_i \text{ to } v_j\; \land \\ \forall\, (v_q, r, v_r) \in E : r = 1
    \end{gathered}  
    \end{equation}   
    The SCRC-problem describes a strongly connected network meaning at least one vertex has an indirect connection to another rather than a direct connection, i.e. there still exists a path from every vertex to every other vertex but at least one pair of vertices requires one or more mediating vertices. In this example communication is once again reliable such that all transmissions will be received. 
    
    \item[SCUC-problem:] Strongly connected unreliable communication problem
    \begin{equation}
    \begin{gathered}
    \text{iff } \forall\, \{v_i, v_j\} \subset V: \text{ there is a directed path from } v_i \text{ to } v_j\; \land \\ \forall\, (v_q, r, v_r) \in E : r \in (0, 1]
    \end{gathered}  
    \end{equation}    
    The SCUC-problem describes the most realistic scenario, a strongly connected network with a chance of not receiving transmissions. 
\end{description}
\bigskip \noindent
The CCRC-problem is is the obvious chioce for the first iteration, since this is the simplest problem, which requires the least amount of design choices to be fulfilled.
Which problem should follow CCRC-solution, is however at more complex descison.
One could argue for expanding into a still reliable but now only strongly connected network (SCRC), as well as arguing for expanding into a unreliably but still completely connected network (CCUC).
Therefor the chosen order does simply reflect the our expations to which order will be the easiset.

Another possibity would be to only split the sub problems into three, e.g. CCRC $\rightarrow$ CCUC $\rightarrow$ SCUC, however we deem that a more optimal solution can de dereived by working with the strongly connected problem and the unreliabe problem and the at last combining the solutions and experince into when solving the SCUC.

In addition to these four sub problems some more pratical problems will be abstraced away for in the first iteration.
An example of such a abstraction could be starting multiple devies at the same time.
Solutions to these pratical problems will be implemented continually during the development. 
These abstractions will be furhter explained as the are taken during the design and implemtations phases of the problems.
This approach increases the probability of having a working solution at the end of this project without over- or underestimating the workload.