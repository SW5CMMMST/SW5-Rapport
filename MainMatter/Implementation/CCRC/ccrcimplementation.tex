\noindent
The implementation contains some code to help the test and a simple application. 
This simple application allows one Arduino to turn on LEDs connected to other Arduinos. 
This is to demonstrate that the protocol works, and can transmit information besides the header used in the protocol. 

First the data structures used for the implementation will be described followed by the mechanism for joining and creating a network and lastly the core loop which is run while connected to a network.
In the code examples some code used for debugging or testing have been removed as it is irrelevant for the working protocol, however the full code is available in \myref{app:CCRC_CODE}. 
\subsection{Data Structures}
The payload is grouped into two parts, the first part is the header necessary for the network to work, and the second is the data which the application can send.
The structs for \texttt{payload} and \texttt{payloadHead} are shown on \myref{lst:ccrc:payload} and \myref{lst:ccrc:payloadHead} respectively.
The devices must transmit the header in its own time-slot, but the data is optional.
This means that the transmission time can be shorter than the maximum time given.

\begin{lstlisting}[style=customc,caption={The payload the network uses.},label={lst:ccrc:payload}]
struct payload {
    struct payloadHead header;
    uint8_t data[13]; // So total size is 16 byte
};
\end{lstlisting}

\begin{lstlisting}[style=customc,caption={The header of the payload which the network uses.},label={lst:ccrc:payloadHead}]
struct payloadHead {
    uint8_t currentSlot;
    uint8_t slotCount;
    uint8_t address;
};
\end{lstlisting}

\noindent
The last important data structure used is the \texttt{networkStatus} struct, shown in \myref{lst:ccrc:networkStatus}.
This data structure is responsible for locally containing the status of the network. 
Note that the names used are the same as in \myref{sec:Pseudo}.

\begin{lstlisting}[style=customc,caption={The \texttt{networkStatus} struct, containing the status of the network locally.},label={lst:ccrc:networkStatus}]
struct networkStatus {
    uint8_t n; // number of timeslots
    uint8_t k; // the timeslot of this device
    uint8_t i; // the timeslot currently in progress
};
\end{lstlisting}

\subsection{Timing on the Arduino}
To acquire relative time on Arduino the following constructs are used: A global variable, a reset function and a get function. 
This construction can accurately time the length of time-slots and its constituents. 
\begin{lstlisting}[style=customc,caption={The variable and functions used to implement timing.},label={lst:ccrc:timing}]
unsigned long x = 0;

unsigned long getClock(unsigned long * x_0){
    return millis() - *x_0; 
}

void resetClock(unsigned long * x_0) {
    if(millis() + 1 < *x_0)
        Serial.println(F("Timer overflowed ..."));

    *x_0 = millis();
}
\end{lstlisting}

\subsection{Upstart: Creating or Joining a Network.}
When each Arduino starts it determine if there is a network in progress and join it. 
If it cannot find a network then it should create its own network, so that others can join it. 
The worst case time required to determine whether a network exists is roughly $3 \times \delta \leq t_0$ for the CCRC-problem, as determined in \myref{sub:setupCCRC}, where $t_0$ is the maximum time. 
This is referred to as \texttt{INIT\_WAIT} in the code and sets the limit for how long an Arduino listens before creating its own network. 
In \myref{lst:ccrc:startup} the startup code is shown. 

\begin{lstlisting}[style=customc,caption={Startup, if a network is found join it, if not create one.},label={lst:ccrc:startup}]
resetClock(&x);

foundNetwork = false;
while(getClock(&x) <= INIT_WAIT && !foundNetwork) {
    if(rx()) {
        Serial.println(F("Found Network, joining!!"));
        if(inPayload.header.slotCount > 1) {
            netStat.i = inPayload.header.currentSlot;
            netStat.n = inPayload.header.slotCount + 1;
            netStat.k = inPayload.header.slotCount - 1;
            setPayloadHead(&outPayload, netStat.i,  netStat.n, address);
            foundNetwork = true;
        }
    }
}

if(!foundNetwork) {
    netStat.n = 2;
    netStat.k = 0;
    netStat.i = 1; 
    outPayload.header.currentSlot = 0;
    outPayload.header.slotCount = 2;
    outPayload.header.address = address;
    Serial.println("Found no network, starting own");
}

resetClock(&x);
// Main loop begins now.
\end{lstlisting}

\subsection{Main Loop: Normal Network Operation}
The main loop executes both the processing phase and the communication phase.
Firstly each device should either transmit or receive depending on the current timeslot. 
If the current timeslot belongs to a device then that device should transmit otherwise it should receive, note that there is no special case for the empty slot in which new devices can join the network.
There exists no special case for a device to join as communication is perfect thus the device can simply transmit in the empty slot letting everyone know that the network has increased, and the device is seizing the former empty slot.
\begin{figure}
\begin{lstlisting}[style=customc,caption={Main loop deciding whether to transmit or receive.},label={lst:ccrc:rxortx}]
nextSlot(); // Increment the timeslot
if(netStat.i == netStat.k) {
    // Guard time
    delay(GUARD_TIME_BEFORE_TX);
    tx(NULL, 0);
    resetClock(&x);
} else {
    foundNetwork = false
    while(getClock(&x) <= TIMESLOT_LEN && !foundNetwork) {
        if(rx()){
            resync(); // Takes the infomation from the recived payload.
            foundNetwork = true;
        }
     }
     resetClock(&x);
}
\end{lstlisting}
\end{figure}

\bigskip \noindent
The \texttt{tx()}-function prepares and transmits the data.
It then resets the clock on the device and the main loop starts over.
The \texttt{rx()}-function asks Radiohead whether a full package has been received. 
If a full package has been received the package is then copied to \texttt{inPayload}, which is a global variable instance of the \texttt{payload} struct described earlier, lastly the function returns true.
Following \texttt{rx()} the while-loop on line 16 in \myref{lst:ccrc:rxortx} will break, causing the clock to reset and run the second part of the main loop.
This means that the communication part of the current timeslot is over after the transmission is complete.
Since it happens at the same time for both the transmitting device and the receiving device(s) then a synchronization of the clocks is obtained, this is based on the assumptions of CCRC. 
 
In the main loop user code is run for a maximum of $\delta_{proc}$ time, known as \texttt{DELTA\_PROC} in the code.
As aforementioned this example allows one Arduino to turn on an LED on another Arduino. 

\begin{lstlisting}[style=customc,caption={While-loop with application code.},label={lst:ccrc:usercode}]
bool runOnce = false;
while(getClock(&x) <= DELTA_PROC) {
    // LED-APPLICATION-CODE
}
\end{lstlisting}