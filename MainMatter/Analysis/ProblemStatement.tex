\section{Problem Statement}\label{sec:problemStatement}

Summary of the analysis

\bigskip

{\addtolength{\leftskip}{10mm}\addtolength{\rightskip}{10mm}\noindent\hrulefill\it

\noindent Is there a way, using a single frequency, to make multiple ATmega328P based devices with RF modules communicate in a network of variable size, while still being reliable and fast enough to be used for systems where this is critical? % - Claus' bud

% How can multiple ATmega328P based devices with RF modules use a single frequency to communicate in a network of variable size, while still being reliable and fast enough to be used for systems where this is critical? - Sørens bud

% How can multiple Arduino-like devices communicate over the same frequency without interfering with each others messages?
\noindent\hrulefill

\noindent How can a peer-to-peer network of Arduino-like devices with radio transceivers of a single frequency communicate, such that any devices can send messages to any other device in the network even while not directly connected to every other device, in a reliable and time-critical way? 

\noindent\hrulefill

\noindent How can multiple Arduinos each fitted with a 433 Mhz radio transceiver communicate with each other, without disrupting each others signals, while being able to reach all devices on the peer-to-peer network which can be of varying size? %- Søren F

\noindent\hrulefill

\noindent How can multiple small embedded systems equipped with a single radio frequency transmitter receiver communicate, such that any device connected to the network can communicate with any other device, whether the connection is direct or requires other devices to repeat the signal and allow for devices to leave or join the network without manual configuration while still maintaining efficient yet reliable communication? %- Marc

\noindent\hrulefill

}

\bigskip

% Explaination and ending of section
% One of such systems could be fire alarms.







