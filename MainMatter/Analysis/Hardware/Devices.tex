% Af: Troels

\subsection{Development Platform Selection}
To be able to develop an embedded system a development platform is needed. 
There are some strict requirements for this development platform (``need to have'') and some which are advantageous but not critical (``nice to have'').
The strict requirements must be upheld for a development platform to be usable in this project and are:
\begin{description}
\item[a)] able to communicate using a single radio frequency, or have a way of using additional hardware to do so;
\item[b)] the ability to debug and develop programs in a fast and reliable way;
\item[c)] a relatively low cost, within the constrains of the budget given by the institution;
and
\item[d)] an ability implement a real-time aspect.
\end{description}

Furthermore the following parameters would be advantageous for this project and e.g. the real world application hereof:
\begin{description}
\item[e)] low power consumption, and possibility for running on batteries;
\item[f)] small form factor of the device itself;
\item[g)] ability to minimise and customise into a production like product;
and
\item[h)] widespread adaptation for fuller documentation.
\end{description}

\todo{Virker imo lidt anti-klimatisk, vi lægger op til en masse godt og så bare vælger 1, 2, 3, vupti. Jeg var ellers med på ideen, men savner dog dybde og noget der ikke virker trivielt - Osama Bin Pepe - Forstår ikke helt hvad du mener ? - Spider Pepe}
After exploring the market the Arduino ATmega328P based development boards seems like an ideal choice.
Specifically the Arduino Uno, which is the most known model seems like a perfect fit.

Another notable candidate is the Raspberry Pi, which features a much faster CPU, HDMI output, two USB ports etc., however it falls short on the real-time aspect in conjunction with the ability to debug them. 
There does exist a version of the real-time operating system RTOS (Open Source Real Time Operating System) for the Raspberry Pi.
However each Raspberry Pi costs above 200 DKK each, this would only allow the project group to buy two devices, and this would severely limit the possibilities of the project.
An Arduino compatible board can be bought from many manufacturers for the low cost of 38 DKK, which means many Arduinoes can be obtained for the project.

Other notable candidates are Teensy, a Arduino compilable device in which size has been minimised.
This features all the same in almost all categories, there is neither a clear advantage of choosing this over the Arduino. 
Therefore it is decided that the Arudino will be the development platform for this project since it uphold the aforementioned requirements, while still being the cheapest platform.
The Arduino platform will be more thoroughly investigated in teh follow section.
\todo[inline]{Argumenterne ovenfor er lidt svage, måske vi skal bare skrive at vi har Arduinoerne til rådighed? Så vi kune skal købe Radio Moduler?}

\subsection{Arduino}
Arduino is a company which designs and produces open source hardware and software.
This means that anyone which would like to can copy their hardware design and make their own improvements or simply a copy.
The most widely known Arduino board is the Arduino Uno.
The software used on Arduinos will be explained in \ref{sec:software}.

\todo[inline]{History of Arduino?}
\subsubsection{Arduino Uno}
The Arduino Uno is a ATmega328P based micro-controller board.
It has female pins in which wires can be attached, most notably there are fourteen digital input/output pins (labeled pin 0 - 13), six analog pins (labeled A0 - A5), some of these pins have special abilities which will be explained later see \ref{subsubsec:arduino-uno-pins}.
Code can be uploaded through a USB connection, which can also power the Arduino, and provide a debugging interface for serial communication.
The total size of an Arduino Uno is roughly five times seven centimeters. 

The processor of the Ardunio is the ATmega328P.
It has 32 KB of flash memory, 2 KB of SRAM, and 1 KB of EEPROM, and clock speed of 16 MHz.
\todo[inline]{Pic?}