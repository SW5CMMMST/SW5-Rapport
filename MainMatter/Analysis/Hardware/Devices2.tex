% Af: Troels

\subsection{Development Platform Selection}
To develop an embedded system a development platform is needed. 
There are some strict requirements for this development platform and some which is advantageous but not critical.
The strict requirements must be upheld For an development platform to be used in this project.

The strict requirements are 
\begin{inparaenum}[\itshape a\upshape)]
\item able to communicate using a single radio frequency, or have a way or adding hardware to do so;
\item the ability to debug and develop programs is a fast and reliable way;
\item a relatively low cost, within the constrains of the budget;
\item an ability implement a real-time aspect.
\end{inparaenum}


Additionally the following parameters are advantageous
\begin{inparaenum}[\itshape a\upshape)]
\item low power consumption;
\item small size device;
\item ability to minimize into a production like product;
\item widespread adaptation for further documentation.
\end{inparaenum}

After exploring the market the Arduino ATmega328P based development boards seems like an idea choice.
Specifically the Arduino Uno, which is the most known model seems like a perfect fit.

Another notable candidates are the Raspberry Pi, which features a much faster CPU, HDMI, 2 USB ports etc., however it falls short on the real-time aspect in conjunction with the ability to debug them. 
There do exist a version of the real-time operating system RTOS (Open Source Real Time Operating System) for the Raspberry Pi.
However each Raspberry Pi costs above 200 DKK each, this would only allow the project group to buy 2 devices, and this would severely limit the possibilities of the project.

Other notable candidates are Teensy, a Arduino compilable device in which the size has been minimized.
This features all the same in almost all categories, there is neither a clear advantage of choosing this over the Arduino. 
\todo[inline]{Argumenterne ovenfor er lidt svage, måske vi skal bare skrive at vi har Arduinoerne til rådighed? Så vi kune skal købe Radio Moduler?}

\subsection{Arduino}
Arduino is a company which designs and produces open source hardware and software.
This means that anyone which would like to can copy their hardware design and make their own improvements or simply a copy.
The most widely known Arduino board is the Arduino Uno.
The software used on Arduinos will be explained in \ref{sec:software}.

\todo[inline]{History of Arduino?}
\subsubsection{Arduino Uno}
The Arduino Uno is a ATmega328P based micro-controller board.
It has female pins in which wires can be attached, most notably there are fourteen digital input/output pins (labeled pin 0 - 13), six analog pins (labeled A0 - A5), some of these pins have special abilities which will be explained later see \ref{subsubsec:arduino-uno-pins}.
Code can be uploaded through a USB connection, which can also power the Arduino, and provide a debugging interface for serial communication.
The total size of an Arduino Uno is roughly five times seven centimeters. 

The processor of the Ardunio is the ATmega328P.
It has 32 KB of flash memory, 2 KB of SRAM, and 1 KB of EEPROM, and clock speed of 16 MHz.
\todo[inline]{Pic?}