%author:        Marc
%1st review:    Troels
\subsubsection{Arduino Uno Pins}\label{subsubsec:arduino-uno-pins}
The Ardiuno Uno has a total of 20 pins related to input or output, all remaining pins relate to powering the device.
14 of the pins are digital I/O pins, whereof 6 can provide 8-bit \textbf{Pulse Width Modulation} (\textbf{PWM}) output using a predefined function called analogWrite(). %expand PVM?
PWM is a method of acquiring an analog like result using a digital resource.
Digital control creates a square wave by disabling and enabling the power to a pin.
analogWrite() takes a value between 0 - 255, this value determines how long the signal is on and off per cycle.

The six remaining pins relating to I/O are analog input pins.
Each digital pin has an internal pull-up resistor providing 20 - 50k ohm.
Furthermore some of the digital pins are in some way different from the rest.
Pin 0 and 1 on the board, also marked with RXD and TXD, are used for receiving and transmitting \textbf{transistor transistor logic} (\textbf{TTL}) serial data. %TTL


Pins 2 and 3 are interrupt pins, using the function attachInterrupt() to provide an external interrupt. %Described further once we handle Arduino functions?
As mentioned 6 pins can provide a PWM output, these are pins 3, 5, 6, 9, 10 and 11.
Pins 10 (SS), 11 (MOSI), 12 (MISO) and 13 (SCK) can be used to support \textbf{Serial Peripheral Interface} (\textbf{SPI}) communication, each providing one of the four logic signals. %Expand SPI?
SPI is an interface used for short distance communication providing full-duplex communication using a slave-master type architecture.
The four pins mentioned each relate to four logic signals the SPI bus uses.
SS, slave select, is an output from the master, MOSI, master output - slave input, is a message from master to slave, MISO, master input - slave output, is a message from slave to master, and lastly SCLK, serial clock, which is an output from master and is the clock which the slaves synchronise.
To use SPI communication one must implement the SPI library.
%TWI

Another type of communication is also supported by certain pins, pin A4 or SDA pin and pin A5 or SCL pin.
These two pins can support \textbf{Two Wire Interface} (\textbf{TWI}) communication using the Wire library. %Expand TWI?
TWI is a derivative of the I\textsuperscript{2}C serial bus, and is essentially the same as I\textsuperscript{2}C.
Like SPI this is a serial bus which implements a master-slave type architecture, however this allows for more masters.
Unlike SPI this is a two-wire rather than four-wire system only using a Serial Data Line (SDL) and a Serial Clock Line (SCL).
%I have no idea what this difference actually does