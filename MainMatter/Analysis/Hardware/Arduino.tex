%author:        Michno
%1st review:    Troels
\subsection{Arduino}

As Arduino has been chosen as the platform for the project, the platform will be further researched in this subsection.

\begin{tcolorbox}[floatplacement=b,float,colback=white!5,colframe=aaublue!50,title=The Birth of the Arduino]
In 2005 a group of students, at the Interaction Design Institute Ivrea in Italy, developed the Arduino board as a project, so there would be a cheaper alternative to the ``BASIC Stamp`` (another microcontroller-based development kit), used by the institute at the time, which roughly cost \$100. \cite{birthofarduino}
\end{tcolorbox}

\subsubsection{Resources}
When using the Arduino and another microcontroller there are some rather serious constraints on its memory and speed.
One of the most popular Arduino boards the ``Arduino Uno'' has 32 KB of flash memory and a mere 2 KB of SRAM, and the frequency of the microcontroller is only 16 MHz.
These constraints are in great contrast to the majority of today's programming practises where e.g. memory is seen more or less as in abundance. 
\subsubsection{Shields and modules}
A core concept of the Arduino and the environment around it, is the ability to extend and interact with the hardware through so called shields and modules.
These shields and modules can be input or output devices which interacts with the microcontroller on the Arduino though the digital or analogue headers on the Arduino board.
These headers can however be used for other purposes e.g. powering a LED or reading analogue and digital input such as button presses or various sensor.
\figur{arduino_shields.jpg}{0.5}{An Arduino UNO with a stack of two shields on top}{arduino_uno_shields}

\todo{Something about what shields and modules are available? - Claus}