%author:        Michno
%1st review:    Troels

\section{Arduino}\label{sec:arduino}
Arduino is a company which designs and produces open source hardware and software.
This means that anyone which would like to, can copy their hardware design and make their own improvements.
This also means that there are many copies of the hardware with no changes sold at at a cheaper price compatible with the free Arduino software.
The software used to program Arduino boards will be explained in \ref{sec:software}.

\begin{tcolorbox}[floatplacement=b,float,colback=white!5,colframe=aaublue!50,title=The Birth of the Arduino \cite{birthofarduino}.]
In 2005 a group of students, at the Interaction Design Institute Ivrea in Italy, developed the Arduino board as a project, so there would be a cheaper alternative to the ``BASIC Stamp'' (another microcontroller-based development kit), used by the institute at the time, which roughly cost \$100.
\end{tcolorbox}

\subsubsection{Arduino Uno}
The Arduino Uno is a ATmega328P based micro-controller board, with a foot print of about five by seven centimetres.
It has a total of 20 pins where most are digital input/output pins.
Most pins have some predefined functions which supports a variety of different communication standards, however none of these are relevant as this project will simply use pins for I/O.
Code can be uploaded through a USB connection, which can also power the Arduino, and provide a debugging interface through serial communication.

The ATmega328P processor, has 32 KB of flash memory, 2 KB of SRAM, 1 KB of EEPROM, and a clock speed of 16 MHz.
EEPROM is a small persistent memory section, where stored data is kept when the devices are turned off, more on this can be found on \cite{EEPROM}.

These specifications pose some rather serious constraints in respect to speed, memory and storage, and are in great contrast to the majority of today's programming practises where e.g. memory is seen more or less as in abundance.