%author:        Troels
%1st review:    Sass
\subsection{Radio Frequency Module} \label{rfmodule}
In order to achieve wireless communication between Arduino devices a Radio Frequency (RF) module is used.
%kun til Arduino?
The module used in this project communicates at 433 MHz; its datasheet specifies the range is 20 to 200 meters, depending on the input voltage, which ranges from 3.5 to 12 volts. 
The Arduino provides 3.3 volts and 5 volts output, and the 5 volts output will be used.
The maximum bandwidth is 4 kilobytes/secound. 
The signal is sent using \gls{ask}.
These values will be tested, to ensure the reliability of the information. 

The RF system consists of a transmitting module (tx) and a receiving module (rx), see \ref{fig:arduino_rf}.
Both have the possibility of connecting an antenna to it in order to increase the range of the device.
The transmitting module have three pins to connect to it.
The pins are VCC, meaning the power supply pin, the GND meaning ground, and DATA.
The DATA pin is turned on by having the same difference in voltage between GND and DATA as VCC and GND. 
The receiving module has the same pins as the transmitting module, however the DATA pins is being read from instead of transmitted to by the Arduino.

\figur{arduino_rf.jpg}{0.5}{The 433MHz RF receiver (left) and transmitter (right)}{arduino_rf}

% Af: Troels 
% D. 2015-09-17
% Review: Søren 21. 09.
\subsection{RadioHead}\label{subsec:RadioHead}

To use the RF module a library called RadioHead can be used. 
RadioHead transmits each message, using \gls{ask}, in the following way:
Each byte (8 bits) gets encoded as two 6 bit symbols, the 6 bit symbols will never contain 3 of the same signal in a row (high or low), this is to ensure that the receiver can decode the signal properly as it needs to differentiate the high and low signals.
When a message for example the string \enquote{Hello}, several steps aret taken, and they are as follows:

First a preamble consisting of the the binary pattern \enquote{1010} six times. 
This is so the receiver can synchronise its timer to the timer of the sender.
Then the pattern \enquote{111000} once, and then \enquote{101100}, and then the start symbol \enquote{101100111000}. 
This is the start code of the message. 
Every four bit from now is encoded as the six bit encoding mentioned before.
The next byte (effectively 12 bits) is the message length including itself and the check-sum bytes.
Then the message \enquote{Hello}, this doesn't need to be terminated as the length is known, has the size 5 bytes $(5 * 12 bits/byte = 60 bits)$.
Then a check-sum to ensure the integrity of the message, consisting of 24 bits after encoding.

\begin{table}[h]
\centering
\colorlet{shadecolor}{gray!40}
\rowcolors{1}{white}{shadecolor}
\caption{A table view of the content of transmitting \enquote{Hello} with RadioHead.}
\label{table:RH}
\resizebox{1\textwidth}{!}{
\begin{tabular}{lllll}
Name           & Purpose                               & Content                  & Size {[}bit{]} \\ \hline
Preamble pt. 1 & Syncronization                        & \enquote{1010} * 6       & 24             \\
Preamble pt. 2 & Syncronization                        & \enquote{111000}         & 6              \\
Preamble pt. 3 & Syncronization                        & \enquote{101100}         & 6              \\
Start symbol   & Indicate start of transmission        & \enquote{101100111000}   & 12             \\
Message length & To indicate the length of the message & 4to6(\enquote{01101000}) & 12             \\
Message        & The payload                           & 4to6(\enquote{Hello})    & 60             \\
Checksum       & To verify the integrity               & 4to6(\enquote{???})      & 12             \\
                
\end{tabular}
}
\end{table}

This brings the total size of the packet to 144 bit, while the message is 5 bytes or 40 bits.

The recommended bit-rate for the transceiver is 2000 bit/sec, this means the message will take at least:
\begin{equation}
\frac{144 bit}{2000 \frac{bit}{sec}} = 0.072 sec
\end{equation}

% Af: Troels 
% D. 2015-09-17
%Review Søren 21. 09.
\subsubsection{Timing} 
After calculating the contents of the package to be sent, the Arduino needs to write it to the RF transmitter. 
To time this the built in ``timer1'' on the Arduino is used, to trigger an interrupt with a relatively high consistency.
Every eighth interrupt the signal is changed to the next bit and written via a digital pin. 
On the receiving end, the signal is sampled eight times pr. period, if five or more of these eight is high then, it is declared a 0 bit, otherwise it is declared a 1 bit. % Dette er ikke en skrivefejl RH_ASK.cpp L. 619

\subsubsection{Amplitude Shift Keying}

As mentioned the RF module sends its signal using \gls{ask}.
\gls{ask} is a form of modulation where the digital input is a radio wave where the digital high (also known as ``1'') is a greater amplitude than the digital low (also known as ``0'').
This principle can also be used with light, here digital high would be a short pulse, and a digital low would be the absence of light. 
To use \gls{ask} the devices must be synchronised meaning that the time period of a single bit must be known by the receiver, in order for it to demodulate it. 
In \myref{fig:ask} is an example of how a radio wave would look like while using \gls{ask}, here the period for each signal is twice the peiod of the wave. \cite{ASKnFSK}

\tikzfigure{ASK.tex}{\gls{ask}: Digital values are signaled by the amplitude of the radio wave.}{ask}
