%author         Claus 
%1st review     Søren
%2nd review     Troels
\chapter{Time Division Multiple Access}\label{TDMA}

\gls{tdma} is a channel access method used for sharing the same radio frequency between two or more devices.
This method was first used in practice in 1991, for use in the cellular telephones, to increase the number of concurrent users on the telephone network, according to \citet{networkencyclopedia2013time}.

This method is a great starting point when designing the network of this paper.
In the following section further explanation of \gls{tdma} is given.

\section{How \acrshort{tdma} works}

If multiple radio-transmitters transmit on the same frequency at the same time while in range of each other, then interference occurs as can be seen on \myref{fig:rangediagram}. 
On the figure the device denoted by a square is in range of both the devices denoted by circles. 
If they broadcast on the same frequency at the same time then the square-device will not be able to decode either of the signals\cite{networkencyclopedia2013time, networkencyclopedia2013advanced}.
This is called a jam and should be avoided in all cases if possible.

\tikzfigure{RangeDiagram.tex}{The two devices (circles) communicate on the same frequency making the receiver (square) unable to read either signal.}{rangediagram}

\gls{tdma} addresses this problem by having the transmitters take turns transmitting their message in allocated time slots as seen on \myref{fig:tdmaoverview}.
The figure shows five devices which share the same frequency, and to do this they get a timeslot each in the frame.
A frame is the complete cycle of these timeslots.

\tikzfigure{TDMAOverview.tex}{An example of a time slot allocation for five devices on a single frequency by using \gls{tdma}}{tdmaoverview} %label: fig:tdmaoverview 

\noindent
Just like frames are divided into time slots; the following five parts are what the individual time slots consists of:

\begin{description}[labelindent=\parindent]
	\item[Guard time]\hfill\\ 
	A period where no important data is sent.
	This is here to make sure that devices not properly synced to the loop is less likely to corrupt the signal of other senders.

	\item[Sync]\hfill\\ 
	A possible additional delay before the signal depending on the distance to the receiver.

	\item[Control]\hfill\\
	One or more control values which determines the nature of the signal. 
	This could be message type, length of the message, reciver, ect.

	\item[Data]\hfill\\
	The main data of the message.

	\item[\acrshort{crc} check]\hfill\\
	The \gls{crc} is an error detecting code. 
	It is there to indicate that the message received was not corrupted. 
\end{description}
\bigskip
\noindent

The aim of these divisions is to isolate each device, so that no jam occurs, and make sure that issues due to noise is minimized. 