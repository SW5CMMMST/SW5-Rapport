\subsection{RadioHead Time Sent Test} % (fold)
\label{sub:radiohead_time_sent_test}
The purpose of this test is to determine how long it takes to send a package of a given length.
This information is useful at later points during the design and implementation phase. 
The result is a formula which given the payload length outputs the time it takes to send said payload. 

\subsubsection*{Theory}
As described in \myref[name]{subsubsec:RadioHead} there is an overhead of 104 bits for each payload, along with the 4-to-6 bit conversion.
The bit-rate is 2000 bits per second, which gives the theoretical smallest time it takes to send a payload:
\begin{align*}
t(n,c)=c+\frac { 104\ bit+n\times\frac { 6 }{ 4 }}{ 2000\ \frac { bit }{ sec }  } 
\end{align*}
Where \textit{n} is the length of the payload in bits, and \textit{c} is some constant in seconds which is the processing of the payload. This equation could result in the following examples:
\begin{align*}
t(80\ bits,\ 0.01\ sec) &= 0.122\ sec\\
t(160\ bits,\ 0.01\ sec) &= 0.182\ sec\\
t(240\ bits,\ 0.01\ sec) &= 0.242\ sec\\
t(320\ bits,\ 0.01\ sec) &= 0.302\ sec
\end{align*}

\subsubsection*{Test Setup}
The hardware setup used in this test is the same as the one used in the test presented in \myref[name]{cha:radio_frequency_module_reception_test}.
However this test was only performed at the distance of 1 metre, but this should not affect the time it takes to transmit a payload, as this is determined by the timer on the Arduino.

The software used on the Arduinos for this test was different than the other test in \myref[name]{cha:radio_frequency_module_reception_test}. 
The full source code is in \myref{app:t2receiver} and \myref{app:t2transmitter}.
In the software the payload lengths of 1 to 60 bytes was set as the range to be tested. This can be seen in a fragment of the transmitter code in \myref{lst:transtestcore}.

\begin{lstlisting}[style=customc,float,floatplacement=!h,caption={The core of the transmitter code.},label={lst:transtestcore}]
void loop() {
    driver.send((uint8_t *)msg, len);
    driver.waitPacketSent();
    len = (len % 60) + 1;
    delay(100); // In milliseconds 
}
\end{lstlisting}

The transmitter will transmit a payload of length \texttt{len}, wait 100 milliseconds, then transmit a payload of length \texttt{len + 1}, and redefine \texttt{len} to \texttt{len + 1}.
If \texttt{len} is 60 then the next value will be 1. 
There is placed a delay, of 100 milliseconds, in the end of the loop so the receiver has some time to process the received payload; this will have to be subtracted when processing the test results.
Code for the receiver can be seen in \myref{lst:recvtestcore}.
The \texttt{millis} function returns the time in milliseconds since the Arduino was powered on. 
This code prints the length of a package received and the time, in milliseconds, since the last package was received.

\begin{lstlisting}[style=customc,float,floatplacement=!h,caption={The core of the receiver code.},label={lst:recvtestcore}]
void loop() {
  uint8_t buf[RH_ASK_MAX_payload_LEN];
  uint8_t buflen = sizeof(buf);

  if (driver.recv(buf, &buflen)) { // Non-blocking
    Serial.print(buflen);
    Serial.print(", ");
    Serial.println(millis() - timer);
    timer = millis();
  }
}
\end{lstlisting}


\subsubsection*{Results}
\figur{Graphs/time_length.pdf}{0.9}{Result of running the test.}{time_length}
These results are gathered over 800 data points, so each point is the worst-case of at least 10 samples. 
In this plot the 100 milliseconds delay as seen in the transmitter code has not been removed.
The worst-case value for 1 byte is 171 milliseconds, and for 60 bytes it its 526 milliseconds. 
Linear regression was performed on the data collected which yielded the formulae $f(x)=6.0101 * x + 165.7826$ with $R^2 = 1$. 
The time it takes to transmit a payload of length x bytes is therefore $f(x)=6.0101 * x + 65.7826$, when the 100 millisecond delay is removed.

\subsubsection*{Conclusion}
This test shows that minimising the payload length has a linear impact on the time it takes to transmit a payload. 
Furthermore the time it takes to transmit such a payload can be estimated in milliseconds, using the formulae: 
\begin{equation}
f(x)=6.0101 \times x + 65.7826 
\label{eq:timeToSendFormular}   
\end{equation}
where x is the length of the payload in bytes. 
This result will impact the timing of the system as a whole.  
% chapter radiohead_time_sent_test (end)  