%original @ Marc
\subsection{Protocol Data Transfer} %Working Title
When developing a protocol for data transfer one must consider the importance of different aspects; evaluate what aspect is more important, such that if they conflict one has priority over the other.
For the data transfer part of the protocol two primary aspects to consider are timely delivery and accurate delivery.
Timely delivery refers to the idea of data being transferred with a consistently high transfer speed, whereas accurate delivery refers to assuring that all data arrives and is in order.
These two aspects of data transfer conflict with each other.
To ensure that the data is intact, will arrive, and is in order; the unit sending the data must ensure that the unit receiving data is available. 
Furthermore the unit receiving must check that all data arrived and is ordered.
This process is a time-constraint upon the data transfer process, and should the system heavily rely on meeting certain deadlines this time-constraint may be too expensive for the system to handle.

%Two existing protocols that represent each of these aspects would be \textbf{User Datagram Protocol} (\textbf{UDP}) and \textbf{Transmission Control Protocol} (\textbf{TCP}).
%TCP guarantees reliable, ordered and error-checked data transfer in its protocol.
%This is done through first establishing that the connection is valid before sending data and then spending time letting the receiving unit recognise that all data is received.
%The receiving unit knows this through flags and acknowledgement bits in the data packages.
%UDP in contrast to TCP completely disregards certainty of delivery.
%UDP provides no guarantee of data integrity, nor does it perform any sort of handshaking with the receiving unit prior to sending data.
%As such UDP is provides no real reliability but instead lowers its latency, which in turn is better for timely delivery.