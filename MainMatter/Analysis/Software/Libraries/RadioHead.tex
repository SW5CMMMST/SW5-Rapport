% Af: Troels 
% D. 2015-09-17
% Review: Søren 21. 09.
\subsection{The RadioHead Library}\label{subsubsec:RadioHead}

In order to successfully use the \gls{rf}-moduless some way of synchronising and encoding the transmissions should be used, to reduce faults like package loss.
A library named RadioHead does exactly that, and this section explores how it works.
The official specification and documentation of RadioHead can be found at the official site \cite{2015ArduinoRadioHead}.

RadioHead transmits each message, using \gls{ask} as described in \myref[name]{subsub:ask}, in the following way:
Each byte (8-bits) is encoded as two 6-bit symbols, such that the 6-bit symbols will never contain more than three of the same signal in a row (high or low).
This is to ensure that the receiver can decode the signal as it needs to differentiate the high and low signals.
When a message like the string \enquote{Hello} is to be transmitted several steps are taken and they are as follows:

To start off a preamble consisting of the binary pattern \enquote{1010} six times. 
This is so the receiver can synchronise its timer to the timer of the transmitter, and the receiver can calibrate its sensors to the high and low value of the message.
Then the pattern \enquote{111000} once, and then \enquote{101100}, and then the start symbol \enquote{101100111000}. 
This is the start code of the message. 

Every four bit from now is encoded as the six bit encoding mentioned before.
The next byte (effectively 12 bits) is the message length including itself and the checksum bytes.
Then the message \enquote{Hello}, this doesn't need to be terminated as the length is known, has the size 5 bytes $(5 \times 12\ bits/byte = 60\ bits)$.
Then a checksum is appended to the message to validate the integrity of the message, consisting of 12 bits after encoding.

\begin{table}[ht]
	\centering
	\rowcolors{2}{GoogleBlueGrey!50}{white}
	\begin{tabularx}{\textwidth}{l X l r}
	Name & Purpose & Content & Size [bit] \\\midrule
	Preamble pt. 1 & Synchronisation & \enquote{1010} * 6 & 24 \\
	Preamble pt. 2 & Syncronization & \enquote{111000} & 6 \\
	Preamble pt. 3 & Syncronization & \enquote{101100} & 6 \\
	Start symbol & Indicate start of transmission & \enquote{101100111000} & 12 \\
	Message length & Indicate the length of the message & 4to6(\enquote{01101000}) & 12 \\
	Message & The payload & 4to6(\enquote{Hello}) & 60 \\
	Checksum & To verify the integrity & 4to6(\enquote{???}) & 12 \\
	\end{tabularx}
	\caption{A table view of the content of transmitting \enquote{Hello} with RadioHead.}
	\label{table:RH}
\end{table}

\noindent
This brings the total size of the packet to 144 bit, while the message is 5 bytes or 40 bits.
The recommended bit-rate for the transceiver is 2000 bit/sec, this means the message will take at least:

\begin{equation}
\frac{144\ bit}{2000 \frac{bit}{sec}} = 0.072\ sec
\end{equation}

\noindent%
One thing to note is that when multiple transmissions are received at the same time, i.e. when two transmitters within proximity of each other transmit at the same time, any receiver receiving both transmissions, will when using Radiohead not know that anything was transmitted.
This means that it is not possible when using RadioHead with our \gls{rf} to detect a collision of transmissions since a collision will result in jamming the signal.

% Af: Troels 
% D. 2015-09-17
%Review Søren 21. 09.
\paragraph{Timing} \hfill \\
After calculating the contents of the package to be sent, the Arduino writes it to the \gls{rf} transmitter. 
To time this the built in timer, \enquote*{timer1}, on the Arduino is used to trigger an interrupt.
Every eighth interrupt the signal is changed to the next bit and written via a digital pin. 
On the receiving end the signal is sampled eight times pr. period if five, or more of these eight are high then it is declared a 0-bit; otherwise it is declared a 1-bit. % Dette er ikke en skrivefejl RH_ASK.cpp L. 619
This is because the process of sending the signal inverts the value.
