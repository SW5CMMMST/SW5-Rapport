\chapter{Conclusion}

The work in the project resulted in a protocol where each device takes turns to communicate in a cyclic manner with an empty slot where new devices can join.
This is the main idea of the protocol.
Other than that the protocol includes mesures to assure that the intended message was recieved regardless of ureliability.

The protocol does not require more than one channel of transmission as all communication is done in turn.
This is first part of the problem statement as stated in \myref{sec:problemStatement}.

The second part is the \enquote{...in a reliable and time-critical way?}.
This is a trade-off, you can increase the reliability by using more time to ensure that the messages are valid.
This project focused on the reliability since much of the time aspect could be improved later on by optimizing on the protocol.

The requirements will also vary from system to system so by allowing the user to determine the desired balance would satistfy this problem.

This work could be a step towards making an easier way to setup local networks.
For example imagine a world where when taking a device home the setup process would be as easy as powering it on.
Then it would figure out what todo about the local network.
This would not only be useful in home automation, but also in other systems such as the main protagonist of this paper; the fire alarm system.
