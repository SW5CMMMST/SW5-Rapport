\section{Additional Performance Improvements}
The following section will briefly explain other thoughts on how to improve the performance of the protocol.
\subsection{Timer Interrupt Based Library}
The Arduino platform includes support for timer interrupt based programming.
This means that once for every time-slot an interrupt in the user program could occur.
Such an interrupt could handle protocol specific details, such as transmitting, receiving and maintenance.

The main advantage of this would be that the interrupt provides a simplified way of ensuring that user code does not exceed the time partitioned for it to run.
The disadvantage of this would be that the interface from the usercode to the protocol would become less obvious, and it would make the code vulnerable to race conditions.
Moreover the user code would need to be preempt-able and context switching would have to be implemented correctly.

Another thing to note is that on the Arduino Nano and Uno there are three timer interrupts.
One is used by the delay function another is used by the RadioHead library leaving exactly one left for the protocol.
This would limit every other library or usercode which uses timers to be incompatible.
A solution could be to alter RadioHead and the protocol to share interrupts, which is anticipated to require a lot of changes to both RadioHead and the protocol.

\subsection{Improving the Protocol's Operating Speed}
The protocol uses approximately $66 + (6 \times s)$ milliseconds to transmit a package of size $s$, as discovered in  \myref[name]{sub:radiohead_time_sent_test}. 
This means that a payload of 16 bytes would take 162 milliseconds to transmit.
If a quarter of the time slot is left for the usercode the length must be approximately 200 milliseconds.
In a network with $k$ devices, this means that the frame length must be equal to $200 \times (k + 1)$ milliseconds.
Due to which a network of four devices would have a frame length of one second.
There is a wide range of ways to improve the speed at which the protocol operates, but this section will focus on the following possible improvements:

\begin{description}[labelindent=\parindent]
    \item [Skip unnecessary transmissions]\hfill\\
When the protocol is fully functioning there is a chance that a device will not have anything new to transmit in its time-slot.
Currently the device would transmit the previous payload again filling the slot with information that might be redundant.
However some kind of signal that indicates to the other devices that this slot should be skipped is needed to implement this idea.
The signal could be in a device's previous times-slot, indicating that it would not have anything to transmit for some time. 
Another solution could be variable time-slots lengths, such that a a device could occupy a shorter time-slot if it has nothing to transmit. 
Both solutions however severely complicates the protocol, especially in regard to new devices joining the network as they will not know exactly when to announce their presence.

    \item[Increase the bit rate]\hfill\\ 
An obvious way to increase communication speed is to increase the bit rate.
This could introduce unreliability for the communication at higher bit rates, but even for smaller increases to the bit rate it could be a significant improvement.
The best way to implement this would be to have the user be in control of the bit rate.
This way the user could set the rate according to the requirements of a given application.
However one must adhere to the limitations of the hardware.
\end{description}

