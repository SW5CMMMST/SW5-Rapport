%\clearpage
\section{Simultaneous Connect}
As described in \myref[name]{sec:simultaneous_connect} the protocol handles the difficulties that arise with respect to simultaneous connect by trying to connect in the empty slot, listen for validation and wait an exponential backoff if validation is not acquired and then try again.

This section will present an alternative solution which is designed to better handle when many devices tries to connect simultaneously.
To improve the likelihood of a device connecting in the first frame the idea of sub slots is introduced and explained in the following text.

\subsection{Introduction of Sub Slots} % (fold)
\label{sub:introduction_of_sub_slots}
To ensure a faster solving of conflicts if multiple devices try to connect simultaneously, a two-step process is designed.
These two steps consist of choosing a random offset of sub slots to wait in the empty slot, along with a strategy of exponential backoff if jamming occurs, as used in the design in \myref[name]{sec:MSI-CCRC}.
The first step is explained further in \nameref{roff} \vpageref{roff} and \nameref{ssub:sub_slots} \vpageref{ssub:sub_slots}.
On \myref{fig:pseudo_flowMultiConnectimp} it can be seen how said two-steps is merged into the existing flowchart, note the version from \myref{sub:setupCCRC} is used to better express the idea of the changes.% flow chart (\myref{fig:psuedo_flowCCRC}).

\begin{figure}[p]
    \centering \footnotesize
    \tikzstyle{decision} = [diamond, draw, fill=yellow!20, 
    text width=4.5em, text badly centered, inner sep=3pt]
\tikzstyle{block} = [rectangle, draw, fill=green!40, 
    text width=6em, text centered, rounded corners, minimum height=4em]
\tikzstyle{line} = [draw, -latex']
\tikzstyle{cloud} = [draw, ellipse,fill=red!20,
    minimum height=2em, align=center]
\tikzstyle{preproc} = [draw,rectangle split, rectangle split horizontal,rectangle split parts=3, fill=green!40, minimum height=4em] 
%\resizebox{1\textwidth}{!}{
\begin{tikzpicture}[node distance = 2.8cm, auto]
    % Place nodes
    \node [cloud] (start) {Start};
    \node [block, right of=start] (search) {Search for network};
    \node [decision, below of=search] (found) {Found network?};
    \node [preproc, right of=found, node distance=4.5cm] (init) {\nodepart{two} Initialize network};
    \node [block, below of=found] (listen) {Listen while waiting for empty slot};
    \node [block, right of=listen] (choosesubs) {Choose which sub slot to take};
    \node [block, right of=choosesubs] (listensubs) {Listen in the prior sub empty slots};
    \node [decision, below of=listensubs] (heardinsub) {Heard announce?};
    \node [block, below of=listen] (waitempty) {Wait rest of the empty slot};  
    \node [decision, below of=listen, node distance=5.6cm] (validated) {Validated?};
    \node [block, right of=validated] (listenvalidate) {Listen for validation from the network};
    \node [block, right of=listenvalidate] (announce) {Announce in sub empty slot};
    \node [block, left of=validated] (expbackoffcalc) {Determine an exponential backoff}; 
    \node [block, left of=listen] (backoff) {Wait for back-off amount of frames}; 
    \node [block, below of=validated] (join) {Set network values $k \leftarrow n$  $n \leftarrow n+1$}; 
    \node [preproc, left of=join] (mainloop) {\nodepart{two} Main loop};

    % Draw edges
    \path [line] (start) -- (search);
    \path [line] (search) -- (found);
    \path [line] (found) -- node {no} (init);
    \path [line] (found) -- node {yes} (listen);
    \path [line] (listen) -- (choosesubs);
    \path [line] (choosesubs) -- (listensubs);
    \path [line] (listensubs) -- (heardinsub);
    \path [line] (announce) -- (listenvalidate);
    \path [line] (listenvalidate) -- (validated);
    \path [line] (validated) -- node {no} (expbackoffcalc);
    \path [line] (validated) -- node {yes} (join);
    \path [line] (heardinsub) -- node {yes} (waitempty);
    \path [line] (heardinsub) -- node {no} (announce); 
    \path [line] (expbackoffcalc) -- (backoff);
    
    \node [draw=none, below of=search, node distance=0.60cm] (seap){};
    \node [draw=none, left of=seap, node distance=1.05cm] (seapp){};
    \path [line] (backoff) |- (seapp);

    \path [line] (join) -- (mainloop); 
    \path [line] (waitempty) -- (listen); 
\end{tikzpicture}                                
%}
    \caption{Revised flow diagram showing how a device acts during the Initialization phase}
    \label{fig:pseudo_flowMultiConnectimp}
\end{figure}

The flowchart in \myref{fig:pseudo_flowMultiConnectimp} shows that the random offset of sub slots is the first method of prevention for the issue of simultaneous connection of multiple devices.
If a device hears another announcement before it has sent its own in the chosen sub slot, it will abort the announce-phase and wait for the next empty slot before trying immediately again.
However if a device announces it self and does not get validated by the network it will use exponential backoff to prevent further collisions as the most likely reason would be jamming.
The following paragraphs will address the aforementioned concepts of random offset and sub slots as well as an altered payload which aims to decrease the length of the empty slot.

\subsubsection*{Random Offset}\label{roff}
When a device is announcing itself in the empty slot, it randomly chooses an offset for when to send the announcement payload.
The random offset is not defined by time but sub slots. 
In the rest of the empty slot it listens for other possible devices announcing themselves.
If such a message is heard it does not attempt to announce itself in a later sub slot of the empty slot.
This method implies that the length of empty slot is a multiple of the time it takes to transmit the announcement payload.

\subsubsection*{Announcement Payload}\label{apay} % (fold)
\label{ssub:announcements}
To minimise the time it takes to announce oneself to a network, a special announcement payload is introduced.
This payload only needs one field namely the unique identifier or address of the device which is sending it.
By shortening the payload significantly from the regular size of at least five fields, the header fields, the time it takes to transmit the payload is decreased.
% subsubsection announcements (end)

\subsubsection{Sub Slots} % (fold)
\label{ssub:sub_slots}
To avoid using the potentially time consuming exponential backoff, if a multiple of devices try to connect simultaneously, the empty slot will be partitioned into smaller sub slots, see \myref{fig:frame_wsubslots} for illustration. 
The length of a sub slot should be slightly longer than the time it takes to transmit an announcement payload; it is important not to have overlapping announcements since none of them would be heard due to jamming.

When determining how many sub slots the empty slot should be partitioned into, it should be considered that more sub slots will increase the overall frame length.
Because of this, the number of partitions should be decided by the programmer implementing the protocol.

\begin{figure}[h]
    \centering \footnotesize
    \resizebox{1\textwidth}{!}{%
\begin{tikzpicture}[scale=\textwidth, node distance = 0cm]
\node[draw, align = center, 
        minimum width=0.33\textwidth, 
        minimum height=10mm] 
    (slot0)
    {$Slot_1$ (occupied)};
\node[draw, right=of slot0,
        minimum width=0.33\textwidth, 
        minimum height=10mm]
    (slot1)
    {$Slot_2$ (occupied)};
\node[draw, right=of slot1,
        minimum width=0.33\textwidth, 
        minimum height=10mm,
        rectangle split, 
        rectangle split horizontal,
        rectangle split parts=3,
        rectangle split draw splits=false]
    (slot2)
    {$Sub\ Slot_1$ \nodepart{two} $Sub\ Slot_2$ \nodepart{three} $Sub\ Slot_3$};
    
\draw [decoration={brace, mirror, amplitude=+20pt}, decorate]
    (slot1.south west) -- (slot1.south east) node [black,midway,below=+21pt] 
    {Regular time slot};
\draw [decoration={brace, mirror, amplitude=+20pt}, decorate]
    (slot2.south west) -- (slot2.south east) node [black,midway,below=+21pt] 
    {Partitioned empty slot};
\draw [decoration={brace, amplitude=+25pt}, decorate]
    (slot0.north west) -- (slot2.north east) node [black,midway,above=+26pt] 
    {Frame};
\draw [dashed] ($(slot2.north)!0.33!(slot2.north west)$) -- ($(slot2.south)!0.33!(slot2.south west)$);
\draw [dashed] ($(slot2.north)!0.33!(slot2.north east)$) -- ($(slot2.south)!0.33!(slot2.south east)$); 
\end{tikzpicture}
}%
    \caption{A frame with two connected devices and a three-way partitioned empty slot}
    \label{fig:frame_wsubslots}
\end{figure}


This method would in most cases perform faster connection of multiple new devices than the solution in \myref[name]{sec:simultaneous_connect} however when comparing the added value in regards to the added cost, the simpler method from \myref{chap:MDA-CCRC} is the better choice.