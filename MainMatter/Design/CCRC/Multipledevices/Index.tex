\chapter{Design of Multiple Device Activation}\label{chap:MDA-CCRC}
So far the \gls{ccrc}-solution have been unable to handle multiple devices being activated at the same time.
Therefore will this chapter explore a design which will make it possible to activate multiple devices at the same time.
The complications this brings will be solved in two sections; one pertaining to the issue that occurs if no network exists i.e. creating of multiple networks, and another pertaining to the issue that occurs if a network already exists i.e. multiple devices simultaneously attempting to connect to an established network.
Firstly a key concept used for solving the multiple device activation problem will be presented.
Following that the different parts of the problem will be discussed and the needed modifications to the existing design will be revealed.

\section{Exponential Backoff}
\label{sec:exponential_backoff}
The method of exponential backoff described here is based on the ethernet standard, see \citep{Ebackoff}, and consists of:
\begin{enumerate*}[label=\itshape \alph*\upshape)]
    \item remembering how many failed attempts of some action a device has made; and
    \item choosing a random number of frames to wait based upon this number.   
\end{enumerate*}
These properties of the exponential backoff can be expressed as follows:
\begin{equation}
     b \in [0, 2^c - 1] \text{ where } c \in \mathbb{N} \text{ and } {c \leq c_{max}}.
\end{equation}

\noindent
Where $b$ is a randomly chosen value from the range, and $c$ is the number of attempts starting at 1. 
The value $b$ can represent a number of frames or any other feasible time period.
With this method a device will wait 0 to 1 time periods after the first failed action, 0 to 3 after the second, 0 to 7 after the third and so on. 
To avoid an infinite increase of the wait time the $c$ should have an upper limit $c_{max}$, which would ensure a maximum wait of $2^{c_{max}}-1$ time periods.

\section{Multiple Startup Issue}\label{sec:MSI-CCRC}
%The activation of multiple devices at any point in time such that two or more devices are active but not in a network causes a problem that must be handled differently for each of the two aforementioned cases.
The first case, handled in this section, is in the case where no network exists prior to the activation of devices.
As it has been established earlier, the time required to guarantee no network exists is $\delta \times 3$.
From the moment a device has determined that no network exists, it spends $\delta$ time to announce the existence of a network.
As such if several devices are activated within $\delta$ time, relative to the firstly activated device, multiple networks will be established.
By this it follows that if one can guarantee that devices start with an offset of at least $\delta$ the problem is solved for every case.
In an effort to solve this problem several different approaches can help, however none of them can guarantee that a multiple of networks will never occur with 100 \% certainty. 
As such implementing concepts from several of the following approaches may yield the best result.

\subsection{Unique Offset}
Unique offset is when each device listens for a network for a uniquely specified amount of $\delta$ times time rather than the initial wait time of $\delta \times 3$.
Using the ID of each device one could guarantee that no device used the same multiple of $\delta$ prior to establishing their network by the formula $\text{\textit{initial wait time}} + \delta \times ID$.
This however assumes device IDs are activated continuously for it to work.
In the case where initially the device with ID 2 is turned on and exactly $\delta$ time later the device with ID 1 is turned on they would still both create a network, as such this is not a valid solution in itself to the problem.
However this approach could significantly increase the chance of only creating only one network in situations where all the devices are started simultaneously, e.g. a scenario where a multiple of devices are connected to the same power grid, which is then turned off and on again.
Hereby making it appropriate for a uniform startup process.

\subsection{Randomly Create}\label{RCreate}
Once a device is done listening for a network it either creates its own network or resets and starts listening again.
For this each device would need to implement a chance factor to determine whether or not to create a network.
This could be determined by running an algorithm on the result of a random seed, depending on how the algorithm is designed one can obtain whichever chance is preferred.
The problem with this is while it gives the opportunity for two devices to start within the same time-slot without both creating a network at the same time, this is no guarantee; as such this would further require the implementation of a recovery method much like unique offset would.

\subsection{Kill the network}\label{KtN}
If a device is alone in a network for too long it may decide to kill the network and start listening again.
This works as a recovery technique, allowing two networks to be established but then solving that very problem.
The technique has the network disassemble itself once some factor determines it needs to.
This can be done either deterministic or non-deterministic.
For the non-determinism it would work similarly to random create, except the outcome would be killing the network rather than creating one.
For determinism some variable should be incremented until it reaches a limit where it forces the network to kill itself.
For this to actually work there would have to be implemented something which guarantees that networks would not follow the same pattern, as that could result in a loop of killing and creating networks.
Another problem that occurs is if more than two networks are created, for it to work all but one network must die within two frames due to the previously set time spent listening.

\subsection{Exponential variables} 
Using the idea of an increasing exponent, similarly to the technique in exponential backoff as described in \myref{sec:exponential_backoff} can be applied to randomly create, killing the network and unique offset to increase odds of success.
\subsubsection*{Randomly Create}
For the idea of using an exponent in randomly create, the chance of creating should start relatively low.
Starting with a lower chance the algorithm would for each unsuccessful attempt of creating a network increase the chance.
In an effort to end up with different chances for each device trying to create a network, the algorithm would have to work on a unique value such as the address of each device.
While this does not remove the chance of two networks still starting at the same time, it does reduce the chance as one device will have better odds of creating than any other device turned on in the same moment.
\subsubsection*{Kill the network}
Once again the non-deterministic solution here would be implemented similarly to randomly create.
For the deterministic approach the unique factor chosen would be altered exponentially.
A complication in doing so would be that networks would be active for exponential lengths of time, and with the specific requirement of networks having to die within two frames it could worsen the time it takes to successfully establish a single network exponentially; for this reason one might consider using another factor to guarantee unique outcomes for each device, such as an exponential change in listening time which might also be uniquely offset, which also solves the problem with the two frames limitation.

\subsection{Modifications to the Upstart phase}                 
Based on the aforementioned possibilities a new upstart phase have been designed which minimises the probabilities of devices creating several networks.
The changes are explained in the following text, as well as in the flowchart in \myref{fig:pseudo_flowMultiStart} representing the new design.

\begin{description}[labelindent=\parindent]
    \item[Search for network]\hfill\\
    The initial action after starting a device states to search for a network.
    This is done for $\text{\textit{initial wait time}} + \delta \times b$ where $b$ denotes a random number chosen using the method of exponential backoff from \myref{sec:exponential_backoff}, i.e. $b \in [0, 2^c-1]$, where $c$ is the number of attempts of creating a network.
    This expression ensures that devices will not listen for the same amount of time when they are started, and this difference will increase with each attempt, ensuring the existence scenarios in which only one device will create a network while others are still listening, also in the case where they died at the same time.
    \item[Chance of creating a network]\hfill\\
    This method will not be used in order to keep the method of handling multiple startup deterministic.
    \item[Alone in network?]\hfill\\
    For this decision the device checks whether or not it is alone in the network as if it is, an altered version of \texttt{mainLoop()} is run, making use of the kill network method depending on the result of \textbf{Time to Die?}, as two or more networks could be jamming each other.
    \item[Time to Die?]\hfill\\
    This decision is completely random based on a given chance of killing the device.
\end{description}

\tikzfigure{PseudoFlowDiagramMultiStart}{Revised flow diagram showing how a device acts during the Initialization phase if no networks are detected.}{pseudo_flowMultiStart}


\subsection{Changes to the Pseudocode}
The previously introduced pseudocode was written with the assumption that only one device would be turned on at a time due to which it does not work when this assumption is removed.
As such new pseudocode must be written for those areas this affects; in \myref{lst:setupCCRC} only five lines handle the creation of a new network.
As is shown on the flowcode model \myref{fig:pseudo_flowMultiStart} the multiple network establishment issue is handled with an altered main loop, as such much of the new pseudocode will resemble  pseudocode from \myref{lst:general_case}.
The procedure seen in \myref{lst:networkMultiStartCCRC} replaces lines 34 through 38 in \myref{lst:setupCCRC} which are the five aforementioned lines.
\begin{figure}
\begin{lstlisting}[label=lst:networkMultiStartCCRC,style=pseudocode,mathescape=true,caption={Pseudocode example of the procedure initializeNetwork() for CCRC for the multiple device activation problem.}]
procedure initializeNetwork()
    // Initialize variables
    $x \leftarrow 0$
    $n \leftarrow 2$
    $k \leftarrow 1$
    $i \leftarrow 0$
    $f \leftarrow 0$
    $C \leftarrow 10$
    if $firstNetwork \neq \bot$
        $a \leftarrow 0$
    
    loop forever       
        run userCode() until $x \geq \delta_{proc}$
        $i \leftarrow (i \text{ mod } n) + 1$ // Update current slot
        if $i = k$ then
            makePayload() //Updates the data to be sent
            transmit($i$, $n$, $id$, $data$)
            if $f < id + C$ //Time to die check
                $a \leftarrow a + 1$
                $n \leftarrow 0$
                $k \leftarrow 0$
                $f \leftarrow 0$
                $i \leftarrow 0$
                goto $\text{``initialize''}$
            else
                $f \leftarrow f + 1$
                if $2 < n$  //alone or not?
                    goto $\text{``mainLoop''}$
        else
            while $x \leq \delta$ do
                if received($i'$, $n'$, $id'$, $data'$) then
                    protocolMaintance($i'$, $n'$, $|data'|$)
                    userRecieve($id'$, $data'$)
                    if $2 < n$  //alone or not?
                        goto $\text{``mainLoop''}$
                end
            end
        end
        wait until $x \geq \delta$
        $x \leftarrow 0$ 
    end
    //Enter main loop
    label: $\text{``mainLoop''}$
    wait until $x \geq \delta$
    $x \leftarrow 0$
    mainLoop()
    //Go back to searching for a network
    label: $\text{``initialize''}$
    $firstNetwork \leftarrow \bot$
    initialize()
\end{lstlisting}
\end{figure}

\bigskip \noindent
With this new procedure for initializing a network, a few new variables are also introduced.
Furthermore the variable $t_0$ which determines the minimum time required to determine a time slot no longer holds true, as such a new variable, $t_1$ is introduced which replaces $t_0$ in \myref{lst:setupCCRC} line 5.
Note that this no longer denotes the minimum time required to find a network, as that is now non-deterministic due to jamming networks.
\begin{table}[h]
    {\setlength{\extrarowheight}{1ex}%
    \begin{tabularx}{\textwidth}{l|l|X|l}
        \toprule
        Name                & Type      & Description & Constraint \\
        \midrule
        $C$                 & integer   & A constant ensuring that low address devices do not die too often                 & $5 \leq C \leq 15$      \\
        $f$                 & integer   & A counter for how many frames a device has been alone                             & $0 \leq f$  \\
        $a$                 & integer   & A 0 indexed exponent for how many attempts have been made at creating a network   & $0 \leq a \leq 5$     \\
        $t_1$               & integer   & The time spent searching for a network                                            & $t_1 = 3 \times \delta + id^a \times \delta$     \\
        $firstN$            & boolean   & A boolean value used to determine whether or not it is the first attempt at a network         & $a \neq 0 \land firstN = \top \lor$     \\
                            &           &                                                                                   & $a = 0 \land firstN = \bot$     \\

        \bottomrule
    \end{tabularx}}
    \caption{Additional local variables used to avoid multiple networks.}
    \label{tab:locals_wmulticonnect}
\end{table}
%\clearpage
\section{Simultaneous Connect} % (fold)
\label{sec:simultaneous_connect}

Another problem arises when two or more devices are started simultaneously while a functioning network is up and running.
If all new devices try to connect in the first empty slot, as they should, they will jam each others messages and neither will be heard.
Moreover the jammed, and jamming, devices will never know that their message was not received by anyone, as there is no way of knowing if one transmission was disrupted or jammed.
A device should always try to connect to a network immediately after starting up, to ensure that in non-jamming situations it is fast.
This also means that jams are bound to happen sooner or later, so some measures must be taken to recover from such situations.
When it comes to multiple devices trying to simultaneously connect to a network, each devices must be able to handle announce conflicts internally.
However the payload should contain some information, which could be used in the solution of these conflicts.

A combination of the following concepts could provide avoidance and recovery from jamming, in the process of joining a network: 
\begin{description}[labelindent=\parindent]
    \item[Validation]\hfill\\
    One should always validate whether or not the announcement has been heard and accepted by the network (one device under CCRC), before joining the network.
    \item[Payload Mode]\hfill\\
    The header of the payload should contain an extra field, which can indicate different modes or some other variable.
    If the design using modes is chosen, it would be possible for other devices implementing the protocol to do various actions upon the data of the payload, e.g. the mode can indicate that the first byte of the data contains the address of a device, which is allowed to join the network.
    Further explanation will be presented in \myref[name]{ssub:modes_and_validation}
    \item[Exponential Backoff]\hfill\\ 
    If no one in the network responds to a device's announcement, or it is not validated, the device will use exponential backoff in regards to how many frames it should wait before trying again.
\end{description} 
\noindent
Using the concepts above it would be possible to design and implement a strategy for handling a scenario, where two or more devices tries to join the same network simultaneously.
The modifications to the existing design from \myref[name]{cha:Design} needed, will be shown in the following paragraphs.

\subsection{Modification of The Payload} % (fold)
\label{sub:modification_of_the_payload}
To pass around information, which is valuable to the protocol, the payload needs to be modified.
This modification should add information where needed and remove unused information in other cases.
In the following paragraph the specific modification, which is needed to accommodate for simultaneously connecting devices, will be presented. 

\subsubsection{Modes and Validation} % (fold)
\label{ssub:modes_and_validation}
The first modification to the payload, is to introduce a way of indicating modes.
With this addition it is possible to utilise the data part of the payload for auxiliary information, e.g. if the mode indicates a validation payload, the first byte of the data could be the address of the device accepted into the network.
Such a modification with mode in the payload header can be seen on \myref{fig:payloadwMode}, where \emph{mode} is inserted after \emph{Address} and before \emph{Data}. 

\tikzfigure{Payload_wModes.tikz}{Modified payload containing an indicator for mode}{payloadwMode}

\noindent
Along side the possibility of sending validation payloads using modes, other functions could benefit from this in the future, e.g. targeted payloads, payload that should be echoed or other special cases.
% subsubsection modes_and_validation (end)

% subsection modification_of_the_payload (end)

\subsection{Introduction of Exponential Backoff} % (fold)
\label{sub:introduction_of_sub_slots_and_exponential_backoff}
To ensure a faster conflict solving if multiple devices try to connect simultaneously, the method of using exponential backoff is chosen.
The strategy of using exponential backoff only applies if jamming occurs.
On \myref{fig:pseudo_flowMultiConnect} it can be seen how method is merged into the existing flow chart (\myref{fig:psuedo_flowCCRC}) from \myref[name]{sub:setupCCRC}.
The flow chart in \myref{fig:pseudo_flowMultiConnect} shows that if a device announces it self and does not get validated by the network, because the announcement was jammed, it will use exponential backoff to prevent further collisions.  

\begin{figure}[ht]
    \centering \footnotesize
    \tikzstyle{decision} = [diamond, draw, fill=yellow!20, 
    text width=4.5em, text badly centered, inner sep=3pt]
\tikzstyle{block} = [rectangle, draw, fill=green!40, 
    text width=6em, text centered, rounded corners, minimum height=4em]
\tikzstyle{line} = [draw, -latex']
\tikzstyle{cloud} = [draw, ellipse,fill=red!20,
    minimum height=2em, align=center]
\begin{tikzpicture}[node distance = 2.8cm, auto]
    % Place nodes
    \node [cloud] (start) {Start};
    \node [block, right of=start] (search) {Search for network};
    \node [decision, below of=search] (found) {Found network?};
    \node [cloud, right of=found, node distance=3.5cm] (init) {GOTO \\ init network};
    \node [block, below of=found] (listen) {Listen while waiting for empty slot};
    \node [block, right of=listen] (choosesubs) {Choose which sub slot to take};
    \node [block, right of=choosesubs] (listensubs) {Listen in the prior sub empty slots};
    \node [decision, below of=listensubs] (heardinsub) {Heard announce?};
    \node [block, below of=listen] (waitempty) {Wait rest of the empty slot};  
    \node [decision, below of=listen, node distance=5.6cm] (validated) {Validated?};
    \node [block, right of=validated] (listenvalidate) {Listen for validation from the network};
    \node [block, right of=listenvalidate] (announce) {Announce in sub empty slot};
    \node [block, left of=validated] (expbackoffcalc) {Determine an exponential backoff}; 
    \node [block, left of=listen] (backoff) {Wait for back-off amount of frames}; 
    \node [block, below of=validated] (join) {Take k after what the empty slot had}; 
    \node [cloud, left of=join] (mainloop) {GOTO \\ main loop};

    % Draw edges
    \path [line] (start) -- (search);
    \path [line] (search) -- (found);
    \path [line] (found) -- node {no} (init);
    \path [line] (found) -- node {yes} (listen);
    \path [line] (listen) -- (choosesubs);
    \path [line] (choosesubs) -- (listensubs);
    \path [line] (listensubs) -- (heardinsub);
    \path [line] (announce) -- (listenvalidate);
    \path [line] (listenvalidate) -- (validated);
    \path [line] (validated) -- node {no} (expbackoffcalc);
    \path [line] (validated) -- node {yes} (join);
    \path [line] (heardinsub) -- node {yes} (waitempty);
    \path [line] (heardinsub) -- node {no} (announce); 
    \path [line] (expbackoffcalc) -- (backoff);
    \path [line] (backoff) -- (listen);
    \path [line] (join) -- (mainloop); 
    \path [line] (waitempty) -- (listen); 
\end{tikzpicture}                                
    \caption{Revised flow diagram showing how a device acts during the Initialization phase}
    \label{fig:pseudo_flowMultiConnect}
\end{figure}

% subsection introduction_of_sub_slots_and_exponential_backoff (end)

\subsection{Changes to the Pseudocode} % (fold)
\label{sub:changes_to_the_pseudocode}
Because the problem of multiple devices connecting simultaneously only exists during the initialisation phase, the change to the pseudocode from \myref[name]{sec:Pseudo} will only affect \myref[name]{sub:setupCCRC}.
Therefore the following modification will focus on line 15-32 (both inclusive) from \myref{lst:setupCCRC}, which deals with what happens if a network is found.
In the modified pseudocode some new variables and constants are introduced, these are presented and explained in \myref{tab:locals_wmulticonnect}.

\begin{table}[ht]
    {\setlength{\extrarowheight}{1ex}%
    \begin{tabularx}{\textwidth}{l|l|X|l}
        \toprule
        Name                & Type      & Description & Constraint \\
        \midrule
        $valid$             & boolean   & Indication of validation              \\
        $c$                 & integer   & Number of failed announcements        & $0 \leq c \leq c_{max}$ \\
        $c_{max}$           & constant   & Maximum value of $c$                        \\
        $b$                 & integer   & Number of frames to wait              & $b \in [0, 2^c-1]$ \\
        \bottomrule
    \end{tabularx}}
    \caption{Additional local variables every device has access to.}
    \label{tab:locals_wmulticonnect}
\end{table}

The modified pseudocode is significantly more complex than the previous verison of the connecting phase.
This is the case because of the two-step strategy previously discussed, which requires addition of several looping and conditional constructs.
In \myref{lst:pseudoInitMulti} the 17 lines from \myref{lst:setupCCRC}, which concerned connecting, have been expanded to 54 lines of code. 

\begin{algorithm}[ht]
\caption{Modified section of initialise procedure for when network found.}
\label{lst:pseudoInitMulti}
\begin{algorithmic}[1]
\Require $c$ is initialised \Comment{Will be 0 on first run}
\Procedure {ConnectToNetwork}{ }
    \While {$i \neq n$}
        \State \textbf{run} \Call{UserCode}{ } $\text{ \textbf{until} }x \ge \delta_{proc}$ 
        \State $i \gets (i \bmod n) + 1$
        \While {$x \le \delta$}
            \If {\Call{Received}{$i', n', id', mode', data'$}}
                \State \Call{ProtocolMaintenance}{$i', n', |data'|$}
                \State \textbf{break}
            \EndIf
        \EndWhile
        \State \textbf{wait until } $x \le \delta$
        \State $x \gets 0$
    \EndWhile
    \State \Call{Transmit}{$i, n, id, data$} \Comment{Send announcement}
    \State $x \gets 0,\, i \gets 0,\, valid \gets \bot$
    \While {$i \neq n$}
        \State \textbf{run} \Call{UserCode}{ } $\text{ \textbf{until} }x \ge \delta_{proc}$
        \State $i \gets (i \bmod n) + 1$
        \While {$x \le \delta$}
            \If {\Call{Received}{$i', n', id', mode', data'$}}
                \State \Call{ProtocolMaintenance}{$i', n', |data'|$}
                \If {\Call{Validated}{$mode', data'$}}
                    \State $valid \gets \top$
                \EndIf
            \EndIf
        \EndWhile
        \State \textbf{wait until } $x \le \delta$
        \State $x \gets 0$ 
    \EndWhile
    \If {$valid$}
        \State $k \gets n$ \Comment{Obtain time-slot}
        \State $n \gets n + 1$ \Comment{Increment slot-count}
        \State \Call{MainLoop}{ }
    \Else
        \If {$c < c_{max}$}
            \State $c \gets c+1$
        \EndIf
        \State $b \gets \text{random } \in [0, 2^c -1]$
        \While {$b > 0$}
            \State \textbf{wait} $1 \text{ frame}$
            \State $b \gets b - 1$
        \EndWhile
        \State \Call{RetryStartUp}{ } \Comment{Go back to searching for networks again}
    \EndIf
\EndProcedure
\end{algorithmic}    
\end{algorithm}

% subsection changes_to_the_pseudocode (end)

% section simultaneous_connect (end) 

\section{Statistical Model Checking}

This section will present the UPPAAL SMC extension, the changes to the model made in UPPAAL to utilise this, and also present the results of the queries which are made using the UPPAAL SMC extension.

When a model rely on stochastic behaviors, such as randomness, its state-space grows exponentially (sometimes called a state-space explosion) therefore an exhaustive search determining wheather if some property holds is not posible in reasonable time. 
Instead one can estimate a percentage chance a property holds, within a given probability of uncertainty, with an statistical model checker (SMC). 

%UPPAAL SMC Explanation
\subsection{UPPAAL SMC}
\begin{wrapfigure}[16]{r}{0.3\textwidth}
\centering
  \includegraphics[width=0.3\textwidth]{Figures/Model/Simple_SMC.pdf} 
\caption{A simple UPPAAL SMC model with weighted edges. }
\label{fig:simpleSMC}
\end{wrapfigure}

To do this in UPPAAL there is an extension called UPPAAL SMC, a full tutorial is avalible here ``Uppaal SMC Tutorial''\cite{DBLP:journals/sttt/DavidLLMP15} by Alexandre David, Kim G. Larsen, Axel Legay, Marius Miku\v{c}ionis, Danny B\o gsted Poulsen.

In UPPAAL SMC there is a new type of edge, this is a weighted edge and is shown as a dashed line. 
An example of this can be seen in figure \myref{fig:simpleSMC}. 
In the figure is a simple UPPAAL SMC model, following the initial state there is a ``branch point'' from which there are two weighted edges.
The number by the weighted edges give their probability of being taken, this simple concept allows UPPAAL SMC to do randomness without having to explore the full state-space, as it can run simulations taking each edge a number of time corrosponding to their weights. 
This example is simulates a cointoss, giving 50 \% chance of heads and 50 \% chance of tails. 

UPPAAL SMC can additionally draw several types of plots which shows what the chance of a given query being true after a given run duration.

It is posible to set the statistical parameteres for UPPAAL SMC, such as: Probability uncertainty, Probability of false negatives and Probability of false positives. 
Setting these parameters closer to zero will cause the time it takes verify a query for a model, however it will also be more accurate. 

%Description of changes to the model in UPPAAL

\subsection*{Changes to the model}

The UPPAAL model shown in \myref{sec:themodel} has gone through extensive changes to accomodate the challenges posed of multiple devices starting at the same time.
The complete model can be seen in the Project folder printed on a piece of A3-paper, and it can also be found on the CD, which is also in the back of the project folder.
Since multiple devices are turned on at the same time, it is now possible that more than one device is transmitting at the same time. 
When this happens, it has been mentioned in \myref{subsubsec:RadioHead}, that nothing is detected by the receivers, therefore a mechanism making sure that when two devices are trying to transmit at teh same time, nothing will be receieved by anyone.

The devices can now start at the same time again, as was possible before the mechanism controlling when a device was released was implemented, as presented in \myref{sec:verifyingTheModel}.
To stop devices from being alone in a network forever, a split node is made when nothing is received in a timeslot, where the device is alone in the network, i.e. the \texttt{local\_n} of the device is equal to 2.
This split node has 2 outgoing edges, one edge which has a probability weight of 90, which makes the device go back into its main loop, and another edge with a probability of 10, which make the device kill the network, and start listening again.
When this happens the exponential back-off method is used, which means for each time a device kills the network it will have a chance of listening for a longer period of time, according to the specification in the previous sections.
This are the changes needed for the model, in order for the devices to start trying to connect to the same network instead of multiple networks.
The other problem is for when devices try to connect to the same network at the same time.
A verification loop has been created for the model, where all devices which transmitted that they wanted to join the network will listen for the number of devices acknowledging their request of joining the network.
If the amount of acknowledgements are lower than half of the devices in the network, they will also use the exponential backoff method, where they will wait for a random number of frames, in an incresingly larger range, with a cap of 31 frames.
This is the only change needed for handling multiple devices connecting to the network, the specifics of the implementation of this on the model will not be presented in this section, but for the curious reader the specifics can be seen in the back as previously mentioned.\todo[inline]{Hvad gør vi med koden der også hører til modellen?} \todo[inline]{Skal der måske alligevel være nogle screenshots af de forskellige dele her ? Det bliver lidt tørt at læse allerede nu synes jeg? Eller er det kun mig der har det sådan ? :) - Søren}

%New Queries using SMC

\subsection*{Verifying using SMC}

Some of the queries presented in \myref{sec:verifyingTheModel} make no sense for this model, as now multiple networks should be running at the same time, and therefore checking if the value of \texttt{i} is the same for all devices in a certain state makes no sense.
It is also of interest to get the probability of the queries being true, rather than checking for true or false.
Therefore the queries will all except for one be checking for the probability of them being true over time, when at least 3000 UPPAAL time units has passed.
It should be noted that the model uses time estiamtes which roughly translate to the time being used in each phase for the hardware used in this project, where a time-slot is equal to \texttt{Delta} which is set to 250 for this model.

All the queries have been run with five devices in the system, and a confidence of 0.999.
The first query is the only query which was also used on the earlier model, and it will still give a result indicating the query holds.

\begin{lstlisting}[style=UPPAAL, title={This query requires that eventually if all devices are connected, then no pair of devices have the same \texttt{k}, unless the pair consists of the same two devices.}]
1. A<> forall(i : id_t) forall(j : id_t) Device(i).Connected and
         Device(j).Connected and Device(i).k == Device(j).k imply i == j
\end{lstlisting}


\begin{lstlisting}[style=UPPAAL, title={This query asks after 3000 UPPAAL time units has passed, what then is the probability that if two devices \texttt{i}, and \texttt{j} are connected to a network that  their local values of \texttt{n} are the same, and that they are both larger than 2. The query will keep checking the probability until it has been found to be 100\% certain or until 300000 UPPAAL time units has passed.}]
2. Pr[<=300000] ( <> forall(i : id_t) forall(j : id_t)  (time > 3000) 
	  	and (Device(i).Connected and Device(j).Connected imply 
	  	  (Device(i).local_n == Device(j).local_n 
	  		and Device(i).local_n > 2 and Device(j).local_n >2)))
\end{lstlisting}
\noindent
UPPAAL responds with [0.998,1], it is possible to get the data UPPAAL uses to make a graph of the development of the probability over time, which can be seen on \myref{fig:ConnectQueryTime}.


\begin{lstlisting}[style=UPPAAL, title={This query asks after 3000 UPPAAL time units has passed, what then is the probability that a device \texttt{i} has a value \texttt{k} which is different from a device \texttt{j}s value of \texttt{n}}]
3. Pr[<=300000] ( <> forall(i : id_t) forall(j :id_t)  (time >3000) 
		and (Device(i).k != Device(j).local_n))
\end{lstlisting}
UPPAAL will again respond with [0.998,1], but this query just like the next query will actually always be true, and as such cannot be found on the graph, as they are just straight lines.

\begin{lstlisting}[style=UPPAAL, title={This query asks after 3000 UPPAAL time units has passed, what then is the probability that if a device \texttt{i} and a device \texttt{j} is both connected that then \texttt{i}'s values of \texttt{k} will be larger than zero and smaller than \texttt{n}}]
4. Pr[<=300000] ( <> forall(i : id_t) forall(j : id_t)  (time >3000) 
	  and (Device(i).Connected and Device(j).Connected 
	  	imply Device(i).k < Device(i).local_n and Device(i).k > 0))
\end{lstlisting}


\begin{lstlisting}[style=UPPAAL, title={This query asks after 3000 UPPAAL time units has passed, what then is the probability that a device \texttt{i} has a local value of \texttt{n} to be equal to the number of devices, which is \texttt{N} + 1, which means that all devices are in the same network.}]
5. Pr[<=300000] Pr[<=300000] ( <>  forall(i: id_t) (time >3000) 
	and  Device(i).local_n == N+1)
\end{lstlisting}

\noindent This query will also result in [0.998,1] chance, and the development of the probability over time can be seen on \myref{fig:ConnectQueryTime}.

%GRAPHS

\begin{figure}
  \includegraphics[width=1\textwidth]{Figures/Graphs/ConnectTime.pdf} 
\caption{Graph showing how the probability of the queries 1, and 4 increase over time.}
\label{fig:ConnectQueryTime}
\end{figure}

As can be seen the change in probability increases logarithimically, and will accorindg to UPPAAL eventually connect, however it might take some time for it to do so.
This is because in the worst case the random numbers generated will keep casuing collisions, or maybe they will take a long time before eventually killing their networks.
There are many possible explanations for what causes the longer runs, but it it important to note the fact that, the longer the devices are running the bigger the chances are of establishing a stable network.

\todo[inline]{Ide: Måske kunne vi vise hvorfor nogle runs tager lang tid: ``Anatomy of a slow run''. Så kan vi vise hvad der skal ske for at det går dårligt. }

%Conlusion

\section{Conclusion}

This chapter presented a way to handle the startup of multiple devices, which made use of the technique exponential back-off, which results in an increasingly randomly generated number of wait time for the devices. 
The randomness results in all devices eventually becoming connected to the network since they will be split up and thus stop jamming eachother.
The UPPAAL model created data, which can be seen on \myref{fig:ConnectQueryTime}.
The data shows that the Arduinos will eventually be connected to the same network, however it does take some time, approximately 150k UPPAAL time units.
This means that an implementation of this design should theoretically be able to solve the issues of turning on more Arduinos at the same time.


%!TEX root = ../../../../master.tex
\newpage
\section{Testing on Arduino}\label{sec:ccrc_test}
The solution for the Simultaneous Connect Problem described in \myref[name]{sec:simultaneous_connect} have been implemented on the Arduino platform along with the multiple start-up.
This is to test if the protocol works in a real life scenario. 
Sourcecode used in this test can be found on the CD attached in the back of the paper.
The test consisted of four devices and the following method was used:
\begin{eletterate}
	\item Start a single device.
	\item After a small time period start the other three devices
	\item Log timestamps for when each device joins the existing network.
\end{eletterate}

\subsection{Results of Test}
The results of the test is shown in \myref{tab:ccrc_test}.
It shows that out of the five test runs, the lowest time it took to make a network of all devices took $13 358$ milliseconds and the highest took $31 959$ milliseconds.
The average was $18780$ and the median $17131$.

The \myref[name]{sec:simultaneous_connect} which was confirmed to be working in UPPAAL SMC also works on Arduinos, where they will eventually connect to the same network, each with different time-slots assigned to each Arduino.

The test also suffers from the fact that the Arduinos do have a chance to miss transmissions, which might result in the devices not being verified to join, and going into backoff without being jammed by another device.
This is not really a problem in terms of correction, but it does impact the speed of the network.

\begin{table}[H]
\footnotesize
\centering  
\rowcolors{2}{GoogleBlueGrey!50}{White}
\begin{tabular}{r | r r r r}
        & Device 0  & Device 1  & Device 2  & Device 3  \\\midrule
 Test 1 &   1251 ms &   6374 ms &   10019 ms&   14081 ms\\
 Test 2 &   1251 ms &   6373 ms &   11948 ms&   13358 ms\\
 Test 3 &   1251 ms &   10278 ms&   10831 ms&   31959 ms\\
 Test 4 &   1251 ms &   13658 ms&   16080 ms&   17372 ms\\
 Test 5 &   1251 ms &   8572 ms &   12502 ms&   17131 ms\\   
\end{tabular}
\caption{Table showing how long time it took (relatively to device 0) for other devices to join the network it made. Note that device 1-3 are not the same in each of the tests, but are sorted in order of joining the network.}
\label{tab:ccrc_test}
\end{table}

\subsection{Testing Simultaneous Start-up of Arduinos}
The simultaneous start-up of devices has also been implemented, where a device has a chance of killing the network it has created if it is alone after the empty-slot.
This has simply been tested by having two identical Arduinos turn on at the same time by connecting their reset pins with a wire, and measuring the time it takes for them to connect to the same network.
The time it took for the two devices to connect to the same network can be seen on \myref{graphConnectime}.
From the data you can determine that the average time to connect to a network is 4458.8 ms with a minimum of 2733 ms and a maximum of 7862 ms.
This time could be improved by implementing a method of avoidance, i.e. starting $c$ greater than zero, thus the time before each device creates the first network is random as well, in addition to the recovery used by killing the network.
If the aforementioned test were to be run on the old implementation, which does not support simultaneous start-up, the two Arduinos would never succeed in creating a single network with both devices connected.
Instead the Arduinos would both create a network, which in turn would collide with each other and as a result of that, jamming would constantly occur.

\begin{table}[H]
\centering\footnotesize
\begin{tabular}{c | c | c | c | c }
Test 1 & Test 2 & Test 3 & Test 4 & Test 5 \\\midrule
6301 ms & 4458 ms & 3299 ms & 7862 ms & 3865 ms \\
\end{tabular}\\
\vspace{15pt}
\begin{tabular}{c | c | c | c | c }
Test 6 & Test 7 & Test 8 & Test 9 & Test 10\\\midrule
3299 ms & 3865 ms & 2733 ms & 4448 ms & 4458 ms \\
\end{tabular}
\caption{Data from test of simultaneous startup on Arduinos. Shows the time it takes two simultaneously started devices to connect to the same network.}
\label{graphConnectime}
\end{table} 
\section{Conclusion}
This chapter presented a way to handle the startup of multiple devices, which made use of the method exponential backoff, which results in an increasingly randomly generated number of wait time for the devices. 
The randomness results in all devices eventually becoming connected to the network since they will be split up and thus stop jamming each other.
The UPPAAL model created data, which can be seen on \myref{fig:ConnectQueryTime}.
The data shows that the Arduinos will eventually be connected to the same network, however it does take some time, approximately $150 000$ UPPAAL time units.
This means that an implementation of this design should be able to solve the issues of turning on more Arduinos at the same time, assuming our assumptions are correct.

