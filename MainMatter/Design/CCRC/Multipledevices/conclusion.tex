\section{Conclusion}
This chapter presented a way to handle the startup of multiple devices, which made use of the method exponential backoff, which results in an increasingly randomly generated number of wait time for the devices. 
The randomness results in all devices eventually becoming connected to the network since they will be split up and thus stop jamming each other.
The UPPAAL model created data, which can be seen on \myref{fig:ConnectQueryTime}.
The data shows that the Arduinos will eventually be connected to the same network, however it does take some time, approximately $150 000$ UPPAAL time units.
According to UPPAAL an implementation of the protocol should therefore be possible, which is what was created and tested in \myref{sec:ccrc_test}.
This implementation did not handle the activation of multiple devices, without having one start a network first. 
This is due to time-constraints, but it is shown with UPPAAL that the design should work.
