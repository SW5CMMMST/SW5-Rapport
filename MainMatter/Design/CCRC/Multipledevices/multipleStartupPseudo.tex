
\bigskip \noindent
The following procedure replaces lines 34 through 38 in \myref{lst:setupCCRC}
\begin{lstlisting}[label=lst:networkMultiStartCCRC,style=pseudocode,mathescape=true,caption={Pseudocode example of the procedure initializeNetwork() for CCRC with multiconnect}]
procedure initializeNetwork()
    // Initialize variables
    $x \leftarrow 0$
    $n \leftarrow 2$
    $k \leftarrow 1$
    $i \leftarrow 0$
    $f \leftarrow 0$
    $C \leftarrow 10$
    if $firstNetwork \neq \bot$
        $a \leftarrow 0$
    
    loop forever
        if $2 < n$  //alone or not?
            goto $\text{``mainLoop''}$
        run userCode() until $x \geq \delta_{proc}$
        $i \leftarrow (i \text{ mod } n) + 1$ // Update current slot
        if $i = k$ then
            makePayload() //Updates the data to be sent
            transmit($i$, $n$, $id$, $data$)
            if $f < id + C$ //Time to die check
                $a \leftarrow a + 1$
                $n \leftarrow 0$
                $k \leftarrow 0$
                $f \leftarrow 0$
                $i \leftarrow 0$
                goto $\text{``initialize''}$
            else
                $f \leftarrow f + 1$
        else
            while $x \leq \delta$ do
                if received($i'$, $n'$, $id'$, $data'$) then
                    protocolMaintance($i'$, $n'$, $|data'|$)
                    userRecieve($id'$, $data'$)
                    if $2 < n$  //alone or not?
                        goto $\text{``mainLoop''}$
                end
            end
        end
        wait until $x \geq \delta$
        $x \leftarrow 0$ 
    end
    //Enter main loop
    label: $\text{``mainLoop''}$
    wait until $x \geq \delta$
    $x \leftarrow 0$
    mainLoop()
    //Go back to searching for a network
    label: $\text{``initialize''}$
    $firstNetwork \leftarrow \bot$
    initialize()
\end{lstlisting}

With this new procedure for initializing a network, a few new variables are also introduced.
Furthermore the variable $t_0$ which determines the minimum time required to determine a time slot no longer holds true, as such a new variable, $t_1$ is introduced which replaces $t_0$ in \myref{lst:setupCCRC} line 5.
Note that this no longer denotes the minimum time required to find a network, as that is now non-deterministic due to jamming networks.
\begin{table}[h]
    {\setlength{\extrarowheight}{1ex}%
    \begin{tabularx}{\textwidth}{l|l|X|l}
        \toprule
        Name                & Type      & Description & Constraint \\
        \midrule
        $C$                 & integer   & A constant ensuring that low address devices do not die too often                 & $5 \leq C \leq 15$      \\
        $f$                 & integer   & A counter for how many frames a device has been alone                             & $0 \leq f$  \\
        $a$                 & integer   & A 0 indexed exponent for how many attempts have been made at creating a network   & $0 \leq a \leq 5$     \\
        $t_1$               & integer   & The time spent searching for a network                                            & $t_1 = 3 \times \delta + id^a \times \delta$     \\
        $firstNetwork$      & boolean   & A boolean value used to determine whether or not to initialize a to zero          & $a \neq 0 \land firstNetwork = \top \lor a = 0 \land firstNetwork = \bot$     \\

        \bottomrule
    \end{tabularx}}
    \caption{Additional local variables used to avoid multiple networks.}
    \label{tab:locals_wmulticonnect}
\end{table}