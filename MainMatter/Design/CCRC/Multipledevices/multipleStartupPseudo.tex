The following replaces lines 34 through 38 in \myref{lst:setupCCRC}

\begin{lstlisting}[label=lst:networkMultiConnectCCRC,style=pseudocode,mathescape=true,caption={Pseudocode example of the procedure initializeNetwork() for CCRC with multiconnect}]
procedure initializeNetwork()
    // Initialize variables
    $x \leftarrow 0$
    $n \leftarrow 2$
    $k \leftarrow 1$
    $i \leftarrow 0$
    $a \leftarrow 0$
    $f \leftarrow 0$
    $C \leftarrow 10$
    procedure mainLoop()
    loop forever
        run userCode() until $x \geq \delta_{proc}$
        $i \leftarrow (i \text{ mod } n) + 1$ // Update current slot
        if $i = k$ then
            makePayload() // Updates the data to be sendt
            transmit($i$, $n$, $id$, $data$)
        else
            while $x \leq \delta$ do
                if received($i'$, $n'$, $id'$, $data'$) then
                    protocolMaintance($i'$, $n'$, $|data'|$)
                    userRecieve($id'$, $data'$)
                end
            end
        end
        wait until $x \geq \delta$
        $x \leftarrow 0$ 
    end
    // Enter main loop
    mainLoop()
\end{lstlisting}

With this new procedure for initializing a network, a few new variables are also introduced.
\begin{table}[h]
    {\setlength{\extrarowheight}{1ex}%
    \begin{tabularx}{\textwidth}{l|l|X|l}
        \toprule
        Name                & Type      & Description & Constraint \\
        \midrule
        $C$                 & integer   & A constant ensuring that low address devices do not die too often  & $5 \leq C \leq 15$      \\
        $f$                 & integer   & A counter for how many frames a device has been alone              & $0 \leq f$  \\
        $a$                 & integer   & A 0 indexed exponent for how many attempts have been made at creating a network & $0 \leq a \leq 5$     \\
        \bottomrule
    \end{tabularx}}
    \caption{Additional local variables used to avoid multiple networks.}
    \label{tab:locals_wmulticonnect}
\end{table}