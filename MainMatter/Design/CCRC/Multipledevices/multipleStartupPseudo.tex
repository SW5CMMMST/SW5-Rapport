\subsection{Changes to the Pseudocode}
This section will cover the new pseudocode which must be written for those areas multiple startup affects; in \myref{lst:setupCCRC} only six lines handle the creation of a new network, which is line 5 and lines 34-38.
As is shown on the flow diagram \myref{fig:pseudo_flowMultiStart} the multiple network establishment issue is handled with an altered main loop, as such much of the new pseudocode will resemble pseudocode from \myref{lst:general_case}.
The procedure seen in \myref{lst:networkMultiStartCCRC2} replaces lines 34 through 38 in \myref{lst:setupCCRC}. 
\begin{algorithm}
\caption{Initialise network in CCRC for CCRC for the multiple device activation problem.}
\label{lst:networkMultiStartCCRC2}
\begin{algorithmic}[1]
\Require $c$ is initialised \Comment{Will be 0 on first run}
\Require $kill_{chance} = 5\%$ 
\Procedure{InitializeNetwork}{ }
    \State $x \gets 0, n \gets 2, k \gets 1, i \gets 0$
    \Repeat \Comment{Loop forever}
        \State \textbf{run} \Call{UserCode}{ } $\text{ \textbf{until} }x \ge \delta_{proc}$
        \State $i \gets (i \bmod n) + 1$ \Comment{Update current slot}
        \If {$i = k$}
            \State \Call{MakePayload}{ } \Comment{Update the data to be sent}
            \State \Call{Transmit}{$i, n, id, data$}
        \Else
            \While {$x \le \delta$}
                \If {\Call{Received}{$i', n', id', data'$}}
                    \State \Call{ProtocolMaintenance}{$i', n', |data'|$}
                    \State \Call{UserReceive}{$id', data'$}
                    \If {$n > 2$}
                        \State \textbf{wait until } $x \ge \delta$  
                        \State $x \gets 0$
                        \State \Call{MainLoop}{ }
                    \EndIf
                \Else
                    \If {\Call{TimeToDie?}{$kill_{chance}$}} \Comment{$kill_{chance}\%$ of killing}
                        \State $c \gets c + 1$
                        \State \Call{RetryStartUp}{ } \Comment{Return to searching for networks}
                    \EndIf
                \EndIf
            \EndWhile
        \EndIf
        \State \textbf{wait until } $x \ge \delta$  
        \State $x \gets 0$
    \Until the end of time
\EndProcedure    
\end{algorithmic}
\end{algorithm}

\bigskip \noindent
With this new procedure for initializing a network, a few new variables are also introduced, all of which are shown in \myref{tab:locals_wmultistartup}.
Furthermore the variable $t_0$ which determines the minimum time required to determine a time slot no longer holds true, as such a new variable, $t_1$ is introduced which replaces $t_0$ on \myref{lst:setupCCRC} line 5.
Note that this no longer denotes the minimum time required to find a network, as that is now non-deterministic due to jamming networks.
\begin{table}[h]
    {\setlength{\extrarowheight}{1ex}%
    \begin{tabularx}{\textwidth}{l|l|X|l}
        \toprule
        Name                & Type       & Description & Constraint \\
        \midrule
        $c$                 & integer    & Number of attempts of creating a network, but capped at 5                         & $0 \leq c \leq 5$     \\
        $t_1$               & integer    & The time spent searching for a network                                            & $t_1 = 3 \times \delta + id^c \times \delta$     \\
        $kill_{chance}$     & percentage & The chance of a device killing it self                                            & $kill_{chance} = 5\%$  \\
        \bottomrule
    \end{tabularx}}
    \caption{Additional local variables used to avoid multiple networks.}
    \label{tab:locals_wmultistartup}
\end{table}