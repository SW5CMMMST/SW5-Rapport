\section{Simultaneous Connect} % (fold)
\label{sec:simultaneous_connect}

Another problem arises when two or more devices are started simultaneously while a functioning network is up and running.
If all new devices tries to connect in the first empty slot, as they should, they will jam each others messages and no one will be heard.
Moreover the jammed, and jamming, devices will never know that their message was not received by anyone, as there is no way of knowing if one transmission was disrupted or jammed.
A device should always try to connect to a network immediately after starting up.
This means that jams are bound to happen sooner or later, so some measures must be taken to avoid and recover from such situations.
When it comes to multiple devices trying to simultaneously connect to a network, each devices must be able to handle announce conflicts internally.
However the payload should contain some information, which could be used in the solution of these conflicts.

A combination of the following concepts could provide avoidance and recovery from jamming, in the process of joining a network: 
\begin{description}[labelindent=\parindent]
    \item[Random Offset]\hfill\\
    When a device is announcing itself in the empty slot, it randomly chooses an offset for when to send the announcement payload.
    The random offset is not time but sub slots. 
    In the rest of the empty slot it listens for other possible devices announcing themselves.
    If such message is heard it does not attempt to announce itself in the empty slot.
    This method implies that the length of empty slot is a multiple of the time it takes to transmit the announcement payload.
    \item[Validation]\hfill\\
    You should always validate whether or not your announcement have been heard and accepted by a majority of the network (over 50 \%), before joining the network.
    \item[Payload Mode]\hfill\\
    The head of the payload should contain an extra field, which can indicate different modes or some other variable.
    If the design using modes is chosen, it would be possible for other devices implementing the protocol to do various actions upon the data of the payload, e.g. the mode can indicate that the first byte of the date contains the address of a device, which is allowed to join the network.
    \item[Exponential Backoff]\hfill\\ 
    If no one in the network responds to a device's announcement, or it is not validated, the device will use exponential backoff in regards to how many frames it should wait before trying again.
\end{description} 
\noindent
Using the concepts above it would be possible to design and implement a strategy for handling a scenario, where two or more devices tries to join the same network simultaneously.
The modifications to the existing design from (REF TO CCRC-design) needed, will be shown in the following paragraphs.

\subsection{Modification of The Payload} % (fold)
\label{sub:modification_of_the_payload}
To pass around information, which is valuable to the protocol, the payload needs to be modified.
This modification should add information where needed and remove unused information in other cases.
In the following paragraphs two modifications, which are needed to accommodate for simultaneously connecting devices, will be presented. 

\subsubsection{Modes and Validation} % (fold)
\label{ssub:modes_and_validation}
\textbf{--WIP--}
The first modification to the payload, is to introduce a way of indicating modes. 
\tikzfigure{Payload_wModes.tikz}{Modified payload containing an indicator for mode}{payloadwMode}
% subsubsection modes_and_validation (end)

\subsubsection{Announcements} % (fold)
\label{ssub:announcements}
\textbf{--WIP--}
% subsubsection announcements (end)

% subsection modification_of_the_payload (end)

\subsection{Introduction of Sub Slots and Exponential Backoff} % (fold)
\label{sub:introduction_of_sub_slots_and_exponential_backoff}
To ensure a faster conflict solving if multiple devices try to connect simultaneously, a two-step process is designed.
These two steps consists of choosing a random offset of sub slots to wait in the empty slot, along with a strategy of exponential back-off if jamming occurs.
On \myref{fig:pseudo_flowMultiConnect} it can be seen how said two-steps is merged into the existing flow chart (\myref{fig:psuedo_flowCCRC}) from \myref{sub:setupCCRC}

\tikzfigure{PsuedoFlowDiagramMULTI_CONNECT}{Revised flow diagram showing how a device acts during the Initialization phase}{pseudo_flowMultiConnect}

\subsubsection{Sub Slots} % (fold)
\label{ssub:sub_slots}
To avoid using the potentially time consuming exponential backoff, if a multiple of devices try to connect simultaneously, the empty slot will be partitioned into smaller sub slots (see \myref{fig:frame_wsubslots} for illutration). 
The length of the sub slots should be the longer than time it takes to transmit an announcement; is important not to have overlapping announcements, since none of them would be heard due to jamming.

When determining how many sub slots the empty slot should be partitioned into, it should be considered that more sub slots will increase the overall frame length.
Because of this, the number of partitions should be decided by the programmer implementing the protocol.

\tikzfigure{SimpleFrame_wSubSlots.tikz}{A frame with two connected devices and a three-way partitioned empty slot}{frame_wsubslots}

% subsubsection sub_slots (end)

\subsubsection{Exponential Backoff} % (fold)
\label{ssub:exponential_backoff}
The method of exponential backoff to be used here consists of:
\begin{enumerate*}[label=\itshape \alph*\upshape)]
    \item remembering how many unvalidated announcements have been sent, this will typically be caused by collisions; and
    \item choosing a random number of frames to wait based upon this number.   
\end{enumerate*}
These properties of the exponential backoff can be expressed as follows:
\begin{equation}
     x \in [0, 2^c - 1] \mid x \in \mathbb{Z}^+
\end{equation}
Where $x$ is a randomly chosen number of frames to wait from the range, and $c$ is number of unvalidated announcements. With this method a device will wait 0 to 1 frames after the first unvalidated announcement, 0 to 3 after the second and 0 to 7 after the third. To avoid an infinite increase of the wait time the $c$ should have an upper limit of e.g. 5, which would ensure a maximum wait of $2^5-1 = 31$ frames.


% subsubsection exponential_backoff (end) 
% subsection introduction_of_sub_slots_and_exponential_backoff (end)


% section simultaneous_connect (end) 