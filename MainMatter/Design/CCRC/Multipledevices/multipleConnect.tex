%\clearpage
\section{Simultaneous Connect} % (fold)
\label{sec:simultaneous_connect}
Another problem arises when two or more devices are started simultaneously while a functioning network is up and running.
If all new devices try to connect in the first empty slot, as they should, they will jam each others messages and neither will be heard.
Moreover the jammed, and jamming, devices will never know that their message was not received by anyone, as there is no way of knowing if one transmission was disrupted or jammed.
A device should always try to connect to a network immediately after starting up, to ensure that in non-jamming situations it is fast.
This also means that jams are bound to happen sooner or later, so some measures must be taken to recover from such situations.
When it comes to multiple devices trying to simultaneously connect to a network all devices must be able to handle announce conflicts internally.
However the payload should contain some information, which could be used in the solution of these conflicts.

A combination of the following concepts could provide avoidance and recovery from jamming in the process of joining a network: 
\begin{description}[labelindent=\parindent]
    \item[Validation]\hfill\\
    One should always validate whether or not the announcement has been heard and accepted by the network (one device under CCRC) before joining the network.
    No other devices should try to join in an empty-slot if the network is validating another device in that frame, in order to ensure the transmission in the frame following a successful validation is not jammed.
    \item[Payload Mode]\hfill\\
    The header of the payload should contain an extra field, which can indicate different modes or some other variable.
    If the design using modes is chosen it would be possible for other devices implementing the protocol to do various actions upon the data of the payload, e.g. the mode can indicate that the first byte of the data contains the address of a device, which is allowed to join the network.
    Further explanation will be presented in \myref[name]{ssub:modes_and_validation}
    \item[Exponential Backoff]\hfill\\ 
    If no one in the network responds to a device's announcement, or it is not validated, the device will use exponential backoff in regards to how many frames it should wait before trying again.
\end{description} 
\noindent
Using the concepts above it would be possible to design and implement a strategy for handling a scenario, where two or more devices tries to join the same network simultaneously.
The modifications to the existing design from \myref[name]{cha:Design} needed, will be shown in the following paragraphs.

\subsection{Modification of The Payload} % (fold)
\label{sub:modification_of_the_payload}
To pass around information, which is valuable to the protocol, the payload needs to be modified.
This modification should add information where needed and remove unused information in other cases.
In the following paragraph the specific modification, which is needed to accommodate for simultaneously connecting devices, will be presented. 

\subsubsection{Modes and Validation} % (fold)
\label{ssub:modes_and_validation}
The first modification to the payload, is to introduce a way of indicating modes.
With this addition it is possible to utilise the data part of the payload for auxiliary information, e.g. if the mode indicates a validation payload, the first byte of the data could be the address of the device accepted into the network.
Such a modification with mode in the payload header can be seen on \myref{fig:payloadwMode}, where \emph{mode} is inserted after \emph{Address} and before \emph{Data}. 

\tikzfigure{Payload_wModes.tikz}{Modified payload containing an indicator for mode}{payloadwMode}

\noindent
Alongside the possibility of sending validation payloads using modes, other functions could benefit from this in the future, e.g. targeted payloads, payloads that should be echoed or other special cases.
% subsubsection modes_and_validation (end)

% subsection modification_of_the_payload (end)

\subsection{Recovery by Exponential Backoff} % (fold)
\label{sub:introduction_of_sub_slots_and_exponential_backoff}
To ensure faster conflict solving if multiple devices try to connect simultaneously, the method of using exponential backoff is chosen.
The strategy of using exponential backoff only applies if a device is not validated after trying to join a network.
On \myref{fig:pseudo_flowMultiConnect} it can be seen how method is merged into the existing flow chart (\myref{fig:psuedo_flowCCRC}) from \myref[name]{sub:setupCCRC}.
The flow chart in \myref{fig:pseudo_flowMultiConnect} shows that if a device announces itself and does not get validated by the network, because the announcement was jammed, it will use exponential backoff to prevent further collisions.  

\begin{figure}[ht]
    \centering \footnotesize
    \tikzstyle{decision} = [diamond, draw, fill=yellow!20, 
    text width=4.5em, text badly centered, node distance=3cm, inner sep=3pt]
\tikzstyle{block} = [rectangle, draw, fill=green!40, 
    text width=6em, text centered, rounded corners, minimum height=4em]
\tikzstyle{line} = [draw, -latex']
\tikzstyle{cloud} = [draw, ellipse,fill=red!20,
    minimum height=2em, align=center]
\begin{tikzpicture}[node distance = 3.8cm, auto]
    % Place nodes
    \node [cloud] (start) {Start};
    \node [block, right of=start] (search) {Search for network};
    \node [decision, below of=search] (found) {Found network?};
    \node [cloud, left of=found] (init) {GOTO \\ init network};
    \node [block, below of=found] (listen) {Listen while waiting for empty slot};
    \node [block, right of=listen] (announce) {Announce in the empty slot, using a random offset};
    \node [decision, below of=listen] (validated) {Validated?};
    \node [block, right of=validated] (listenvalidate) {Listen for validation from the network};
    \node [block, left of=validated] (expbackoffcalc) {Determine an exponential backoff}; 
    \node [block, left of=listen] (backoff) {Wait for back-off amount of frames}; 
    \node [block, below of=validated] (join) {Take k after what the empty slot had}; 
    \node [cloud, left of=join] (mainloop) {GOTO \\ main loop};

    % Draw edges
    \path [line] (start) -- (search);
    \path [line] (search) -- (found);
    \path [line] (found) -- node {no} (init);
    \path [line] (found) -- node {yes} (listen);
    \path [line] (listen) -- (announce);
    \path [line] (announce) -- (listenvalidate);
    \path [line] (listenvalidate) -- (validated);
    \path [line] (validated) -- node {no} (expbackoffcalc);
    \path [line] (validated) -- node {yes} (join);
    \path [line] (expbackoffcalc) -- (backoff);
    \path [line] (backoff) -- (listen);
    \path [line] (join) -- (mainloop);  
\end{tikzpicture}                                   
    \caption{Revised flow diagram showing how a device acts during the Initialisation phase}
    \label{fig:pseudo_flowMultiConnect}
\end{figure}

% subsection introduction_of_sub_slots_and_exponential_backoff (end)

\subsection{Changes to the Pseudocode} % (fold)
\label{sub:changes_to_the_pseudocode}
Because the problem of multiple devices connecting simultaneously only exists during the initialisation phase, the change to the pseudocode from \myref[name]{sec:Pseudo} will only affect \myref[name]{sub:setupCCRC}.
Therefore the following modification will focus on line 15-32 (both inclusive) from \myref{lst:setupCCRC}, which deals with what happens if a network is found.
In the modified pseudocode some new variables and constants are introduced, these are presented and explained in \myref{tab:locals_wmulticonnect}.

\begin{table}[H]
    {\setlength{\extrarowheight}{1ex}%
    \begin{tabularx}{\textwidth}{l|l|X|l}
        \toprule
        Name                & Type      & Description & Constraint \\
        \midrule
        $valid$             & boolean   & Indication of validation              \\
        $c$                 & integer   & Number of failed announcements        & $0 \leq c \leq c_{max}$ \\
        $c_{max}$           & constant   & Maximum value of $c$                        \\
        $b$                 & integer   & Number of frames to wait              & $b \in [0, 2^c-1]$ \\
        \bottomrule
    \end{tabularx}}
    \caption{Additional local variables every device has access to.}
    \label{tab:locals_wmulticonnect} 
    \vspace{-10pt}
\end{table}

\noindent
The modified pseudocode is significantly more complex than the previous version of the connecting phase.
This is the case because of the two-step strategy previously discussed, which requires addition of several looping and conditional constructs.
In \myref{lst:pseudoInitMulti} the 17 lines from \myref{lst:setupCCRC}, which concerned connecting, have been expanded to 54 lines of code. 

\begin{algorithm}[H]
\caption{Modified section of initialise procedure for when network found.}
\label{lst:pseudoInitMulti}
\begin{algorithmic}[1]
\Require $c$ is initialised \Comment{Will be 0 on first run}
\Procedure {ConnectToNetwork}{ }
    \While {$i \neq n$}
        \State \textbf{run} \Call{UserCode}{ } $\text{ \textbf{until} }x \ge \delta_{proc}$ 
        \State $i \gets (i \bmod n) + 1$
        \While {$x \le \delta$}
            \If {\Call{Received}{$i', n', id', mode', data'$}}
                \State \Call{ProtocolMaintenance}{$i', n', |data'|$}
                \State \textbf{break}
            \EndIf
        \EndWhile
        \State \textbf{wait until } $x \le \delta$
        \State $x \gets 0$
    \EndWhile
    \State \Call{Transmit}{$i, n, id, data$} \Comment{Send announcement}
    \State $x \gets 0,\, i \gets 0,\, valid \gets$ \textit{false}
    \While {$i \neq n$}
        \State \textbf{run} \Call{UserCode}{ } $\text{ \textbf{until} }x \ge \delta_{proc}$
        \State $i \gets (i \bmod n) + 1$
        \While {$x \le \delta$}
            \If {\Call{Received}{$i', n', id', mode', data'$}}
                \State \Call{ProtocolMaintenance}{$i', n', |data'|$}
                \If {\Call{Validated}{$mode', data'$}}
                    \State valid $\gets$ \textit{true}
                \EndIf
            \EndIf
        \EndWhile
        \State \textbf{wait until } $x \le \delta$
        \State $x \gets 0$ 
    \EndWhile
    \If {$valid$}
        \State $k \gets n$ \Comment{Obtain time-slot}
        \State $n \gets n + 1$ \Comment{Increment slot-count}
        \State \Call{MainLoop}{ }
    \Else
        \If {$c < c_{max}$}
            \State $c \gets c+1$
        \EndIf
        \State $b \gets \text{random } \in [0, 2^c -1]$
        \While {$b > 0$}
            \State \textbf{wait} $1 \text{ frame}$
            \State $b \gets b - 1$
        \EndWhile
        \State \Call{RetryStartUp}{ } \Comment{Go back to searching for networks again}
    \EndIf
\EndProcedure
\end{algorithmic}    
\end{algorithm}

% subsection changes_to_the_pseudocode (end)

% section simultaneous_connect (end) 
