\clearpage
\section{Simultaneous Connect} % (fold)
\label{sec:simultaneous_connect}

Another problem arises when two or more devices are started simultaneously while a functioning network is up and running.
If all new devices tries to connect in the first empty slot, as they should, they will jam each others messages and neither will be heard.
Moreover the jammed, and jamming, devices will never know that their message was not received by anyone, as there is no way of knowing if one transmission was disrupted or jammed.
A device should always try to connect to a network immediately after starting up, to ensure that in non-jamming situations it is fast.
This also means that jams are bound to happen sooner or later, so some measures must be taken to avoid and recover from such situations.
When it comes to multiple devices trying to simultaneously connect to a network, each devices must be able to handle announce conflicts internally.
However the payload should contain some information, which could be used in the solution of these conflicts.

A combination of the following concepts could provide avoidance and recovery from jamming, in the process of joining a network: 
\begin{description}[labelindent=\parindent]
    \item[Random Offset]\hfill\\
    When a device is announcing itself in the empty slot, it randomly chooses an offset for when to send the announcement payload.
    The random offset is not time but sub slots. 
    In the rest of the empty slot it listens for other possible devices announcing themselves.
    If such a message is heard it does not attempt to announce itself in a later sub slot of the empty slot.
    This method implies that the length of empty slot is a multiple of the time it takes to transmit the announcement payload.
    \item[Validation]\hfill\\
    You should always validate whether or not your announcement have been heard and accepted by  the network (one device under CCRC), before joining the network.
    \item[Payload Mode]\hfill\\
    The header of the payload should contain an extra field, which can indicate different modes or some other variable.
    If the design using modes is chosen, it would be possible for other devices implementing the protocol to do various actions upon the data of the payload, e.g. the mode can indicate that the first byte of the date contains the address of a device, which is allowed to join the network.
    \item[Exponential Backoff]\hfill\\ 
    If no one in the network responds to a device's announcement, or it is not validated, the device will use exponential backoff in regards to how many frames it should wait before trying again.
\end{description} 
\noindent
Using the concepts above it would be possible to design and implement a strategy for handling a scenario, where two or more devices tries to join the same network simultaneously.
The modifications to the existing design from \myref{cha:Design} needed, will be shown in the following paragraphs.

\subsection{Modification of The Payload} % (fold)
\label{sub:modification_of_the_payload}
To pass around information, which is valuable to the protocol, the payload needs to be modified.
This modification should add information where needed and remove unused information in other cases.
In the following paragraphs two modifications, which are needed to accommodate for simultaneously connecting devices, will be presented. 

\subsubsection{Modes and Validation} % (fold)
\label{ssub:modes_and_validation}
The first modification to the payload, is to introduce a way of indicating modes.
With this addition it is possible to utilise the data part of the payload for auxiliary information, e.g. if the mode indicates a validation payload, the first byte of the data could be the address of the device accepted into the network.
Such a modification with mode in the payload header can be seen on \myref{fig:payloadwMode}, where \emph{mode} is inserted after \emph{Address} and before \emph{Data}. 

\tikzfigure{Payload_wModes.tikz}{Modified payload containing an indicator for mode}{payloadwMode}

\noindent
Along side the possibility of sending validation payloads using modes, other functions could benefit from this in the future, e.g. targeted payloads, payload that should be echoed or other special cases.
% subsubsection modes_and_validation (end)

\subsubsection{Announcements} % (fold)
\label{ssub:announcements}
To minimise the time it takes to announce oneself to a network, a special announcement payload is introduced.
This payload specification only need to carry one information that is the unique identifier or address of the device which is sending it.
By shortening the payload significantly from the regular size of at least five fields (the header in \myref{fig:payloadwMode}) to one, the time it takes to sent it also is drastically decreased.
% subsubsection announcements (end)

% subsection modification_of_the_payload (end)

\subsection{Introduction of Sub Slots and Exponential Backoff} % (fold)
\label{sub:introduction_of_sub_slots_and_exponential_backoff}
To ensure a faster conflict solving if multiple devices try to connect simultaneously, a two-step process is designed.
These two steps consists of choosing a random offset of sub slots to wait in the empty slot, along with a strategy of exponential back-off if jamming occurs.
On \myref{fig:pseudo_flowMultiConnect} it can be seen how said two-steps is merged into the existing flow chart (\myref{fig:psuedo_flowCCRC}) from \myref{sub:setupCCRC}

\begin{figure}[p]
    \centering \footnotesize
    \tikzstyle{decision} = [diamond, draw, fill=yellow!20, 
    text width=4.5em, text badly centered, node distance=3cm, inner sep=3pt]
\tikzstyle{block} = [rectangle, draw, fill=green!40, 
    text width=6em, text centered, rounded corners, minimum height=4em]
\tikzstyle{line} = [draw, -latex']
\tikzstyle{cloud} = [draw, ellipse,fill=red!20,
    minimum height=2em, align=center]
\begin{tikzpicture}[node distance = 3.8cm, auto]
    % Place nodes
    \node [cloud] (start) {Start};
    \node [block, right of=start] (search) {Search for network};
    \node [decision, below of=search] (found) {Found network?};
    \node [cloud, left of=found] (init) {GOTO \\ init network};
    \node [block, below of=found] (listen) {Listen while waiting for empty slot};
    \node [block, right of=listen] (announce) {Announce in the empty slot, using a random offset};
    \node [decision, below of=listen] (validated) {Validated?};
    \node [block, right of=validated] (listenvalidate) {Listen for validation from the network};
    \node [block, left of=validated] (expbackoffcalc) {Determine an exponential backoff}; 
    \node [block, left of=listen] (backoff) {Wait for back-off amount of frames}; 
    \node [block, below of=validated] (join) {Take k after what the empty slot had}; 
    \node [cloud, left of=join] (mainloop) {GOTO \\ main loop};

    % Draw edges
    \path [line] (start) -- (search);
    \path [line] (search) -- (found);
    \path [line] (found) -- node {no} (init);
    \path [line] (found) -- node {yes} (listen);
    \path [line] (listen) -- (announce);
    \path [line] (announce) -- (listenvalidate);
    \path [line] (listenvalidate) -- (validated);
    \path [line] (validated) -- node {no} (expbackoffcalc);
    \path [line] (validated) -- node {yes} (join);
    \path [line] (expbackoffcalc) -- (backoff);
    \path [line] (backoff) -- (listen);
    \path [line] (join) -- (mainloop);  
\end{tikzpicture}                                   
    \caption{Revised flow diagram showing how a device acts during the Initialization phase}
    \label{fig:pseudo_flowMultiConnect}
\end{figure}

The flow chart in \myref{fig:pseudo_flowMultiConnect} shows that the random offset of sub slots is the first method of prevention when it comes to simultaneous connection of multiple devices.
If a device hears another announcement before it has sent it's own in the chosen sub slot, it will abort the announce-phase and wait for the next empty slot before trying immediately again.
However if a device announces it self and does not get validated by the network, because the announcement was jammed, it will use exponential backoff to prevent further collisions.
The following paragraphs will discuss said sub slots and exponential backoff.

\subsubsection{Sub Slots} % (fold)
\label{ssub:sub_slots}
To avoid using the potentially time consuming exponential backoff, if a multiple of devices try to connect simultaneously, the empty slot will be partitioned into smaller sub slots (see \myref{fig:frame_wsubslots} for illutration). 
The length of the sub slots should be the longer than time it takes to transmit an announcement; is important not to have overlapping announcements, since none of them would be heard due to jamming.

When determining how many sub slots the empty slot should be partitioned into, it should be considered that more sub slots will increase the overall frame length.
Because of this, the number of partitions should be decided by the programmer implementing the protocol.

\begin{figure}[h]
    \centering \footnotesize
    \resizebox{1\textwidth}{!}{%
\begin{tikzpicture}[scale=\textwidth, node distance = 0cm]
\node[draw, align = center, 
        minimum width=0.33\textwidth, 
        minimum height=10mm] 
    (slot0)
    {$Slot_1$ (occupied)};
\node[draw, right=of slot0,
        minimum width=0.33\textwidth, 
        minimum height=10mm]
    (slot1)
    {$Slot_2$ (occupied)};
\node[draw, right=of slot1,
        minimum width=0.33\textwidth, 
        minimum height=10mm,
        rectangle split, 
        rectangle split horizontal,
        rectangle split parts=3,
        rectangle split draw splits=false]
    (slot2)
    {$Sub\ Slot_1$ \nodepart{two} $Sub\ Slot_2$ \nodepart{three} $Sub\ Slot_3$};
    
\draw [decoration={brace, mirror, amplitude=+20pt}, decorate]
    (slot1.south west) -- (slot1.south east) node [black,midway,below=+21pt] 
    {Regular time slot};
\draw [decoration={brace, mirror, amplitude=+20pt}, decorate]
    (slot2.south west) -- (slot2.south east) node [black,midway,below=+21pt] 
    {Partitioned empty slot};
\draw [decoration={brace, amplitude=+25pt}, decorate]
    (slot0.north west) -- (slot2.north east) node [black,midway,above=+26pt] 
    {Frame};
\draw [dashed] ($(slot2.north)!0.33!(slot2.north west)$) -- ($(slot2.south)!0.33!(slot2.south west)$);
\draw [dashed] ($(slot2.north)!0.33!(slot2.north east)$) -- ($(slot2.south)!0.33!(slot2.south east)$); 
\end{tikzpicture}
}%
    \caption{A frame with two connected devices and a three-way partitioned empty slot}
    \label{fig:frame_wsubslots}
\end{figure}
%\tikzfigure{SimpleFrame_wSubSlots.tikz}{A frame with two connected devices and a three-way partitioned empty slot}{frame_wsubslots}

% subsubsection sub_slots (end)

\subsubsection{Exponential Backoff} % (fold)
\label{ssub:exponential_backoff}
The method of exponential backoff to be used here consists of:
\begin{enumerate*}[label=\itshape \alph*\upshape)]
    \item remembering how many unvalidated announcements have been sent, this will typically be caused by collisions; and
    \item choosing a random number of frames to wait based upon this number.   
\end{enumerate*}
These properties of the exponential backoff can be expressed as follows:
\begin{equation}
     x \in [0, 2^c - 1] \mid x \in \mathbb{Z}^+, c \in [1,5]
\end{equation}
Where $x$ is a randomly chosen number of frames to wait from the range, and $c$ is number of unvalidated announcements. With this method a device will wait 0 to 1 frames after the first unvalidated announcement, 0 to 3 after the second and 0 to 7 after the third. To avoid an infinite increase of the wait time the $c$ should have an upper limit of e.g. 5, which would ensure a maximum wait of $2^5-1 = 31$ frames.

% subsubsection exponential_backoff (end) 
% subsection introduction_of_sub_slots_and_exponential_backoff (end)

\subsection{Changes to the Pseudocode} % (fold)
\label{sub:changes_to_the_pseudocode}
Because the problem of multiple devices connecting simultaneously only exists during the initialisation phase, the change to the pseudocode from \myref{sec:Pseudo} will only affect \myref{sub:setupCCRC}.
Therefore the following modification will focus on line 15-32 (both inclusive) from \myref{lst:setupCCRC}, which deals with what happens if a network is found.
In the modified pseudocode some new variables and constants are introduced, these are presented and explained in \myref{tab:locals_wmulticonnect}.

\begin{table}[h]
    {\setlength{\extrarowheight}{1ex}%
    \begin{tabularx}{\textwidth}{l|l|X|l}
        \toprule
        Name                & Type      & Description & Constraint \\
        \midrule
        $n_s$               & integer   & Number of sub slot in the empty slot  & $1 \leq n_s$      \\
        $s$                 & integer   & A sub slot in the empty slot          & $s \in [1, n_s]$  \\
        $\delta_{empty}$    & integer   & The time length of the empty slot     \\
        $\delta_{sub}$      & integer   & The time length of a sub slot         & $\delta_{empty} \text{ mod } \delta_{sub} = 0$ \\
        $valid$             & boolean   & Indication of validation              \\
        $c$                 & integer   & Number of failed announcements        & $0 \leq c \geq c_{max}$ \\
        $c_{max}$           & integer   & Ceiling of $c$                        \\
        \emph{backoff}      & integer   & Number of frames to wait              & \emph{backoff}$ \in [0, 2^c-1]$ \\
        \bottomrule
    \end{tabularx}}
    \caption{Additional local variables every device has access to.}
    \label{tab:locals_wmulticonnect}
\end{table}

The modified pseudocode is significantly more complex than the previous verison of the connecting phase.
This is the case because of the two-step strategy previously discussed, which requires addition of several looping and conditional constructs.
In \myref{lst:pseudoInitMulti} the 17 lines from \myref{lst:setupCCRC}, which concerned connecting, have been expanded to 54 lines of code. 

\begin{lstlisting}[style=pseudocode, mathescape=true, float, floatplacement=p, caption={Modified section of initialise procedure for when network found.}, label=lst:pseudoInitMulti]
label: $\text{``wait for empty slot''}$
// Wait for empty slot
while $i \neq n$ do
    $i \leftarrow (i \text{ mod } n) + 1$
    while $x \leq \delta$ do
        if received($i', n', id', data'$) then
            protocolMaintance($i', n'$, $|data'|$)
            break
        end
    end
    wait until $x \geq \delta$
    $x \leftarrow 0$
end
$s \leftarrow random\ \in [1, n_s]$
while $x \leq \delta_{sub} \times (s-1)$ do
    if received($id'$) then
        $x \leftarrow 0$
        goto $\text{``wait for empty slot''}$
    end
end
// Announce 
transmit($id$)
$x \leftarrow 0$
$i \leftarrow 0$
$valid \leftarrow \bot$
while $i \neq n$ do
    run userCode() until $x \geq \delta_{proc}$
    $i \leftarrow (i \text{ mod } n) + 1$
    while $x \leq \delta$ do
        if received($i'$, $n'$, $id'$, $mode'$, $data'$) then
            protocolMaintance($i'$, $n'$, $|data'|$)
            if validated($mode'$, $data'$) then
                $valid \leftarrow \top$
            end    
        end
    end
    wait until $x \geq \delta$
    $x \leftarrow 0$
end
if $valid$ then
    // Obtain timeslot
    $k \leftarrow n$
    mainLoop()
else
    if $c < c_{max}$ then
        $c \leftarrow c + 1$
    end
    $\text{\emph{backoff}}\leftarrow random \in [0, 2^c-1]$
    while $\text{\emph{backoff}} > 0$ do
        wait 1 frame
        $\text{\emph{backoff}}\leftarrow \text{\emph{backoff}}-1$ 
    end
    goto $\text{``wait for empty slot''}$
end
\end{lstlisting}

% subsection changes_to_the_pseudocode (end)

% section simultaneous_connect (end) 