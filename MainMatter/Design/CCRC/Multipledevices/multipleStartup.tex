\section{Multiple Start-up Issue}\label{sec:MSI-CCRC}
%The activation of multiple devices at any point in time such that two or more devices are active but not in a network causes a problem that must be handled differently for each of the two aforementioned cases.
The first case, handled in this section, is in the case where no network exists prior to the activation of devices.
As it has been established earlier, the time required to guarantee no network exists is $\delta \times 3$.
From the moment a device has determined that no network exists, it spends $\delta$ time to announce the existence of a network.
As such if several devices are activated within $\delta$ time, relative to the firstly activated device, multiple networks will be established.
By this it follows that if one can guarantee that devices start with an offset of at least $\delta$ the problem is solved for every case.
In an effort to solve this problem several different approaches can help, however none of them can guarantee that a multiple of networks will never occur with 100 \% certainty. 
As such implementing concepts from several of the following approaches may yield the best result.

\subsection{Deterministic Solutions}
Firstly deterministic solutions will be described.
These solutions are easy to check for errors as the result of any operation is predictable, as this limits the statespace to the amount of start configurations possible.

\subsubsection{Unique Offset}
Unique offset is when each device listens for a network for a uniquely specified amount of $\delta$ rather than the initial wait time of $\delta \times 3$.
Using the ID of each device one could guarantee that no device used the same multiple of $\delta$ prior to establishing their network by the formula $\text{\textit{initial wait time}} + \delta \times ID$.

This however assumes devices are activated either simultaneously or continuously with rising addresses for it to work.
In the case where initially the device with ID 2 is turned on and exactly $\delta$ time later the device with ID 1 is turned on they would still both create a network, as such this is not a valid solution in itself to the problem.
However this approach could significantly increase the chance of only creating one network in situations where all the devices are started simultaneously, e.g. a scenario where a multiple of devices are connected to the same power grid, which is then turned off and on again.
Hereby making it appropriate for a uniform start-up process.

\subsubsection{Kill the Network (Deterministic)}\label{KtN}
If a device is alone in a network for too long it may decide to kill the current network and start listening again.
This works as a recovery technique, allowing two networks to be established but then solving that very problem.
The technique has the network disassemble itself once some factor determines it needs to.

This factor could be when a number of frames has passed without hearing from any other device.
For this to actually work there would have to be implemented something which guarantees that networks would not follow the same pattern, as that could result in a loop of killing and creating networks.
Another problem that occurs is that if more than two networks are created, for this to work all but one network must die within two frames due to the previously set time spent listening.

\subsection{Stochastic Solutions}
The deterministic solutions could be used, but there will always be a chance for ending up in an infinite loop of unsuccessful network creations.
In order to avoid that a random factor is introduced.
With a stochastic approach whether or not the network is established is now based on probability.
Using these methods will guarantee that a network will be created at some point.
This might take a while but given infinite tries it is bound to be successful at some point.


\subsubsection{Randomly Create}\label{RCreate}
Once a device is done listening for a network it either creates its own network or resets and starts listening again.
For this each device would need to implement a chance factor to determine whether or not to create a network.
This could be determined by running an algorithm on the result of a random seed, depending on how the algorithm is designed one can obtain whichever chance is preferred.

The problem with this is while it gives the opportunity for two devices to start within the same time-slot without both creating a network at the same time, this is no guarantee; as such this would further require the implementation of a recovery method much like unique offset would.

\subsubsection{Kill the Network (Stochastic)}\label{KtNR}
A version of kill the network could be implemented using a random chance to kill the network if the device is alone.
This would avoid the problem of multiple devices following the same pattern.
As the pattern would be random a scenario where only one device has created a network, and the rest are listening for it is inevitable given a long enough run.
In other words at some point an isolated network should exist to which other devices can join.

One could also implement exponential back-off as described in \myref{sec:exponential_backoff}. This would result in exponential increasing duration where devices which were unsuccessful in creating a network would listen for other networks.
This could worsen the time it takes to set-up at network exponentially.
To avoid this one could use another factor which is unique for each device and use this to ensure that each device will eventually generate different numbers.
This also solves the problem with the two frames limitation.

\subsection{Modifications to the Upstart Phase}                 
Based on the aforementioned possibilities a new upstart phase have been designed which minimises the probabilities of devices creating several networks.
The changes are explained in the following text, as well as in the flowchart in \myref{fig:pseudo_flowMultiStart} representing the new design.

\begin{description}[labelindent=\parindent]
    \item[Search for network]\hfill\\
    The initial action after starting a device states to search for a network.
    This is done for $\text{\textit{initial wait time}} + \delta \times b$ where $b$ denotes a random number chosen using the method of exponential backoff from \myref[name]{sec:exponential_backoff}, i.e. $b \in [0, 2^c-1]$, where $c$ is the number of attempts of creating a network.
    This expression ensures that devices will not listen for the same amount of time when they are started, and this difference will increase with each attempt, ensuring the existence scenarios in which only one device will create a network while others are still listening, also in the case where they died at the same time.
    \item[Chance of creating a network]\hfill\\
    This method will not be used in order to keep the method of handling multiple start-up deterministic.
    \item[Alone in network?]\hfill\\
    For this decision the device checks whether or not it is alone in the network; if it is an altered version of \textsc{MainLoop()} is run, making use of the kill network method depending on the result of \textbf{Time to Die?}, as two or more networks could be jamming each other.
    \item[Time to Die?]\hfill\\
    This decision is completely random based on a given chance of killing the device.
\end{description}

\begin{figure}[p]
\centering
\footnotesize
\tikzstyle{decision} = [diamond, draw, fill=yellow!20, 
    text width=4.5em, text badly centered, inner sep=3pt]
\tikzstyle{block} = [rectangle, draw, fill=green!40, 
    text width=6em, text centered, rounded corners, minimum height=4em]
\tikzstyle{line} = [draw, -latex']
\tikzstyle{cloud} = [draw, ellipse,fill=red!20,
    minimum height=2em, align=center]
\tikzstyle{preDefProc} = [draw,rectangle split, rectangle split horizontal,rectangle split parts=3, fill=green!40,minimum height=4em]
%\node [preDefProc, right of=search] (test){\nodepart{two}{textHere}};
\resizebox{1\textwidth}{!}{
\begin{tikzpicture}[node distance = 2.8cm, auto]
    % Place nodes
    \node [cloud, align=center] (start) {Start\\$c \leftarrow 0$};
    \node [block, right of=start, text width=8em] (search) {Search for network for $t_1$ miliseconds};
    \node [decision, below of=search] (found) {Found network?};
    \node [preDefProc, right of=found, node distance=3.5cm] (join) {\nodepart{two}{Join network}};
    \node [block, below of=found, node distance = 3.25cm] (tocreate) {Create Network};
    \node [block, below of=tocreate] (init) {Initialize network};
    \node [decision, below of=init] (unitcount) {Alone in network?};
    \node [preDefProc, below of=unitcount, node distance=3cm] (mainLoop) {\nodepart{two}{Main Loop}};
    \node [block, right of=unitcount, node distance=3.5cm] (wait) {Wait for next time-slot};
    \node [block, right of=wait, node distance =3.25cm] (userCode) {Run User Code};
    \node [decision, below of=userCode] (myTime) {My timeslot?};
    \node [block, below of=myTime, node distance = 3.25cm] (receive) {Receive};
    \node [decision, below of=receive] (rmsg) {Received Message?};
    \node [block, right of=rmsg, node distance = 3.25cm] (pmsg) {Process Message};
    \node [block, right of=myTime, node distance = 3.25cm] (payload) {Make Payload}; 
    \node [block, above of=payload] (transmit) {Transmit};
    \node [decision, left of=receive, node distance =3.25cm] (perhapsdie) {Time to die?};
    \node [block, left of=perhapsdie, node distance=3.5cm] (seppuku) {$c \leftarrow c + 1 $\\Kill network};
    
    

    % Draw edges
    \draw [line] (start) -- (search);
    \draw [line] (search) -- (found);
    \draw [line] (found) -- node {yes} (join);
    \draw [line] (found) -- node {no} (tocreate);
    \draw [line] (tocreate) --  (init);
    \draw [line] (init) -- (unitcount);
    \draw [line] (unitcount) -- node {yes} (wait);
    \draw [line] (unitcount) -- node {no} (mainLoop);
    \draw [line] (wait) -- (userCode);
    \draw [line] (userCode) -- (myTime);
    \draw [line] (myTime) -- node {no} (receive);
    \draw [line] (receive) -- (rmsg);
    \draw [line] (rmsg) -- node {yes} (pmsg);
    \node [right of=pmsg, node distance =2cm] (rmsgp){};
    \draw [-] (pmsg) -| (rmsgp.center);
    \node [above of=unitcount, node distance =1.75cm] (aucp) {};
    \draw [line] (rmsgp.center) |- (aucp.center);
    \node [above of=transmit, node distance =1.25cm] (a_transmit){};
    \draw [line] (rmsg) -| node {no} (perhapsdie);
    \draw [line] (a_transmit.center) -| (wait);
    \draw [line] (myTime) -- node {yes} (payload);
    \draw [line] (payload) -- (transmit);
    \draw [-] (transmit) -- (a_transmit.center);
    \draw [line] (perhapsdie) -- node {no} (wait);
    \draw [line] (perhapsdie) -- node {yes} (seppuku);
    \node [draw=none, left of=seppuku, node distance=1.95cm] (seppup) {};
    \draw [-] (seppuku) -- (seppup.center);
    \node [draw=none, below of=search, node distance=0.60cm] (seap){};
    \node [draw=none, left of=seap, node distance=1.4cm] (seapp){};
    \draw [line] (seppup.center) |- (seapp);


\end{tikzpicture}                                   
}
\vspace{-15pt}
\caption{Revised flow diagram showing how a device acts during the Initialization phase if no networks are detected.}
\label{fig:pseudo_flowMultiStart}
\vspace{-10pt}    
\end{figure}

