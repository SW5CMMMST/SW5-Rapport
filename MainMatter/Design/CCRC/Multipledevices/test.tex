%!TEX root = ../../../../master.tex
\newpage
\section{Testing on Arduino}\label{sec:ccrc_test}
The solution for the Simultaneous Connect Problem described in \myref[name]{sec:simultaneous_connect} have been implemented on the Arduino platform along with the multiple start-up.
This is to test if the protocol works in a real life scenario. 
Sourcecode used in this test can be found on the CD attached in the back of the paper.
The test consisted of four devices and the following method was used:
\begin{eletterate}
	\item Start a single device.
	\item After a small time period start the other three devices
	\item Log timestamps for when each device joins the existing network.
\end{eletterate}

\subsection{Results of Test}
The results of the test is shown in \myref{tab:ccrc_test}.
It shows that out of the five test runs, the lowest time it took to make a network of all devices took $13 358$ milliseconds and the highest took $31 959$ milliseconds.
The average was $18780$ and the median $17131$.

The \myref[name]{sec:simultaneous_connect} which was confirmed to be working in UPPAAL SMC also works on Arduinos, where they will eventually connect to the same network, each with different time-slots assigned to each Arduino.

The test also suffers from the fact that the Arduinos do have a chance to miss transmissions, which might result in the devices not being verified to join, and going into backoff without being jammed by another device.
This is not really a problem in terms of correction, but it does impact the speed of the network.

\begin{table}[H]
\footnotesize
\centering  
\rowcolors{2}{GoogleBlueGrey!50}{White}
\begin{tabular}{r | r r r r}
        & Device 0  & Device 1  & Device 2  & Device 3  \\\midrule
 Test 1 &   1251 ms &   6374 ms &   10019 ms&   14081 ms\\
 Test 2 &   1251 ms &   6373 ms &   11948 ms&   13358 ms\\
 Test 3 &   1251 ms &   10278 ms&   10831 ms&   31959 ms\\
 Test 4 &   1251 ms &   13658 ms&   16080 ms&   17372 ms\\
 Test 5 &   1251 ms &   8572 ms &   12502 ms&   17131 ms\\   
\end{tabular}
\caption{Table showing how long time it took (relatively to device 0) for other devices to join the network it made. Note that device 1-3 are not the same in each of the tests, but are sorted in order of joining the network.}
\label{tab:ccrc_test}
\end{table}

\subsection{Testing Simultaneous Start-up of Arduinos}
The simultaneous start-up of devices has also been implemented, where a device has a chance of killing the network it has created if it is alone after the empty-slot.
This has simply been tested by having two identical Arduinos turn on at the same time by connecting their reset pins with a wire, and measuring the time it takes for them to connect to the same network.
The time it took for the two devices to connect to the same network can be seen on \myref{graphConnectime}.
From the data you can determine that the average time to connect to a network is 4458.8 ms with a minimum of 2733 ms and a maximum of 7862 ms.
This time could be improved by implementing a method of avoidance, i.e. starting $c$ greater than zero, thus the time before each device creates the first network is random as well, in addition to the recovery used by killing the network.
If the aforementioned test were to be run on the old implementation, which does not support simultaneous start-up, the two Arduinos would never succeed in creating a single network with both devices connected.
Instead the Arduinos would both create a network, which in turn would collide with each other and as a result of that, jamming would constantly occur.

\begin{table}[H]
\centering\footnotesize
\begin{tabular}{c | c | c | c | c }
Test 1 & Test 2 & Test 3 & Test 4 & Test 5 \\\midrule
6301 ms & 4458 ms & 3299 ms & 7862 ms & 3865 ms \\
\end{tabular}\\
\vspace{15pt}
\begin{tabular}{c | c | c | c | c }
Test 6 & Test 7 & Test 8 & Test 9 & Test 10\\\midrule
3299 ms & 3865 ms & 2733 ms & 4448 ms & 4458 ms \\
\end{tabular}
\caption{Data from test of simultaneous startup on Arduinos. Shows the time it takes two simultaneously started devices to connect to the same network.}
\label{graphConnectime}
\end{table} 