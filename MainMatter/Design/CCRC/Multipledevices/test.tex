%!TEX root = ../../../../master.tex
\section{Testing on Arduino}\label{sec:ccrc_test}
The solution for the Simultaneous Connect Problem described in \myref[name]{sec:simultaneous_connect} have been implemented on the Arduino platform along with the multiple startup.
This is to test if the protocol works in a real life scenario. 

The code can be found on the CD attached in the back of the paper.

The test consisted of four devices and the following method was used:
\begin{eletterate}
	\item Start a single device.
	\item After a small time period start the other three devices
	\item Log timestamps for when each device joins.
\end{eletterate}

\subsection{Results of Test}
The results of the test is shown in \myref{tab:ccrc_test}.
It shows that out of the five test runs, the lowest time it took to make a network of all devices took $13 358$ milliseconds and the highest took $31 959$ milliseconds.
The average was $18780$ and the median $17131$.

The \myref[name]{sec:simultaneous_connect} which was confirmed to be working in UPPAAL SMC also works on Arduinos, where they will eventually connect to the same network, each with different time-slots assigned to each Arduino.

The test also suffers from the fact that the Arduinos do have a chance to miss transmissions, which might result in the devices not being verified to join, and going into backoff without being jammed by another device.
This is not really a problem in terms of correction, but it does impact the speed of the network.

\begin{table}[ht]
\centering
\begin{tabular}{lllll}
                            & Device 0                  & Device 1                   & Device 2                   & Device 3                   \\ \cline{2-5} 
\multicolumn{1}{r|}{Test 1} & \multicolumn{1}{r|}{$1251$} & \multicolumn{1}{r|}{$6374$}  & \multicolumn{1}{r|}{$10019$} & \multicolumn{1}{r|}{$14081$} \\ \cline{2-5} 
\multicolumn{1}{r|}{Test 2} & \multicolumn{1}{r|}{$1251$} & \multicolumn{1}{r|}{$6373$}  & \multicolumn{1}{r|}{$11948$} & \multicolumn{1}{r|}{$13358$} \\ \cline{2-5} 
\multicolumn{1}{r|}{Test 3} & \multicolumn{1}{r|}{$1251$} & \multicolumn{1}{r|}{$10278$} & \multicolumn{1}{r|}{$10831$} & \multicolumn{1}{r|}{$31959$} \\ \cline{2-5} 
\multicolumn{1}{r|}{Test 4} & \multicolumn{1}{r|}{$1251$} & \multicolumn{1}{r|}{$13658$} & \multicolumn{1}{r|}{$16080$} & \multicolumn{1}{r|}{$17372$} \\ \cline{2-5} 
\multicolumn{1}{r|}{Test 5} & \multicolumn{1}{r|}{$1251$} & \multicolumn{1}{r|}{$8572$}  & \multicolumn{1}{r|}{$12502$} & \multicolumn{1}{r|}{$17131$} \\ \cline{2-5} 
\end{tabular}
\caption{Table showing how long time it took, in milliseconds, (relatively to device 0) for other devices to join the network it made. Note that device 1-3 is not the same in each of the tests, but are sorted in order of joining the network.}
\label{tab:ccrc_test}
\end{table}

\subsection{Testing Simultaneous Startup of Arduinos}
The simultaneous startup of devices has also been implemented, where a device has a chance of killing the network it has created if it is alone after the empty-slot.
This has simply been tested by having two identical Arduinos turn on at the same time by connecting their reset pins with a wire, and measuring the time it takes for them to connect to the same network.
The time it took for the two devices to connect to the same network can be seen on \myref{graphConnectime}.
From the data you can determine that the average time to connect to a network is 4,458.8 ms. 

\begin{table}[ht]
\centering\scriptsize
\begin{tabularx}{\textwidth}{X|c c c c c c c c c c}
                          & Test 1 & Test 2 & Test 3 & Test 4 & Test 5 & Test 6 & Test 7 & Test 8 & Test 9 & Test 10\\
                          \midrule
    Time to connect (ms)  & 6301   & 4458   & 3299   & 7862   & 3865   & 3299   & 3865   & 2733   & 4448   & 4458
\end{tabularx}
\caption{CAPTION GOES HERE}
\label{graphConnectime}
\end{table}


min 2733 ms
avg 4458.8 ms
max 7862 ms