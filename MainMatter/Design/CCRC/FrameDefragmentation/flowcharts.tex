\section{Additions to the Flowchart}
These changes are to be put into the flowchart from \myref{fig:main_psuedo_flow}.

\tikzfigure{PsuedoMainFlowDiagramFrameDefragmaybe}{Placeholder txt.}{mainFrameDefrag}


The first flowchart, \myref{fig:jd1}, is necessary for the fist and last device to monitor if they should change the frame, here the last device increments a counter for each device from which it does not receive a payload.
\begin{figure}[!p]
    \footnotesize
    \begin{subfigure}{0.48\linewidth}
        \centering
        \begin{tikzpicture}[auto, on grid]           
            \node [cloud] (start) {Start};
            \node [decision, below =2.2cm of start] (decide) {$k = 1 \lor k = n-1$?};    
            \node [draw=none, below =2.2cm of decide] (a) {};
            \node [block, left =1.8cm of a] (increment) {Increment missed counter};   
            \node [cloud, right =1.8cm of a] (end) {End};
            
            \path [line] (start) -- (decide);
            \path [line] (decide) -| node [near end] {yes} (increment);
            \path [line] (decide) -| node [near end] {no} (end);
            \path [line] (increment) -- (end);
        \end{tikzpicture}
        \caption{Increment missed counter subroutine}
        \label{fig:jd1a}
    \end{subfigure} \hfill
    \begin{subfigure}{0.48\linewidth}
        \centering
        \begin{tikzpicture}[auto, on grid]
            \node [cloud] (start) {Start};
            \node [decision, below =2.2cm of start] (decide) {$k = 1 \lor k = n-1$?};    
            \node [draw=none, below =2.2cm of decide] (a) {};
            \node [block, left =1.8cm of a] (reset) {Reset counter for device received from};   
            \node [cloud, right =1.8cm of a] (end) {End};
            
            \path [line] (start) -- (decide);
            \path [line] (decide) -| node [near end] {yes} (reset);
            \path [line] (decide) -| node [near end] {no} (end);
            \path [line] (reset) -- (end); 
        \end{tikzpicture}
        \caption{Increment counter subroutine}
        \label{fig:jd1b}
    \end{subfigure}
    \caption{The needed changes to main loop, to count which devices misses their times-slot.}
    \label{fig:jd1}
\end{figure}


The second flowchart, \myref{fig:jd2}, is the jump itself.
If the counter for any timeslot is higher than $m_{max}$, then it should change its $k$ to the $k$ of the device which missed counter was higher than $m_{max}$.
This makes the last device of the frame now have a new k, and thus it has performed the jump.	
\tikzfigure{DefragmentationFlowchart_2.tikz}{The Overtake subroutine. The needed changes to main loop to performing the jump.}{jd2}

The third flowchart, \myref{fig:jd3}, shows how to handle the last device of the frame leaving.
\tikzfigure{DefragmentationFlowchart_3.tikz}{The Reduce subroutine. The needed changes to main loop to decrease the number of time-slots in the network.}{jd3}