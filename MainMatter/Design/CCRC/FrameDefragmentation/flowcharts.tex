\section{Additions to the Flowchart}
There are added four changes to the main loop flowchart from \myref{fig:main_psuedo_flow}, as can be seen in \myref{fig:mainFrameDefrag}.

The first additions can be seen detailed on the flowchart in, \myref{fig:jd1},
It is necessary for the fist and last device to monitor if they should change the frame, in \myref{fig:jd1a} the last device increments a counter for each device from which it does not receive a payload.
In \myref{fig:jd1b} the counter is reset for a device when a payload is received.

The additions presented in \myref{fig:jd2}, is the jump itself.
If the counter for any timeslot is higher than $m_{max}$, then it should change its place in the frame, $k$, to the $k$ of the device which missed counter was higher than $m_{max}$.
This makes the last device of the frame now have a new $k$, and thus it has performed the jump.	

The last additions seen in \myref{fig:jd3}, shows how to handle the last device of the frame leaving the network and thus reducing the number of time-slots.
                                                                         
\begin{figure}[!p] 
\vspace{-15pt} 
\centering
\footnotesize
\begin{tikzpicture}[node distance = 3.4cm, auto]
    % Place nodes
    \node [block] (wait) {User-code for $\delta_{proc}$};
    \node [decision, right of=wait, node distance=3.4cm] (equal) {Is it my timeslot?};
    \node [preDefProc, below of=wait] (missCount) {\nodepart{two}{Increment missed counter}};
    \node [preDefProc, above of=equal, node distance=3cm] (reduce) {\nodepart{two}{Reduce Network}};
    \node [block, above of=reduce, node distance=2cm] (make) {Make payload};
    \node [block, left of=make] (tx) {Transmit};
    \node [preDefProc, left of=tx, node distance=3.6cm] (overtake) {\nodepart{two}{Overtake}};
    \node [decision, below of=missCount] (recieved) {Received message?};
    \node [block, right of=recieved] (rx) {Recieve for max $\delta_{comm}$};
    \node [preDefProc, left of=recieved, node distance = 3.6cm] (resetCount) {\nodepart{two}{Reset counter}};
    \node [block, above of=resetCount] (compute) {Process message};
    \node [block, left of=wait, node distance=3.6cm] (wait2) {Wait until end of timeslot};
    \node [cloud, left of=wait2] (start) {Start};

    % Draw edges
    \path [line] (start) -- (wait2);
    \path [line] (wait) -- (equal);
    \path [line] (equal) -- node {no} (rx);
    \path [line] (equal) -- node {yes} (reduce);
    \path [line] (reduce) -- (make);
    \path [line] (make) -- (tx);
    \path [line] (tx) -- (overtake);
    \path [line] (overtake) -- (wait2);
    \path [line] (missCount) -- (wait);
    \path [line] (compute) -- (wait2);
    \path [line] (rx) -- (recieved);
    \path [line] (recieved) -- node {yes} (resetCount);
    \path [line] (resetCount) -- (compute);
    \path [line] (recieved) -- node [near start] {no} (missCount);
    \path [line] (wait2) -- (wait);
\end{tikzpicture}
\caption{The main loop loop with the four additional processes.}
\label{fig:mainFrameDefrag}   
\end{figure}  

\begin{figure}[!h]
    \begin{subfigure}{\linewidth}
        \centering
        \begin{tikzpicture}[auto]
            \node [decision]        (iflastdevice) at (0,0)  {If first or last device in frame?};    
            \node [block, right = of iflastdevice]       (increment)     {Increment missed counter};
            \node [cloud_nospace, right = of increment]       (dummy1)     {End};
            \node [cloud_nospace, left = of iflastdevice]       (dummy2)     {Start};
            \path [line] (iflastdevice) -- node {yes} (increment);
            \path [line] (iflastdevice.north) -| node [near start] {no} (dummy1);
            \path [line] (increment) -- (dummy1);
            \path [line] (dummy2) -- (iflastdevice);
        \end{tikzpicture}
        \caption{Increment missed counter subroutine}
        \label{fig:jd1a}
    \end{subfigure}
    \begin{subfigure}{\linewidth}
        \centering
        \begin{tikzpicture}[auto]
            \node [decision]        (ifrecieved) at (0,-4)  {If first or last device in frame?};    
            \node [block, right = of ifrecieved] (reset)     {Reset counter for device received from};
            \node [cloud_nospace, right = of reset] (dummy3) {End};
            \node [cloud_nospace, left = of ifrecieved] (dummy4) {Start};
            \path [line] (ifrecieved) -- node {yes} (reset);
            \path [line] (ifrecieved.north) -| node [near start] {no} (dummy3);
            \path [line] (dummy4) -- (ifrecieved);
            \path [line] (reset) -- (dummy3);
        \end{tikzpicture}
        \caption{Increment counter subroutine}
        \label{fig:jd1b}
    \end{subfigure}
    \caption{The needed changes to main loop, to count which devices misses their times-slot.}
    \label{fig:jd1}
\end{figure}
  

\begin{figure}[!p]
\centering
\footnotesize
\begin{tikzpicture}[auto]
    \node [decision]        (iflastdevice2) at (0,0)  {If last device in frame?};
    \node [decision, above = 1cm of iflastdevice2]        (ifabovecounter)  {If any device above counter?};
    \node [block, right = of ifabovecounter] (overtake) {Overtake its position};
    \node [block, below = of overtake] (reducesize) {Reduce size of network};
    \node [cloud_nospace, right = of reducesize] (dummy5) {dummy};
    \node [cloud_nospace, left = of iflastdevice2] (dummy6) {dummy};


    \path [line] (ifabovecounter) -- node {yes} (overtake);
    \path [line] (iflastdevice2) -- node {yes} (ifabovecounter);

    \path [line] (ifabovecounter.north) -| node [near start]  {no} (dummy5);
    \path [line] (iflastdevice2) -| node [near start] {no} (dummy5);

    \path [line] (overtake) -- (reducesize);
    \path [line] (reducesize) -- (dummy5);
    \path [line] (dummy6) -- (iflastdevice2);    

\end{tikzpicture}
\caption{The Overtake subroutine, the changes to main loop to performing the jump.}
\label{fig:jd2}    
\end{figure}

\begin{figure}[!p]
\centering
\footnotesize
\resizebox{1\textwidth}{!}{
\begin{tikzpicture}[auto]
    \node [decision] (iffirstdevice) at (0,0)  {If first device in frame?};
    \node [decision, above = 1cm and 1cm of iffirstdevice]        (iflastslot)  {If last device above counter?};
    \node [block, right = of iflastslot]       (reduce)     {Reduce size of network};
    \node [cloud_nospace, right = of reduce]       (dummy3)     {Make payload};
    \node [cloud_nospace, left = of iffirstdevice, text width=7em, text centered]       (dummy4)     {Is it my timeslot?};

    \path [line] (iffirstdevice) -- node {yes} (iflastslot);
    \path [line] (iflastslot) -- node {yes} (reduce);
    \path [line] (iffirstdevice) -| node [near start] {no} (dummy3);
    \path [line] (iflastslot.north) -| node {no} (dummy3);
    \path [line] (reduce) -- (dummy3);
    \path [line] (dummy4) -- (iffirstdevice);

\end{tikzpicture}
}
\caption{The Reduce subroutine. The needed changes to main loop to decrease the number of time-slots in the network.}
\label{fig:jd3}    
\end{figure}   
