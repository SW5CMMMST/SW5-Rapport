\section{Additions to the Flowchart}
These changes are to be put into the flowchart from \myref{fig:main_psuedo_flow}.

\tikzfigure{PsuedoMainFlowDiagramFrameDefragmaybe}{Placeholder txt.}{mainFrameDefrag}


The first flowchart, \myref{fig:jd1}, is necessary for the fist and last device to monitor if they should change the frame, here the last device increments a counter for each device from which it does not receive a payload.
\begin{figure}[!h]
    \begin{subfigure}{\linewidth}
        \centering
        \begin{tikzpicture}[auto]
            \node [decision]        (iflastdevice) at (0,0)  {If first or last device in frame?};    
            \node [block, right = of iflastdevice]       (increment)     {Increment missed counter};
            \node [cloud_nospace, right = of increment]       (dummy1)     {End};
            \node [cloud_nospace, left = of iflastdevice]       (dummy2)     {Start};
            \path [line] (iflastdevice) -- node {yes} (increment);
            \path [line] (iflastdevice.north) -| node [near start] {no} (dummy1);
            \path [line] (increment) -- (dummy1);
            \path [line] (dummy2) -- (iflastdevice);
        \end{tikzpicture}
        \caption{Increment missed counter subroutine}
        \label{fig:jd1a}
    \end{subfigure}
    \begin{subfigure}{\linewidth}
        \centering
        \begin{tikzpicture}[auto]
            \node [decision]        (ifrecieved) at (0,-4)  {If first or last device in frame?};    
            \node [block, right = of ifrecieved] (reset)     {Reset counter for device received from};
            \node [cloud_nospace, right = of reset] (dummy3) {End};
            \node [cloud_nospace, left = of ifrecieved] (dummy4) {Start};
            \path [line] (ifrecieved) -- node {yes} (reset);
            \path [line] (ifrecieved.north) -| node [near start] {no} (dummy3);
            \path [line] (dummy4) -- (ifrecieved);
            \path [line] (reset) -- (dummy3);
        \end{tikzpicture}
        \caption{Increment counter subroutine}
        \label{fig:jd1b}
    \end{subfigure}
    \caption{The needed changes to main loop, to count which devices misses their times-slot.}
    \label{fig:jd1}
\end{figure}


The second flowchart, \myref{fig:jd2}, is the jump itself.
If the counter for any timeslot is higher than $m_{max}$, then it should change its $k$ to the $k$ of the device which missed counter was higher than $m_{max}$.
This makes the last device of the frame now have a new k, and thus it has performed the jump.	
\tikzfigure{DefragmentationFlowchart_2.tikz}{The Overtake subroutine. The needed changes to main loop to performing the jump.}{jd2}

The third flowchart, \myref{fig:jd3}, shows how to handle the last device of the frame leaving.
\tikzfigure{DefragmentationFlowchart_3.tikz}{The Reduce subroutine. The needed changes to main loop to decrease the number of time-slots in the network.}{jd3}