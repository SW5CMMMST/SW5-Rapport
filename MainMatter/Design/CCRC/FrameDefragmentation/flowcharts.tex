\subsection{Additions to the Flowchart}
For designing the Jumping Defragmentation-method in a flow chart four changes are added to the main loop flowchart from \myref{fig:main_psuedo_flow}, note that the flowchart form CCRC is used for the sake of simplicity, as can be seen in \myref{fig:mainFrameDefrag}.

The first additions can be seen on the flowchart in \myref{fig:jd1}.
It is necessary for the first and last device to monitor whether they should change the frame, in \myref{fig:jd1a} the first or last device increments a counter for each device from which it does not receive a payload.
In \myref{fig:jd1b} the counter is reset for a device when a payload is received.

The additions presented in \myref{fig:jd2}, is the jump itself.
If the counter for any time slot is higher than $m_{max}$, then it should change its place in the frame, $k$, to the $k$ of the device which missed counter was higher than $m_{max}$.
This makes the last device of the frame now have a new $k$, and thus it has performed the jump.	

The last additions seen in \myref{fig:jd3}, shows how to handle the last device of the frame leaving the network and thus reducing the number of time-slots.
                                                                         
\begin{figure}[!p] 
\vspace{-15pt} 
\centering
\footnotesize
\begin{tikzpicture}[node distance = 3cm, on grid, auto]
    % Place noded   
    \node [cloud] (start) {Start};
    \node [block, right of= start, node distance = 2.5cm] (waittime) {Wait until end of timeslot};
    \node [block, right of= waittime, node distance = 3.4cm] (waituser) {User-code for $\delta_{proc}$};
    \node [decision, right of= waituser] (mytime) {Is it my timeslot?};
    \node [block, below of= waittime, node distance = 2cm] (pmsg) {Process message};
    \node [preDefProc, below of= pmsg] (resetcnt) {\nodepart{two}{\begin{tabular}{c} Reset \\ counter \end{tabular}}};
    \node [preDefProc, below of= waituser, node distance = 2cm] (misscnt) {\nodepart{two}{\begin{tabular}{c} Increment \\ missed \\ counter \end{tabular}}};
    \node [decision, below of= misscnt] (received) {Received message?};  
    \node [block, right of= received] (rx) {Recieve for max $\delta_{comm}$};
    \node [preDefProc, above of= mytime] (reduce) {\nodepart{two}{\begin{tabular}{c} Reduce \\ network \\ size \end{tabular}}};
    \node [block, above of= reduce, node distance = 2cm] (make) {Make payload}; 
    \node [block, left of= make] (tx) {Transmit}; 
    \node [preDefProc, left of= tx, node distance = 3.4cm] (overtake) {\nodepart{two}{\begin{tabular}{c} Overtake \\ frame \\ position \end{tabular}}};
     
    % Draw edges
    \path [line] (start) -- (waittime);
    \path [line] (waittime) -- (waituser);
    \path [line] (waituser) -- (mytime);
    \path [line] (mytime) -- node {yes} (reduce);
    \path [line] (reduce) -- (make);
    \path [line] (make) -- (tx);
    \path [line] (tx) -- (overtake);
    \path [line] (overtake) -- (waittime);
    \path [line] (mytime) -- node [near start] {no} (rx); 
    \path [line] (rx) -- (received);
    \path [line] (received) -- node [near start] {yes} (resetcnt); 
    \path [line] (resetcnt) -- (pmsg);
    \path [line] (pmsg) -- (waittime);
    \path [line] (received) -- node [near start] {no} (misscnt);
    \path [line] (misscnt) -- (waituser);
    
\end{tikzpicture}
\caption{The main loop loop with the four additional processes.}
\label{fig:mainFrameDefrag}   
\end{figure}  

\begin{figure}[!p]
    \footnotesize
    \begin{subfigure}{0.48\linewidth}
        \centering
        \begin{tikzpicture}[auto, on grid]           
            \node [cloud] (start) {Start};
            \node [decision, below =2.2cm of start] (decide) {$k = 1 \lor k = n-1$?};    
            \node [draw=none, below =2.2cm of decide] (a) {};
            \node [block, left =1.8cm of a] (increment) {Increment missed counter};   
            \node [cloud, right =1.8cm of a] (end) {End};
            
            \path [line] (start) -- (decide);
            \path [line] (decide) -| node [near end] {yes} (increment);
            \path [line] (decide) -| node [near end] {no} (end);
            \path [line] (increment) -- (end);
        \end{tikzpicture}
        \caption{Increment missed counter subroutine}
        \label{fig:jd1a}
    \end{subfigure} \hfill
    \begin{subfigure}{0.48\linewidth}
        \centering
        \begin{tikzpicture}[auto, on grid]
            \node [cloud] (start) {Start};
            \node [decision, below =2.2cm of start] (decide) {$k = 1 \lor k = n-1$?};    
            \node [draw=none, below =2.2cm of decide] (a) {};
            \node [block, left =1.8cm of a] (reset) {Reset counter for device received from};   
            \node [cloud, right =1.8cm of a] (end) {End};
            
            \path [line] (start) -- (decide);
            \path [line] (decide) -| node [near end] {yes} (reset);
            \path [line] (decide) -| node [near end] {no} (end);
            \path [line] (reset) -- (end); 
        \end{tikzpicture}
        \caption{Increment counter subroutine}
        \label{fig:jd1b}
    \end{subfigure}
    \caption{The needed changes to main loop, to count which devices misses their times-slot.}
    \label{fig:jd1}
\end{figure}
  

\begin{figure}[!p]
\centering
\footnotesize
\begin{tikzpicture}[auto]
    \node [decision]        (iflastdevice2) at (0,0)  {If last device in frame?};
    \node [decision, above = 1cm of iflastdevice2]        (ifabovecounter)  {If any device above counter?};
    \node [block, right = of ifabovecounter] (overtake) {Overtake its position};
    \node [block, below = of overtake] (reducesize) {Reduce size of network};
    \node [cloud_nospace, right = of reducesize] (dummy5) {dummy};
    \node [cloud_nospace, left = of iflastdevice2] (dummy6) {dummy};


    \path [line] (ifabovecounter) -- node {yes} (overtake);
    \path [line] (iflastdevice2) -- node {yes} (ifabovecounter);

    \path [line] (ifabovecounter.north) -| node [near start]  {no} (dummy5);
    \path [line] (iflastdevice2) -| node [near start] {no} (dummy5);

    \path [line] (overtake) -- (reducesize);
    \path [line] (reducesize) -- (dummy5);
    \path [line] (dummy6) -- (iflastdevice2);    

\end{tikzpicture}
\caption{The Overtake subroutine, the changes to main loop to performing the jump.}
\label{fig:jd2}    
\end{figure}

\begin{figure}[!p]
\centering
\footnotesize
\resizebox{1\textwidth}{!}{
\begin{tikzpicture}[auto]
    \node [decision] (iffirstdevice) at (0,0)  {If first device in frame?};
    \node [decision, above = 1cm and 1cm of iffirstdevice]        (iflastslot)  {If last device above counter?};
    \node [block, right = of iflastslot]       (reduce)     {Reduce size of network};
    \node [cloud_nospace, right = of reduce]       (dummy3)     {End};
    \node [cloud_nospace, left = of iffirstdevice]       (dummy4)     {Start};

    \path [line] (iffirstdevice) -- node {yes} (iflastslot);
    \path [line] (iflastslot) -- node {yes} (reduce);
    \path [line] (iffirstdevice) -| node [near start] {no} (dummy3);
    \path [line] (iflastslot.north) -| node {no} (dummy3);
    \path [line] (reduce) -- (dummy3);
    \path [line] (dummy4) -- (iffirstdevice);

\end{tikzpicture}
}
\caption{The Reduce subroutine. The needed changes to main loop to decrease the number of time-slots in the network.}
\label{fig:jd3}    
\end{figure}   
