\subsection{Initializing} % (fold)
\label{sub:setupCCRC} 
When first turning on a device, which should be part of the network, the first task of the device must be to try to connect to an existing network.
First the device will be set to listening for any signal for a given time period. 
If the device does not pick up any transmission during this period the device will start initialising its own network.

When a device creates a network, the network contains two times slots; one for the device itself and one empty slot, which allows another device to connect to the network.
When the frame is in the device's own time slot it will transmit so new devices can detect said network, when they are listening in their start-up period.  
A flow diagram of this setup can be seen on \myref{fig:psuedo_flowCCRC}.

\tikzfigure{PsuedoFlowDiagramRC}{Flow diagram showing how a device acts during the Initialization phase}{psuedo_flowCCRC}
 
\noindent
\textsc{Initialize()}, the function in \myref{lst:setupCCRC}, performs the initialization of the device.
This includes searching for a network or establishing its own network if no network can be detected.
\myref{lst:setupCCRC} introduces a new constant $t_0$, this constant decides when the device should give up listening for a network and instead establish a network.

For the completely connected reliable communication case, represented in \myref{chp:Problems}, $t_0$ has to be at least the length of $2 \times \delta + \delta_{comm}$, however it being slightly longer taking guard time and the chance of a slight miss synchronization into consideration $3 \times \delta$ will be considered the minimum. 
This worst case scenario can be seen on \myref{fig:worsttimeccrc}.
This scenario involves a connecting device starting to connect after the last device of the frame has begun its transmission, then it has to wait for the first device of the next frame to transmit before it discovers the network. 

\tikzfigure{WorstCaseCCRCSingleConnect.tex}{Diagram showing worst case scenario for a device connecting to a network.}{worsttimeccrc} 

\noindent%
The minimum is calculated from the case where the device is turned on after the $n-1$'th slot has started, but not yet ended.
It is worth noting here that this minimum is only true for this case, both strongly connected and non-reliable communication present their own reasons for this not being true for those cases.

If a network is found prior to \textit{x} exceeding $t_0$ the device will process the data and use this to properly obtain a slot in the network, this is done in the function \textsc{ProtocolMaintenance} which was explained in \myref[name]{subsec:protocolmaintanaince}.
Should it occur that no network is found the device will start its own network with the initial configuration that lines 35 - 38 in \myref{lst:setupCCRC} shows.

\begin{algorithm}[ht]
\caption{Pseudocode example of the special case procedure Initialise()}
\label{lst:setupCCRC}
\begin{algorithmic}[1]
\Procedure {Initialise}{ } 
    \State $x \gets 0$
    \State \textit{found} $\gets$ \textit{false}
    \While {$x \le t_0$}
        \If {\Call{Received}{$i', n', id', data'$}}
            \State \Call{ProtocolMaintenance}{$i', n', |data'|$} 
            \State \textit{found} $\gets$ \textit{true}
            \State \textbf{wait until } $x \ge \delta$
            \State $x \gets 0$
            \State \textbf{break}
        \EndIf                                        
    \EndWhile
    \If {$found$}
        \While {$i \neq n$}
            \State $i \gets (i \bmod n) + 1$
            \While {$x \le \delta$}
                \If {\Call{Received}{$i', n', id', data'$}}
                    \State \Call{ProtocolMaintenance}{$i', n', |data'|$}
                    \State \textbf{break}
                \EndIf
            \EndWhile
            \State \textbf{wait until } $x \ge \delta$
            \State $x \gets 0$ 
        \EndWhile
        \State \textbf{wait until } $x \ge \delta_{proc}$
        \State $k \gets n,\, n \gets n+1$ \Comment{Obtain timeslot}
        \State \Call{Transmit}{$i, n, id, data$} \Comment{Announce} 
        \State \textbf{wait until } $x \ge \delta$ \Comment{Wait for time-slot end}
    \Else
        \State $x \gets 0,\, n \gets 2,\, k \gets 1,\, i \gets 0$ \Comment{Init new network}
    \EndIf
    \State \Call{MainLoop}{ } \Comment{Enter the main loop}
\EndProcedure        
\end{algorithmic}    
\end{algorithm} 

% subsection special_cases (end)
