\subsection{Initializing} % (fold)
\label{sub:setupCCRC} 
When first turning on a device, which should be part of the network, the first task of the device must be to try to connect to an existing network.
First the device will be set to listening for any signal for a given time period. 
If the device does not pick up any transmission during this period the device will start initialising its own network.

When a device creates a network, the network contains two times slots; one for the device itself and one empty slot, which allows another device to connect to the network.
When the frame is in the device's own time slot it will transmit so new devices can detect said network, when they are listening in their startup period.  
A flow diagram of this setup can be seen on \myref{fig:psuedo_flowCCRC}

\todo{Flowdiagram som passer med reliable communication(ingen verify) skal være her, er udkommenteret i rapporten men klar til creation}
\tikzfigure{PsuedoFlowDiagramRC}{Flow diagram showing how a device acts during the Initialization phase}{psuedo_flowCCRC}
 
\bigskip \noindent
\texttt{Initialize()}, the function in \myref{lst:setupCCRC}, performs the initialization of the device.
This includes searching for a network or establishing its own network if no network can be detected.
\myref{lst:setupCCRC} introduces are new local variable $t_0$, this constant decides when the device should give up listening for a network and instead establish a network.
For the completely connected reliable communication case represented in \myref{chp:Problems} $t_0$ has to be at least the length of $2 \times \delta + \delta_{com}$, however it being slightly longer taking guard time and the chance of a slight miss synchronization into consideration $3 \times \delta$ will be considered the minimum.
The minimum is calculated from the case where the device is turned on after the $n-1$'th slot has started, but not yet ended.
It is worth noting here that this minimum is only true for this case, both strongly connected and non-reliable communication present their own reasons for this not being true for those cases.

If a network is found prior to \textit{x} exceeding $t_0$ the device will process the data and use this to properly attain a slot in the network as is seen on lines 7 - 9 in \myref{lst:setupCCRC}.
Should it occur that no network is found the device will start its own network with the initial configuration that lines 14 - 16 in \myref{lst:setupCCRC} shows.

\begin{minipage}{\linewidth} %minpage to avoid page break
\begin{lstlisting}[label=lst:setupCCRC,style=pseudocode,mathescape=true,caption={Pseudocode example of the special case procedure Initialize()},basicstyle=\ttfamily]
Device$_{id} =$ local $n$, $i$, $k$, $t_0$, clock $x$
procedure Initialize()
    $x \leftarrow 0$
    while $x \leq t_0$ do
        if received($i'$, $n'$, $id'$, $data'$) then
            $x \leftarrow 0$ //Reset the clock to synchronize
            //Prepare variables for transmit in emptyslot
            $n \leftarrow n' + 1$ // Increment number of slots
            $k \leftarrow n'$ // Claim empty slot
            $i \leftarrow i'$ // Get current slot
            mainLoop()
        end
    end
    // Create new network
    $n \leftarrow 2$ // Two slots; one for this device and one empty
    $k \leftarrow 1$ // Claim the first slot
    $i \leftarrow 1$ // Set turn to mine
    mainLoop()
\end{lstlisting}   
\end{minipage}
% subsection special_cases (end)