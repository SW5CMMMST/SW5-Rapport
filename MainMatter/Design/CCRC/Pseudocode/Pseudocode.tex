\section{Pseudocode Description}\label{sec:Pseudo}
This section presents key components of the designed TDMA-based protocol. 
First an informal description is given followed by relevant flowcharts and pseudocode examples. 
The model of the protocol is divided into the two phases - the initilization phase and the loop phase. 

These phases will be described in the following subsections using the same model for a general device. 
That device can only know its local variables and what it receives via the \enquote{received} function.

\begin{table}[H]
    \begin{tabularx}{\textwidth}{l|l|X|l}
        \toprule
        Name                & Type      & Description & Constraint \\
        \midrule
        $n$                 & integer   & The number of slots in the network                        & $2 \leq n \leq 256$\\
        $k$                 & integer   & The slot claimed by this device                           & $1 \leq k \leq n - 1$\\
        $i$                 & integer   & The current slot                                          & $1 \leq i \leq n$\\
        $\delta$            & integer   & The duration of a time slot                               & $\delta = \delta_{com} + \delta_{proc}$\\
        $\delta_{com}$      & integer   & The duration of a time slot reserved for communication    & $0 < \delta_{com}$ \\
        $\delta_{proc}$     & integer   & The duration of a time slot reserved for processing       & $0 < \delta_{proc}$ \\
        $t_0$               & integer   & The minimum time to determine if a network exists         & $3 \times \delta \leq t_0$ \\
        $x$                 & clock     & A clock used for timing, and always running\\
        \bottomrule
    \end{tabularx}
    \caption{The local variables every device has access to.}
    \label{tab:locals}
\end{table}

The devices store their knowledge in their local variables as seen in \myref{tab:locals}. 
The aim of the network should be to manipulate those values while running, to satisfy the statements seen in \myref{tab:invariants}. 
This should ensure that two devices will not send messages at the same time.

\begin{table}[H]
	\centering
	$\begin{array}{l l l}
		(a)&\forall\ \{d_a, d_b\} \subset Network: & d_a.k \neq d_b.k \\
		(b)&\forall\ \{d_a, d_b\} \subset Network: & d_a.n = d_b.n \\
		(c)&\forall\ \{d_a, d_b\} \subset Network: & d_a.i = d_b.i \\[1ex]
		(d)&\forall\ d \in Network: & d.k \neq n \\[1ex]
		(e)&\forall\ k \in \mathbb{N}_{<n}: & \exists!\ d \in Network: d.k = k 
	\end{array}$
	\caption{The requested situation where $Network$ is the set of devices currently connected in a network.}
    \label{tab:invariants}
\end{table}

This is accomplished by having the devices agree on the number of slots and current slot while no devices have the same slot. 
Finally, no device should occupy the last slot as this is used by new devices to connect to the network.

The devices are assumed to have an implementation of the \enquote{receive} and \enquote{transmit} function.
The \enquote{transmit} function is simple, it sends a message over radio waves containing the parameters parsed into it.
The \enquote{received} function is harder, it takes some parameters which are set to the received values if any was received. 
It then returns true if a message was received; otherwise false.
                    
\subsection{General Case} % (fold)
\label{sub:general_case}

During the lifetime of the devices, they should spend most of their time in this general case or in the user code. 
This code determines whose turn it is, synchronises devices, and handles communication.

As seen in \myref{fig:main_psuedo_flow} the flow of the code is a straightforward loop with a few logical checks. 
\enquote{Make payload} and \enquote{Process message} are the only complicated processes. 
\enquote{Make payload} collects the known data into a package, that is ready to send. 
\enquote{Process message} then receives one of such packages and unpacks it. 
Unpacking would usually involve storing the data on the device.

\tikzfigure{PsuedoMainFlowDiagram}{Flow diagram showing how a device acts when connected to a network}{main_psuedo_flow}

This loop should only be stopped when the network is reconfigured. 
Reconfiguration of the network involves setting up and connecting to a network. 
These processes is described in \myref{sub:setupCCRC}.

\begin{minipage}{\linewidth} %minpage to avoid page break
\begin{lstlisting}[style=pseudocode,mathescape=true, label={lst:general_case}, caption={Pseudocode example of the main loop}] 
procedure mainLoop()
    while true do
        loopstart:
        run userCode() until $x \geq \delta_{proc}$
        $i \leftarrow (i \text{ mod } n) + 1$ // Update current slot
        if $i = k$ then
            makePayload() // Updates the data to be sendt
            transmit($i$, $n$, $id$, $data$)
            $x \leftarrow 0$ //Reset the clock to synchronize
        else
            while $x \leq \delta$ do
                if received($i'$, $n'$, $id'$, $data'$) then
                    $x \leftarrow 0$ //Reset the clock to synchronize
                    processMessage($id'$, $data'$)
                    goto loopstart
                end
            end
            $x \leftarrow 0$ //Reset the clock to synchronize
        end
    end
\end{lstlisting}  
\end{minipage}

The full pseudocode for the general case can be seen in \myref{lst:general_case}. 
In the code, there are a few extra details but the main structure is the same. 
The code describes how a device knows whether it is its turn or not. 

What is worth noting is the clock $x$ which controls the timing. 
It is reset at three different places in the code; on line 9, 13, and 18. 
The first two resets on line 9 and 13 are just before sending and just after receiving a message. 
This is the closest point in the program we can synchronize to. 

The last reset on line 13 is when no message was received in the timeslot. 
In this case, the device should reset after the calculated $\delta_{com}$ duration. 
In a completely reliable network, this should only occur in the empty slot.

% subsection general_case (end)   
            