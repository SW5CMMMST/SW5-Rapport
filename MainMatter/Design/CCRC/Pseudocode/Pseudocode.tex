\section{Pseudocode Description}\label{sec:Pseudo}
This section presents key components of the designed \gls{tdma}-based protocol. 
First an informal description is given followed by relevant flowcharts and pseudocode examples. 
The model of the protocol is divided into the two phases - the initilization phase and the loop phase. 

These phases will be described in the following subsections using the same model for a general device. 
That device can only know its local variables and what it receives via the \enquote{received} function.

\begin{table}[H]
    \begin{tabularx}{\textwidth}{l|l|X|l}
        \toprule
        Name                & Type      & Description & Constraint \\
        \midrule
        $n$                 & integer   & The number of slots in the network                        & $2 \leq n \leq 256$\\
        $k$                 & integer   & The slot claimed by this device                           & $1 \leq k \leq n - 1$\\
        $i$                 & integer   & The current slot                                          & $1 \leq i \leq n$\\
        $\delta$            & integer   & The duration of a time slot                               & $\delta = \delta_{comm} + \delta_{proc}$\\
        $\delta_{comm}$      & integer   & The duration of a time slot reserved for communication    & $0 < \delta_{comm}$ \\
        $\delta_{proc}$     & integer   & The duration of a time slot reserved for processing       & $0 < \delta_{proc}$ \\
        $t_0$               & integer   & The minimum time to determine if a network exists         & $3 \times \delta \leq t_0$ \\
        $x$                 & clock     & A clock used for timing\\
        \bottomrule
    \end{tabularx}
    \caption{The local variables every device has access to.}
    \label{tab:locals}
\end{table}

The devices store their knowledge in their local variables as seen in \myref{tab:locals}. 
The aim of the network should be to manipulate those values while running, to satisfy the statements seen in \myref{tab:invariants}. 
This should ensure that no jam should occur.

\begin{table}[H]
	\centering
	$\begin{array}{l l l}
        (a)&\forall\ d_a, d_b \in Network: & d_a.k = d_b.k  \iff d_a=d_b\\
		(b)&\forall\ d_a, d_b \in Network: & d_a.n = d_b.n \\
		(c)&\forall\ d_a, d_b \in Network: & d_a.i = d_b.i \\[1ex]
		(d)&\forall\ d \in Network: & d.k \neq n \\[1ex]
		(e)&\forall\ k \in \mathbb{N}_{<n}: & \exists!\ d \in Network: d.k = k 
	\end{array}$
	\caption{The requested situation where $Network$ is the set of devices currently connected in a network.}
    \label{tab:invariants}
\end{table}

This is accomplished by having the devices agree on the number of slots and current slot while no devices have the same slot. 
Finally, no device should occupy the last slot as this is used by new devices to connect to the network.

The devices are assumed to have an implementation of the \enquote{receive} and \enquote{transmit} function.
The \enquote{transmit} function is simple, it sends a message over radio waves containing the parameters parsed into it.
The \enquote{received} function is harder, it takes some parameters which are set to the received values if any was received. 
It then returns true if a message was received; otherwise false.
                    
\subsection{Main Loop} % (fold)
\label{sub:general_case}

During the lifetime of the devices, they should spend most of their time in this loop or in the user code. 
This code determines whose turn it is, synchronises devices, and handles communication.

As seen in \myref{fig:main_psuedo_flow} the flow of the code is a straightforward loop with a few logical checks. 
\enquote{Make payload} and \enquote{Process message} are the only complicated processes. 
\enquote{Make payload} collects the known data into a package, that is ready to send. 
\enquote{Process message} then receives one of such packages and unpacks it. 
Unpacking would usually involve storing the data on the device.

\tikzfigure{PsuedoMainFlowDiagram}{Flow diagram showing how a device acts when connected to a network}{main_psuedo_flow}

This loop should only be stopped when the network is reconfigured. 
Reconfiguration of the network involves setting up and connecting to a network. 
These processes is described in \myref{sub:setupCCRC}.

\begin{algorithm}[h]
\caption{Design example of the main loop}
\label{lst:general_case}
\begin{algorithmic}[1]
\Procedure {MainLoop}{ }
    \Repeat
        \State \textbf{run} \Call{UserCode}{ } $\text{ \textbf{until} }x \ge \delta_{proc}$ 
        \State $i \gets (i \bmod n) + 1$
        \If {$i = k$}
            \State \Call{MakePayload}{ }
            \State \Call{Transmit}{$i, n, id, data$}
        \Else
            \While {$x \le \delta$}
                \If {\Call{Received}{$i', n', id', data'$}}
                    \State \Call{ProtocolMaintenance}{$i', n', |data'|$}
                    \State \Call{UserReceive}{$id', data'$}
                \EndIf
            \EndWhile
        \EndIf
        \State \textbf{wait until } $x \le \delta$
        \State $x \gets 0$ 
    \Until {the end of time}
\EndProcedure        
\end{algorithmic}    
\end{algorithm}

The full pseudocode for the general case can be seen in \myref{lst:general_case}. 
In the code, there are a few extra details but the main structure is the same. 
The code describes how a device knows whether it is its turn or not.  

Another thing to realize is because the timing is static, the clock $x$ needs to be reset only after the duration of a full timeslot; $\delta$.
Other resyncronization is handled in the ``protocolMaintance'' procedure.
% subsection general_case (end)   
            
