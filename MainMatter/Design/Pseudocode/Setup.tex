\subsection{Setup} % (fold)
\label{sub:setup} 
While the network will spend most of its run time executing \texttt{mainLoop()} occasionally special case procedures are required, one of these is \texttt{startup}.
This procedure is run when a device is turned on, reset or for some other reason may not be part of a network. 
\texttt{startup} Performs the initial setup of the device, this includes searching for a network or establishing its own network in the case where no network can be detected.
 
\bigskip \noindent
In \myref{lst:setup} a new local variable is introduced, \textit{timeout}, this constant decides when the device should give up listening for a network and instead establish a network.
For the completely connected reliable communication case represented in \myref{chp:Problems} \textit{timeout} has to be at least the length of $2 \times \delta + \delta_{Communication}$, however it being slightly longer taking in guard time and the chance of a slight desync into consideration $3 \times \delta$ will be considered the minimum.
The minimum is calculated from the case where the device is turned on after the $n-1$'th slot has started.
It is worth noting here that this minimum is only true for this case, both strongly connected and non-reliable communication present their own reasons for this not being true for those cases.

If a network is found prior to \textit{x} exceeding \textit{timeout} the device will process the data and use this to proberly attain a slot in the network as is seen on lines 7 - 9 in \myref{lst:setup}.
Should it occur that no network is found the device will start its own network with the initial configeration that lines 14 - 16 in \myref{lst:setup} shows.
\begin{lstlisting}[label=lst:setup,style=pseudocode,mathescape=true,caption={Pseudocode example of the special case procedure startup()}]
Device$_{id} =$ local $n, i, k, timeout, \text{clock } x$
procedure startup()
	$x \leftarrow 0$
	while $x \leq timeout$ do
		if $recived(i', n', id', data')$ then
			// Join network
			$n \leftarrow n' + 1$	// Increment number of slots
			$k \leftarrow n'$		// Claim empty slot
			$i \leftarrow i'$		// Get current slot
			goto loopstart
		end
	end
	// Create new network
	$n \leftarrow 2$					// Two slots; one for this device and one empty
	$k \leftarrow 1$					// Claim the first slot
	$i \leftarrow 1$					// Set turn to mine
	loopstart:
	mainLoop()
\end{lstlisting}   
% subsection special_cases (end)