\section{Pseudocode Description}
This pseudo-code will show how our TDMA protocol works.

First the main loop of the protocol will be shown, this is the state where devices are connected to the network, and no new members are trying to join.
\bigskip

\noindent
\textbf{Variables:} 
\begin{description}[labelindent=\parindent]
    \item[$\alpha$] Timeslots per frame
    \item[$\beta$] Units in network
    \item[$\kappa$] The current unit's timeslot id in frame
    \item[$\lambda$] The current timeslot in the frame
    \item[$\rho$] Time of a timeslot
\end{description}
\bigskip

\noindent
\textbf{Description:}
\begin{itemize}
    \item Each unit gets a timeslot in the frame, this is given by their join time, meaning that the first device will have the first timeslot in each frame, and so on.
    \item There is an empty timeslot in the end of each frame which can be used by new devices to join the network.
    \item Devices currently in the network should alter their current state to adapt to the new device.
    \item In your own timeslot you transmit a message to the other devices
    \item In other devices' timeslots you receive their transmission and act upon them if applicable. Otherwise you use the time to do some preemptable execution of a calculation
\end{itemize}

\begin{lstlisting}[style=customc,mathescape=true]
startup() // This is called at startup
  network = listenForNetwork(timeout) // Listen for and return active network, otherwise stops after timeout.
  if(network != null) // If there already exists a network...
    connectToNetwork(network)
  else                // If no existing network is found...
    createNetwork()
  while(true)
    mainLoop()
  
createNetwork()
  $\alpha$ = 2
  $\beta$ = 1
  $\kappa$ = 1
  $\lambda$ = 1
  $\rho$ = THIS_UNIT_WORST_CASE_EXEC_TIME
  
connectToNetwork(network)
  $\alpha$ = network.$\alpha$
  $\beta$ = network.$\beta$
  $\kappa$ = network.$\alpha$ // This is the empty slot.
  $\lambda$ = network.$\lambda$ + 1 mod $\alpha$
  $\rho$ = max(THIS_UNIT_WORST_CASE_EXEC_TIME, network.$\rho$)
  transmitPayload(CreateAnnouncePacket())

mainLoop() // This is repeated forever after startup
  if($\lambda$ == $\kappa$) // This device's turn to transmit
  	tx_payload = makePayload() // Returns a payload with appropriate variables set.
  	transmitPayload(tx_payload)
  else // This device should receive
  	timer = startTimer($\rho$) // Starts a timer that shows time left of timeslot
  	rx_payload = receive($\rho$) // $\rho$ is max timeout to wait. 
  	if(rx_payload != null) // If a payload is received...
  	    processPayload(rx_payload)
  	while(timer.timeLeft > 0 ) // While the timeslot still has some time left
  	    executeTask(timer.timeLeft)   // Do some preemptable task for the remainder of the timeslot
  	$\lambda$ = $\lambda$ + 1 mod $\alpha$

processPayload(rx_payload)
  syncDevice(rx_payload) // Use information in payload to re-synchronise
  switch(rx_payload.type)
    case PING:
      return
    case ANNOUNCEMENT:
      $\beta$ += 1
      $\alpha$ += 1
      if(rx_payload.Worst_Case > $\rho$)
        $\rho$ = rx_payload.Worst_Case
      return
    case ACTION:
      if(rx_payload.target == $\kappa$)
        execute(rx_payload.instruction)
      return
\end{lstlisting}