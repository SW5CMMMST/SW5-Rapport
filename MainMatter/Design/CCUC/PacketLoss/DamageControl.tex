\subsection{Package Loss and Damage Control}
As one cannot reach complete reliability one must consider the effect of a packet being lost regardless of how low that chance may be.
While the effect of packet loss is different depending on what data is being transfered, for the protocol in this paper corrupt data will render the transmission useless.
On the off-chance that a device completely misses a package regardless of any avoidance methods used, it should be able to recover and continue in the network without issues; furthermore in the case that the data was of importance to the device which missed it, this information should somehow be relayed back to the device that transmitted the message.
This way the previous data can be retransmitted if it was of any importance to the device which missed it.
Another problem faced is that syncing is handled through each message, as such missing messages could cause desynchronization, however this would only occur with either a significant clock drift, or a significant amount of continous transmissions missed as the time-slots are fixed. 

\subsubsection*{Continous listening}
\subsubsection*{Bitfield Time-Slot ID Index}

Methods of Recovery/Avoiding jams/desync:
Keep listening - worked earlier, no drift/desync issues;;; How would this handle if device is next?
Bitmask ID index in msg, allows devices to know if others missed their packet, works for dead device detection as well.
