\section{Package Loss}
As we are now working with a realistic scenario of unreliable communication one must also realise that this is an impossible problem to solve completely.
One can never guarantee that a packet will be received, the best one can do is increase the odds of it happening.
%If adjustments are made to the UPPAAL model in, \myref[name]{sec:uppaalccuc}, so that it models a 2~\% packet loss for each transmission, which is more than the results in \myref{tbl:packageloss} show, the network still works approximately 96~\% of the time, without doing anything to make the network better.
%As mentioned one cannot reach 100~\% when considering unreliable communication, however this does not mean that nothing can be done.
By employing different methods to increase the odds of packets being received one can attempt to approximate a reliability of \textasciitilde100~\%.

Another aspect of unreliable communication is the effect of packet loss.
Depending on what data is being transferred the impact of packet loss differs.
In regards to small data packages such as those this protocol uses for network management, packet loss completely corrupts the payload.
As such aside from implementing methods to decrease the odds of packet loss, one must also consider what the repercussions of packet loss will bring and if possible reduce this as much as possible. 

\subsection{Methods of Avoidance}\label{sub:avoidance}
As aforementioned one can attempt to approximate 100~\% reliability in a network by reducing the odds of package loss.
In an effort to reduce package loss two ideas will be presented, both based on the concept of conditional probability, a subject of probability theory.

\subsubsection*{Message Redundancy}\label{redundancy}
If a time-slot was extended such that a device has time to transmit the same payload more than once, then the probability of the other devices not receiving the transmission will be reduced, and thus increase the reliability of the network.
While the increase in probability the time-slot length will be increased, and thus the network will become slower.
Therefore this is a trade-off between time and reliability. 

\subsubsection*{Multimessage Echo}
It could be useful to repeat the last $x$ received messages a device have received in its own time-slot, such that if other devices had not received a messages they could be received from this.
This would echo messages from devices several times over, but from different devices which in the case that the problem is between two devices could help where Message Redundancy might not.  
Furthermore this would implicitly allow for devices to hear whether or not another device heard them, given their time-slots are within the amounts of messages echoed.
Moreover echoing can increase the message length greatly, and thus will also reduce the speed of the network as message redundancy would.
The size of this reduction would be dependant on how many messages are being echoed. 