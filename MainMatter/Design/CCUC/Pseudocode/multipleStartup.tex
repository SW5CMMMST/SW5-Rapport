\section{Simultaneous Startup} % (fold)
\label{sec:simultaneous_startup}

Solve the problem this time, consider at least the three following possibilities

\noindent
Multiple Startup Ideas:

\begin{description}[labelindent=\parindent]
    \item[Random Seed]\hfill\\ 
    Creates a random seed from ID or some other number, in that time i will create network - Does not work

    \item[Chante to create]\hfill\\
    Multiple devices may still get through at once

    \item[Reset if alone]\hfill\\
    This could work, if a device is alone in a network is alone  1/20 chance for resetting
\end{description}   

\noindent
Another problem arises when two or more devices are started simultaneously while a functioning network is up and running.
If all new devices tries to connect in the first empty slot, as they should, they will jam each others messages and no one will be heard.
Moreover the jammed, and jamming, devices will never know that their message was not received by anyone, as there is no way of knowing if one transmission was disrupted or jammed.
A device should always try to connect to a network immediately after starting up.
This means that jams are bound to happen sooner or later, so some measures must be taken to avoid and recover from such situations.

A combination of the following concepts could provide avoidance and recovery from jamming, in the process of joining a network: 
\begin{description}[labelindent=\parindent]
    \item[Random Offset]\hfill\\
    When a device is announcing itself in the empty slot, it randomly chooses an offset for when to send the announcement payload.
    In the rest of the empty slot it listens for other possible devices announcing themselves.
    If such message is heard it does not attempt to announce itself in the empty slot.
    This method implies that the empty slot is significantly bigger than the time it takes to transmit the announcement payload.
    \item[Validation]\hfill\\
    You should always validate whether or not your announcement have been heard and accepted by a majority of the network (over 50 \%), before joining the network.
    \item[Payload Mode]\hfill\\
    The head of the payload should contain an extra field, which can indicate different modes or some other variable.
    If the design using modes is chosen, it would be possible for other devices implementing the protocol to do various actions upon the data of the payload, e.g. the mode can indicate that the first byte of the date contains the address of a device, which wishes or is allowed to join the network.
    \item[Exponential Back-off]\hfill\\ 
    If no one in the network responds to a device's announcement, or it is not validated, the device will use exponential back-off in regards to how many frames it should wait before trying again.
\end{description} 
\noindent
Using the concepts above it would be possible to design and implement a strategy for handling a scenario, where two or more devices tries to join the same network simultaneously.
The modifications to the exsisting design from (REF TO CCRC-design) needed, will be shown in the following paragraphs.

\subsubsection{Modification of The Payload} % (fold)
\label{ssub:modification_of_the_payload}
\textbf{--WIP--}
\begin{itemize}
    \item The head should be able to indicate mode and/or address of the joining device, along side some validation parameter.
    \item The payload should maybe be shorter for announcing devices, so transmission time will be shorter and therefore less risk of transmitting simultaneously i.e. using the random offset.
\end{itemize}
% subsubsection modification_of_the_payload (end)

\subsubsection{Introduction of Randomness} % (fold)
\label{ssub:introduction_of_randomness}
\textbf{--WIP--}
\begin{itemize}
    \item Random offset should be used in every empty slot and exponential back-off only when a jam is ``´detected''.
    \item Both in the random offset and exponential back-off.
    \item However the random offset should probably be based of off the unique address.
    \item The emptyslot should be long enough for the random offset to make sense i.e. transmission time times 3 or 4 (maybe defined by a parameter that can in- or decrease jamming-avoidance).
\end{itemize}  
% subsubsection introduction_of_randomness (end)


% section simultaneous_startup (end) 