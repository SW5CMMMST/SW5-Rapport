\section{Methods of Avoidance}
As aforementioned one can attempt to approximate 100 \% reliability in a network by reducing the odds of package loss.
In an effort to reduce package loss two ideas will be represented, both based on the concept of conditional probability, a subject of probability theory.
\myref{eq:conditionalProb} shows the formula for conditional probability in the form where it is used to calculate the odds of both event A and B occurring given they are not independent.
One might notice that given the probability of A and B are the same, which is the case for our assumed packet loss, the result may also be described as \myref{eq:conditionalProb2}.
\begin{align}
P(A \land B) &= P(A)P(B|A) \label{eq:conditionalProb} \\  
P(A \land B) &= P^{events} \label{eq:conditionalProb2}
\end{align}

\subsection{Message Redundancy}\label{redundancy}
Considering conditional probability one may consider transmitting the same payload more than once in the communication phase.
While this would increase the length of the communication phase it would also decrease the odds of any given message being lost due to the principle of conditional probability.
Each message would provide an exponential decrease in the odds of a packet being lost as is evident by \myref{eq:conditionalProb2} where $events$ would simply be the multiple of redundant messages $+ 1$, an example would be by simply sending each message twice, that would give one redundant message, with a packet loss of 2 \% this would change to a $0.02^2$ i.e. 0.04 \% chance of both messages being lost.

\subsection{Multimessage Echo}
In essence this is the same idea as \myref{redundancy} but spread out to several devices.
For this idea each device would save a number of previously received messages and make them part of its own payload.
This would echo messages from devices several times over, but from different devices which in the case that the problem is between two devices could help where Message Redundancy might not.
An echo would have to be at least two previous messages for it to affect all cases.
This is such that if a device misses the message from the previous device, and it is the following device, it can still acquire that message from the device following itself.
Furthermore this would also allow for devices to hear whether or not another device heard them, given their time-slots are withing the amounts of messages echoed.
With three devices this provides the same packet loss as before due to the aforementiond example.

\section{Effects of Packet Loss}
\section{Methods of Damage Control}
\subsection{Continous listening}
\subsection{Bitfield Time-Slot ID}

Effects of packet loss:
Corrupt data; depends on the application; delayed information; not syncing;

Methods of Recovery/Avoiding jams/desync:
Keep listening - worked earlier, no drift/desync issues;;; How would this handle if device is next?
Bitmask ID index in msg, allows devices to know if others missed their packet, works for dead device detection as well.
