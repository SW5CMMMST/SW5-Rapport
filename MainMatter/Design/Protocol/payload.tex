%Author: Marc
\subsection{Payload}
When developing for an Arduino one thing to always keep in mind is the limited amount of memory one has to work with.
Due to the limited amount of memory one would typically like to reduce the amount of memory used on data.
One workaround that can reduce at least the storing of some variables, is to include them into the payload.
This would in turn make the payload larger, another thing we would prefer to keep as small as possible.
We want to keep the payload small as With a bigger payload you increase the chances of packages being lost, as well as the time it takes to transmit the payload. 
Thus a balance between storing on the device or transmitting the data is sought after, one may however be more important than the in some cases.

\bigskip \noindent
The amount data useful in the payload largely depends on the complexity of the network itself, and the devices that make up the network.
In a simpler network where all devices serve the same purpose and the network is strongly connected, one payload may suit the entire network where no variables are irrelevant.
An example would be some sort of sensory network which observe one thing, like a fire alarm.
The only variable required here, aside from those part of the communication, is whether or not there is a fire.

This however only works if all devices serve the same purpose, as in the case of the fire alarm.
While such a solution may suffice for the previous case, if one intend on using the protocol for more complex networks, the payloads must be more general.

\bigskip \noindent
In the case of a network with a variety of devices both observers and actuators, these may require some data that is only relevant for their type of device.
Consider the case in which one wishes to turn on the light, depending on some sensory input.
When the sensor is triggered the payload should contain a command and an address of which device is to perform said command.

While this is needed in the mentioned scenario, the variables will also serve as garbage in payloads where no commands or specific target is required.
The more generally the payload is outfitted, the more garbage data will be contained in some amount of messages.

\bigskip \noindent
In an effort to remove garbage data several different payloads can be created.
Each payload should consist of the basic data required for communication such as who is sending, current slot and slot count.
Furthermore a variable, mode, determines the content of the rest of the payload.
The rest of the payload will differ depending on each configuration of mode.
This way no garbage data exists in the universal data that all payloads share, and with the mode variable all devices will know what data follows.

As an example if the mode is a command, all devices will know to expect a target and a command in the rest of the payload.
This also allows devices that may not be able to execute commands to ignore the remainder of the payload.