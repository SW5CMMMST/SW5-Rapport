\section{Summary}\label{cha:conclusion}
%ProblemStatement
Throughout this paper efforts to find a solution to the problem statement in \myref{sec:problemStatement} has been made.
In short the problem statement is to develop a protocol for multiple devices to communicate over a single radio frequency in a reliable and time-critical way, without jamming each other.
In order to develop a protocol for this particular problem a choice in hardware was made, which was to use the Arduino.

% Hardware
Using the Arduino platform limits the hardware and computational options although as the protocol would require neither intense computation or powerful hardware, the Arduino platform proved a good fit for the purpose of this paper and the resources available.
For real world applications unreliability is also a factor to consider and \gls{rf}-modules are no exception; furthermore as the problem statement states cost-efficiency, cheaper \gls{rf}-modules were chosen which in turn affects the quality.

% Problems: CCRC CCUC SCRC SCUC
\bigskip \noindent
Graph theory was used to split the problem domain into four separate issues in order to handle one sub-problem at a time; these were \acrfull{ccrc}, \acrfull{ccuc}, \acrfull{scrc}, and \acrfull{scuc}; which can all be further read about in \myref{chp:Problems}.

% TDMA  
The design of the protocol is based on the concept of \acrfull{tdma}, which is the de facto standard protocol for single-frequency multiple device communication.
UPPAAL was used to check if the protocol upheld different statements through queries.
The first protocol successfully arranged all devices into a network such that no two devices would transmit at the same time thus causing no jam.

% Multistartup & Results UPPAAL
The first design intended for \acrshort{ccrc} did not handle the activation of multiple devices at the same time, as shown by UPPAAL.
The multiple device activation problem was split into two sub-problems: \myref[name]{sec:MSI-CCRC} and \myref[name]{sec:simultaneous_connect}.
This was not in the original plan, and the time it took to design and implement this was underestimated, which caused time-constraints in the end.

% Results Implementation
The protocol was implemented to work on the Arduinos, and a test showed that if one Arduino had made a network, and three other Arduinos were turned on at the same time a network was successfully created.
It was also tested whether if two Arduinos turned on at the same time they would eventually connect to the same network, which was also the case.
The test data can be found in \myref{sec:ccrc_test}.

% Design CCUC
\bigskip \noindent
For the \acrshort{ccuc}-protocol not much had to change for it to work properly.
Using UPPAAL it was discovered that it was only creating and connecting the network which caused problems for \acrshort{ccuc}; once a network was created the protocol still had a high probability of success.
It was hypothesised that transmitting every payload twice could increase the probability of success.
UPPAAL was used to check this hypothesis, and it showed that this indeed was the case, as can be seen in \myref[name]{sec:uppaalccuc}.
% Results UPPAAL

% Optimisations
%   Delete, Multi-Subslots
% Time-constraints -> Not made all of the design
Further improvements for the protocol were designed in \myref[name]{mortenStuff}, but were not implemented due to time-constraints posed by the unforeseen time it took to solve problems posed by multiple device activation.
The improvements included a way to remove \enquote{dead} devices and defragmenting the frame.

% If more time Future works.
\bigskip \noindent
If time was not a factor the next step for the project would be implementing the designed solution for defragmentation followed altering to the protocol to also work for strongly connected networks.
A closer look upon strongly connected networks follows in \myref{futureWorks}.
