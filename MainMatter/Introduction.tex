%Author: Søren
%Author: Søren/Claus
\chapter{Introduction}
\vspace{-20pt}
Wired communication is considered the fastest and most reliable means of digital communication.
Wireless communication allows for a wider field of use cases, this is a trade-off between reliability and speed on one side and flexibility on the other. 
\citet{wirelessTradeoffs} describes this phenomenon in their article \emph{Some Optimisation Trade-offs in Wireless Network Coding}.

Devices which use wireless communication have influenced the world in the last couple of years, with smartphones becoming more and more common to most people in the western world, according to \citet{2013-SmartPhoneUse}.
However there exists plenty of uses for wireless technology other than just being used in mobile phones, tablets, and laptops, it can also be implemented in smaller embedded systems\footnote{In this project \enquote{embedded systems} is used to describe a category of computational units, which have limited resources and are relatively small in size. } in everyday use.
These system are often referred to as smart-systems.

An example of these smaller embedded systems could be modern fire alarms for larger buildings, where the signal of a fire in one wing of a building would be transmitted throughout the entire building.
If this system was developed with a wired network, it would require a lot of setup to begin with 
and adding more alarms to the network is a bigger hurdle for wired solutions. 
For devices such as fire alarms this is less of a problem as new units probably do not come often, but other uses, such as home automation, where new devices might enter and leave the network at any time this problem might arise more often and thus pose a bigger challenge.

This could be solved by using Wi-Fi networks.
Wi-Fi provides decades of research in communications which allows them to transfer lots of data reasonably fast.
For embedded systems this might be excessive depending on the system in question.
The fire alarm, for instance, would only need to send a few bits of data, so a Wi-Fi solution would require more power but still only use a fraction of the features available.

\newpage
\section{Problem Statement}\label{sec:problemStatement}
For this project, the focus is going to be on a solution which can be modified for a variety of use cases, while being just enough to solve the problem.
In other words; a cost efficient solution to the problem, and it is believed that there might be enough power in radio frequencies modules that even when using only one frequency, a communications network can still be established and maintained.
This can be described as the following problem which will be explored throughout this paper.

\medskip
{\addtolength{\leftskip}{10mm}\addtolength{\rightskip}{10mm}\noindent\hrulefill\it

\noindent How can a network of devices with radio transceivers of a single frequency communicate, such that any devices can send messages to other devices in the network, in a reliable and time-critical way?

\noindent\hrulefill

}

\bigskip \noindent
The following part of the paper will describe and analyse the problem defined above, which concerns the difficulties of having many embedded systems communicate on the same radio frequency.
Furthermore some possible aspects which could be used to solve the problem including hardware, software, and techniques will be introduced and discussed.
It could make sense to use more frequencies to communicate with so communication could be done on many channels at the same time.
But, this is not a solution as it postpones the problem since at some point we will run out of frequencies. 
So, the problem of solving how many devices communicate on the same frequency is seen as an important issue to explore.