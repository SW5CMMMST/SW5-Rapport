%Author: Søren
\chapter{Introduction}
\vspace{-20pt}
Wired communication is considered the fastest and  most reliably mean of digital communication.
Wireless communication allows for a wider field of use cases, this is a trade-off between reliability and speed on the one side and flexibility on the other. \cite{wirelessTradeoffs}

Devices which utilise wireless communication have heavily influenced the world in the last couple of years, with smartphones becoming more and more common to most people in the western world. \cite{2013-SmartPhoneUse}
However, there exists many more uses for wireless technology than just being used in mobile phones, tablets and laptops, it can also be implemented in smaller embedded systems in every day use. 
\footnote{In this project \textit{embedded systems} is used to describe a category of computational units, which have limited resources and are relatively small in size. 
E.g. microcontroller based devices, such as the Arduino}

%An example of these smaller embedded systems could be wireless temperature sensors or other sensors, which need a placement unsuitable for a wired connection.

%These wireless temperature sensors can work by having a sensor measure the temperature and a transmitter which transmits the temperature via a wireless radio.
%The receiver is then listening to the frequency of the transmitter waiting for the transmission to be received.
%When transmitting data either wireless or wired, the data can be corrupted in the transmission, due to interfering signals or general noise.
%To account for the corruption a system can implement error detection and correction.
%A compromise would make the data transfer more reliable for a trade off in speed.

An example of these smaller embedded systems could be modern fire alarms for larger buildings.
Currently a connected fire alarm system such as those found in office buildings, require wire to run between the individual units.
This could be a problem for older buildings where places to hide wires for this system was not though into the design of the building.

That is where a system of wireless devices could be useful. 
However such a system is in competition with its predecessor, and the wired technology has the edge of being reliable and fast once it is set up. 

It depends on the system, whether a fast connection is more important than a reliable one.
For example it is important for an alarm system, which uses multiple devices e.g. fire alarms, to be reliable.
This is required to guarantee that all devices will receive a message when one alarm goes off;
As this would be an important aspect of such a system.

%The example concerning the temperature sensor was described with one transmitter and one receiver, but what opportunities could arise using more than just one of each?

\bigskip

Using multiple embedded systems each fitted with a transmitter and a receiver, the embedded systems can use a single frequency radio, which is cheap, to communicate with one another, without any of the devices explicitly controlling the communication.
This could possibly be used in home automation to control the lights, or to control the ventilation system in a building etc.
