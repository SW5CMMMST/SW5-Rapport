\makeatletter
% code borrowed from Andrew Stacey; See
% http://tex.stackexchange.com/a/50054/3954
\tikzset{%
  remember picture with id/.style={%
    remember picture,
    overlay,
    save picture id=#1,
  },
  save picture id/.code={%
    \edef\pgf@temp{#1}%
    \immediate\write\pgfutil@auxout{%
      \noexpand\savepointas{\pgf@temp}{\pgfpictureid}}%
  },
  if picture id/.code args={#1#2#3}{%
    \@ifundefined{save@pt@#1}{%
      \pgfkeysalso{#3}%
    }{
      \pgfkeysalso{#2}%
    }
  }
}

\def\savepointas#1#2{%
  \expandafter\gdef\csname save@pt@#1\endcsname{#2}%
}

\def\tmk@labeldef#1,#2\@nil{%
  \def\tmk@label{#1}%
  \def\tmk@def{#2}%
}

\tikzdeclarecoordinatesystem{pic}{%
  \pgfutil@in@,{#1}%
  \ifpgfutil@in@%
    \tmk@labeldef#1\@nil
  \else
    \tmk@labeldef#1,(0pt,0pt)\@nil
  \fi
  \@ifundefined{save@pt@\tmk@label}{%
    \tikz@scan@one@point\pgfutil@firstofone\tmk@def
  }{%
  \pgfsys@getposition{\csname save@pt@\tmk@label\endcsname}\save@orig@pic%
  \pgfsys@getposition{\pgfpictureid}\save@this@pic%
  \pgf@process{\pgfpointorigin\save@this@pic}%
  \pgf@xa=\pgf@x
  \pgf@ya=\pgf@y
  \pgf@process{\pgfpointorigin\save@orig@pic}%
  \advance\pgf@x by -\pgf@xa
  \advance\pgf@y by -\pgf@ya
  }%
}

% end of Andrew's code

\newlength\AlgIndent
\setlength\AlgIndent{0pt}
% main command to draw the colored background
\newcounter{mymark}
\newcommand\ColorLine{%
  \stepcounter{mymark}%
  \tikz[remember picture with id=mark-\themymark,overlay] {;}%
  \begin{tikzpicture}[remember picture,overlay]%
    \filldraw[YellowGreen!40]%
   let \p1=(pic cs:mark-\themymark), 
   \p2=(current page.east)  in 
   ([xshift=-\ALG@thistlm-0.3em,yshift=-0.7ex]0,\y1)  rectangle ++(\linewidth+\AlgIndent,\baselineskip);
  \end{tikzpicture}%
}%

% colored loops and declarations
\makeatletter
\algnewcommand\CRequire{\item[\ColorLine\algorithmicrequire]}%
\algnewcommand\CEnsure{\item[\ColorLine\algorithmicensure]}%
\algnewcommand\CState{\State\ColorLine}%
\algnewcommand\CStatex{\Statex\ColorLine}%

\algblockdefx[Procedure]{CProcedure}{CEndProcedure}%
   [2]{\ColorLine\algorithmicprocedure\ \textsc{#1}(#2)}%
   {\ColorLine\algorithmicend\ \algorithmicprocedure}%
\algblockdefx[While]{CWhile}{CEndWhile}%
   [1]{\ColorLine\algorithmicwhile\ #1\ \algorithmicdo}%
   {\ColorLine\algorithmicend\ \algorithmicwhile}%
\algblockdefx[For]{CFor}{CEndFor}%
   [1]{\ColorLine\algorithmicfor\ #1\ \algorithmicdo}%
   {\algorithmicend\ \algorithmicfor}%
\algblockdefx[Loop]{CLoop}{CEndLoop}%
   {\ColorLine\algorithmicloop}%
   {\ColorLine\algorithmicend\ \algorithmicloop}%
\algblockdefx[Repeat]{CRepeat}{CUntil}%
   {\ColorLine\algorithmicrepeat}%
   [1]{\ColorLine\algorithmicuntil\ #1}%
\algblockdefx[If]{CIf}{CEndIf}%
   [1]{\ColorLine\algorithmicif\ #1\ \algorithmicthen}%
   {\ColorLine\algorithmicend\ \algorithmicif}%
\algdef{C}[If]{If}{CElsIf}%
   [1]{\ColorLine\algorithmicelse\ \algorithmicif\ #1\ \algorithmicthen}%
\algdef{Ce}[Else]{If}{CElse}{CEndIf}%
   {\ColorLine\algorithmicelse}%
\makeatother