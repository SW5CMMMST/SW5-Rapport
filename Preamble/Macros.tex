\newenvironment{enumberate}{\begin{enumerate}[label=\itshape \arabic*\upshape)]}{\end{enumerate}}
\newenvironment{eletterate}{\begin{enumerate}[label=\itshape \alph*\upshape)]}{\end{enumerate}}

\newcommand{\filefigure}[3]{
    \begin{figure}[H]
        \centering \footnotesize
        \input{#1}
        \caption{#2}
        \label{fig:#3}
    \end{figure}
}

\newcommand{\figur}[4]{
    \begin{figure}[H]
        \centering
        \includegraphics[width=#2\textwidth]{Figures/#1}
        \caption{#3}
        \label{fig:#4}
    \end{figure}
}

\newcommand{\wrapfig}[5][5=o]{
    \begin{wrapfigure}{#5}{#2\textwidth}
        \centering
        \includegraphics[width=(#2\textwidth)-10pt]{Figures/#1}
        \caption{#3}
        \label{fig:#4}
    \end{wrapfigure}
}

\newcommand{\tikzfigure}[3]{
    \filefigure{Figures/TikZ/#1}{#2}{#3}
}

\newcommand{\code}[2][c]{
    \begin{figure}[H]
        \lstinputlisting[caption=#2, escapechar=, style=custom#1]{CodeExamples/#2}
    \end{figure}
}

\newcommandx{\unsure}[2][1=]{
    \todo[linecolor=red,backgroundcolor=red!25,bordercolor=red,#1]{#2}
}

\newcommandx{\change}[2][1=]{
    \todo[linecolor=blue,backgroundcolor=blue!25,bordercolor=blue,#1]{#2}
}

\newcommandx{\info}[2][1=]{
    \todo[linecolor=OliveGreen,backgroundcolor=OliveGreen!25,bordercolor=OliveGreen,#1]{#2}
}

\newcommandx{\improvement}[2][1=]{
    \todo[linecolor=Plum,backgroundcolor=Plum!25,bordercolor=Plum,#1]{#2}
}

\newcommandx{\thiswillnotshow}[2][1=]{\todo[disable,#1]{#2}}

\renewcommand*{\CustomAcronymFields}{%
  name={\the\glsshorttok},% name is abbreviated form
  description={\the\glslongtok},% description is long form
  first={\noexpand\textbf{\the\glslongtok\space(\the\glsshorttok)}},%
  firstplural={\noexpand\textbf{\the\glslongtok\noexpand\acrpluralsuffix\space(\the\glsshorttok)}},%
  text={\the\glsshorttok},%
  plural={\the\glsshorttok\noexpand\acrpluralsuffix}%
}

\SetCustomStyle

\newcommand{\TITLE}{Timely Wireless Arduino Communication}

\makeatletter
\def\ifdraft{\ifdim\overfullrule>\z@
    \expandafter\@firstoftwo\else\expandafter\@secondoftwo\fi}
\makeatother  

\newcommand\titlepagedecoration{%
\begin{tikzpicture}[remember picture,overlay,shorten >= -10pt]

\coordinate (aux1) at ([yshift=-15pt]current page.north east);
\coordinate (aux2) at ([yshift=-410pt]current page.north east);
\coordinate (aux3) at ([xshift=-4.5cm]current page.north east);
\coordinate (aux4) at ([yshift=-150pt]current page.north east);

\begin{scope}[aaublue!40,line width=12pt,rounded corners=12pt]
\draw
  (aux1) -- coordinate (a)
  ++(225:5) --
  ++(-45:5.1) coordinate (b);
\draw[shorten <= -10pt]
  (aux3) --
  (a) --
  (aux1);
\draw[opacity=0.6,aaublue,shorten <= -10pt]
  (b) --
  ++(225:2.2) --
  ++(-45:2.2);
\end{scope}
\draw[aaublue,line width=8pt,rounded corners=8pt,shorten <= -10pt]
  (aux4) --
  ++(225:0.8) --
  ++(-45:0.8);
\begin{scope}[aaublue!70,line width=6pt,rounded corners=8pt]
\draw[shorten <= -10pt]
  (aux2) --
  ++(225:3) coordinate[pos=0.45] (c) --
  ++(-45:3.1);
\draw
  (aux2) --
  (c) --
  ++(135:2.5) --
  ++(45:2.5) --
  ++(-45:2.5) coordinate[pos=0.3] (d);   
\draw 
  (d) -- +(45:1);
\end{scope}
\end{tikzpicture}%
}